\documentclass[12pt]{article}%,leqno
\usepackage{cmap}
\usepackage[utf8x]{inputenc}
\usepackage[russian]{babel}
\usepackage[T2A]{fontenc}
\usepackage{amsmath}
\usepackage{amssymb}
\usepackage{amsthm}
\usepackage{graphicx}
\usepackage{tikz}
\usetikzlibrary{automata,shapes,arrows,positioning,calc}
\usepackage{clrscode}
\usepackage{sectsty}
\usepackage{verbatim}
\usepackage{ifthen}
\usepackage{forloop}
\usepackage{calc}
\usepackage{multicol}
\usepackage{indentfirst}

%\sectionfont{\large}

\usepackage[urlbordercolor={1 1 1}]{hyperref}
%colorlinks,urlcolor=black

% \topmargin=0cm
% %\headsep=10pt
% \oddsidemargin=0cm
% \textheight=21cm
% \textwidth=17cm

\usepackage[vscale=0.75,hscale=0.75,centering]{geometry}

\usepackage{fancyhdr}
\pagestyle{fancy}
\lfoot{\footnotesize ЮФУ, Мехмат}
%\rfoot{\footnotesize А.~Э.~Маевский}
\cfoot{--- \thepage{} ---}
\lhead{\itshape Методы защиты информации}
\rhead{}

\parindent=1.25cm

\tolerance=1000

%\renewcommand{\thesubsubsection}{\arabic{\subsection}}
%\arabic{\subsubsection}
\renewcommand{\emptyset}{\varnothing}
\renewcommand{\leq}{\leqslant}
\newcommand{\es}{\emptyset}
\newcommand{\N}{\ensuremath{\mathbb N}}
\newcommand{\NO}{\ensuremath{\mathbb N_0}}
\DeclareMathOperator{\RE}{RE}
\DeclareMathOperator{\Gen}{Gen}
\DeclareMathOperator{\Reach}{Reach}
\renewcommand{\qed}{\hfill \vrule width1.5ex height1.5ex depth0pt}
\newcommand{\pseudoqed}{\hfill\Box}
\newcommand{\HW}{\ensuremath{{\otimes}}}
\newcommand{\coleq}{\ensuremath{\mathrel{\mathop:}=}}
\renewcommand{\labelenumi}{(\theenumi)}
\newcommand{\InsertList}[1]{%
\input{listok#1}
}
\newcommand{\nspace}{\hspace{0pt}}
\newcommand{\nbdash}{\nobreakdash-\nspace}
\newcommand{\ul}{\underline}

\newtheorem{Thm}{Теорема}[section]
\renewcommand{\theThm}{\arabic{Thm}}

\theoremstyle{remark}
\newtheorem*{SketchOfProof}{Набросок доказательства}

\theoremstyle{definition}
\newtheorem{Ex}{Пример}
\newtheorem{Exec}{Упражнение}
\newtheorem*{Proposition}{Утверждение}
\newtheorem*{Remark}{Замечание}
\newtheorem{NumRemark}{Замечание}
\newtheorem*{Algo}{Алгоритм}
\newtheorem{NumAlgo}{Алгоритм}
\newtheorem{Def}{Определение}[section]
\renewcommand{\theDef}{\arabic{Def}}
\newtheorem*{Task}{Задача}

\title{Методы защиты информации}
\author{А.~Э.~Маевский}
\date{}

\begin{document}
\maketitle
\thispagestyle{fancy}

\section{Методы экспоненцирования}

\subsection{Экспоненцирование с фиксированной экспонентой}

\begin{Def}
\emph{Аддитивной цепочкой, вычисляющей число $n$}, называются две
последовательности, $v=(v_0,\ldots,v_s)$ — натуральных чисел и $w=(w_1, \ldots, w_s)$ — пар
натуральных чисел , таких что:
\begin{gather*}
v_0 = 1, \: v_s = n\\
\forall i \in [1, s]\colon v_i = v_j + v_k, \quad
0 \leqslant j, k \leqslant i-1;\\
 w_i = (j, k).
\end{gather*}
\end{Def}

\begin{Def}
А.Ц. называется \emph{звёздной}, если $j = i - 1$ для всех $i$.
\end{Def}

\begin{Ex}
$$v = (1, 2, 3, 6, 7, 14, 15)$$
$$w = ((0,0), (0,1), (22), (0,3), (4,4), (0,5)).$$
\end{Ex}

\begin{NumAlgo}\nspace\\Вход: $x \in G$, $n$, $v, w$ — А.Ц. для $n$.

Выход: $x^n$. 
\begin{enumerate}
  \item $x_1 \coleq x$;
  \item Для $i$ от $1$ до $s$ вычислить:
  $$x_i \coleq x_j \cdot x_k, \quad (j,k)=w_i.$$
  \item Вернуть $x$.    
\end{enumerate}
\end{NumAlgo}

\begin{Def}
\emph{Аддитивной-разностной цепочкой, вычисляющей число $n$} называется пара
последовательностей, удовлетворяющая аналогичным А.Ц. требованиям, за исключением того, что для $v_i$
требуется выполнение равенства $v_i = v_j + v_k$ либо $v_i = v_j - v_k$ для
некоторых $j, k$.
\end{Def}

\begin{Def}
Пусть $k, s \in \N$. \emph{Векторной аддитивной цепочкой, вычисляющей
последовательность $(n_0, \ldots, n_{k-1})$}, называются последовательности $v$
— $k$-мерных векторов неотр. целых чисел и $w$ — последовательность пар
неотрицательных целых, таких что:
\begin{multicols}{2}
\begin{gather*}
v_{-k+1} = (1, 0, \ldots ,0),\\
v_{-k+2} = (0, 1, \ldots ,0),\\
%\columnbreak
\ldots \\
v_{0} = (0, 0, \ldots ,1),
\end{gather*}
\columnbreak
\begin{gather*}
v_{i} = v_j + v_k, \quad (j, k) = w_i,\\
v_{s} = (n_0 \ldots ,n_{k-1}).
\end{gather*}
\end{multicols}
\end{Def}

\begin{Def}\emph{Словарём $D=\{d_i\}$ для числа $n$} называется набор целых
чисел, такой что:
$$n = \sum_{i=0}^k b_{i, d_i} d_i 2^i, \quad b_{i, d_i}\in\{0,1\},\; d_i \in
D.$$
\end{Def}

\begin{Exec}Убедиться, что словарём для $2^k$-ичного метода явл $D = 1, 3, 5,
\ldots, 2^k - 1$, а для знаковой модификации:
$D = \{ \pm 1,\pm 3, \ldots, \pm(2^k - 1) \}$.
\end{Exec}

\subsubsection{Метод Кунихиро—Ямамото}
Идея похожа на кодирование по Хаффману.
\begin{gather*}
n = (n_{l-1}, \ldots , n_0)_2,\\
p = \frac {\text{кол-во «0»}} {l},\\
q = 1 - p, \quad \hat{q} = \frac{1 - p}{2}.
\end{gather*}
Выбираем $k$ — количество листов будущего дерева минус $1$. Генерируем корень
дерева. Затем потомки генерируются рекурсивно: от узла с весом $w$ порождаются
два потомка с весами в вершинах $wp$, $wq$ и метками на дугах $0$ и $1$, соотв, для
беззнакового представления и три потомка с весами $wp$, $w\hat{q}$, $w\hat{q}$ и
метками на дугах $0$, $\bar{1}0$, $10$. Процедура повторяется для потомка с
наибольшим весом.

\ldots

Проходим по всем маршрутам от корня к листьям, собирая метки на дугах.
Удаляем trailing zeroes. Получаем словарь.

\begin{Ex}$n=587257$, $k = 4$.
\begin{gather*}
n = (10010000\bar{1}0\bar{1}00000\bar{1}001).\\
p = \frac{14}{20}. \quad \hat{q}=\frac 3 {20}.
\end{gather*}
\begin{center}
\pgfdeclareimage[height=8cm]{fig1}{fig1}%
\pgfuseimage{fig1}
\end{center}
\end{Ex}
\begin{gather*}
1000 \Rightarrow 1 \Rightarrow 1,\\
100\bar{1}0 \Rightarrow 100\bar{1} \Rightarrow 7,\\
10010 \Rightarrow 1001 \Rightarrow 9,\\
10\bar{1}0 \Rightarrow 10\bar{1} \Rightarrow 3,\\
1010 \Rightarrow 101 \Rightarrow 5. 
\end{gather*}
$$D = \{1,3,5,7,9\}.$$
$$n = 9 \cdot 2^16 - 5 \cdot 2^9 - 7.$$

\subsubsection{Метод Якоби}
Кроме Хаффмана есть другие хорошие методы сжатия, например, метод Лемпела—Зива.

$n = (n_{l-1}, \ldots , n_0)$. 

\begin{Ex}
Просматривая двоичное представление справа-налево, выделяем новые куски и
добавляем их в словарь.
\begin{gather*}
n=587257 = (\overline{10}\; \overline{00}\;
\overline{1111}\; \overline{010}\; \overline{111}\; \overline{11}\; 
\overline{10}\; \overline {0}\; \overline{1})_2, \\
\{1, 0, 10, 11, 111, 010, 1111, 00\},\\
\{1, 0, 2, 3, 7, 2, 15, 0\}, \quad \text{($0$ и двойки можно отбросить)}\\
D = \{ 1, 3, 7, 15 \}.
\end{gather*}
Теперь можно получить разложение $n$ по этому словарю:
$$
    n = 1+ 2^3 + 3\cdot 2^4 + 7 \cdot 2^6 + 1\cdot2^11
    + 15 \cdot 2^13 + 1\cdot 20.
$$
\end{Ex}

\subsection{Экспоненцирование с фиксированным показателем}
Первая идея: хранить «все» степени $x$.

\subsubsection{Метод Яо}
$$
    n = \sum_{i=0}^{l-1}n_i b_i, \quad n_i < h, 
$$
где $h$ — некоторое фиксированное число.
$$
    x^n = \prod_{i=0}^{l-1}x_i^{n_i}
        = \prod_{j = 1}^{h-1} \left ( \prod_{i, n_i = j} x_i \right )^j
$$
\begin{Ex}
$x_i = x^{b_i}$, $i \in [0, l-1]$. $x^{2989}$ — ? $h=4$, $b_i = 4^i$.
$$ 2989 =(232231)_4. $$
$$
    x^{2989} = x_0^1 x_2^3 x_3^2 x_3^2 x_4^3 x_5^2 = 
    (x_0)^1(x_2x_3x_5)^2 (x_1x_4)^3 =
    (x_1x_4)(x_1x_4x_2x_3x_5)(x_1x_4x_2x_3x_5x_0).
$$
\end{Ex}

\begin{NumAlgo}\nspace\\

Вход:

Выход:

\begin{enumerate}
  \item $y \coleq 1$, $u \coleq 1$, $j \coleq h - 1$.
  \item Пока $j \leqslant 1$ выполнять…
\end{enumerate}
\end{NumAlgo}
\subsubsection{Метод, основанный на алгоритме Евклида}
\begin{NumAlgo}\nspace\\

Вход: $x \in G$, $n = \sum n_i b_i$, $x_0 = x^{b_0}, \ldots , x_{l-1} =
x^{b_{l-1}}$.

Выход:

\begin{enumerate}
  \item Повторять:
    \begin{enumerate}
        \item Найти $M$: $n_M \leq n_i$, $i \in [0, l-1]$.
        \item Найти $N$ ($N \neq M$):
            $n_M \leqslant n_i$, $i \in [0, l-1] \setminus \{ M \}$.
        \item Если $n_N \neq 0$, то
        $$ 
            q \coleq \lfloor n_M / n_N \rfloor, \quad x_N \coleq x_M^q x_N, \quad 
            n_M \coleq n_M \mod n_N, 
        $$
        иначе прервать цикл.
    \end{enumerate}
    \item Вернуть $x_M$.
\end{enumerate}
\end{NumAlgo}

\begin{Ex}
$x_i = x^{b_i}$, $i \in [0, l-1]$. $x^{2989}$ — ? $h=4$, $b_i = 4^i$,
$ 2989 =(232231)_4$.
\begin{center}
\pgfdeclareimage[height=7cm]{fig2}{fig2}%
\pgfuseimage{fig2}
\end{center}
\end{Ex}

\subsubsection{Метод Пайпенджера (Лима—Ли)}
$n = (n_{l-1} \ldots n_0)_2$, $h \in [1, l]_{\N}$, $\lceil l / h \rceil$.
$$
\begin{matrix}
    a - 1 & \ldots & 1 & 0 \\
    \_ & \_ & \_ & \_ \\
    %\hdotsfor[0.01]{4}\\
    n_{a-1} & \ldots & n_1 & n_0 \\
    \hdotsfor{4}\\
    n_{ah-1} & \ldots & n_{ah - a +1} & n_{ah - a} \\ 
\end{matrix}
$$

$$
    G[j, i] = \left (\prod x^{i_s 2^{as}} \right )^{2^{jr}} \quad
    j \in [0, v-1], \; i = (i_{h-1}\ldots i_0) \in [0, 2^n - 1]. 
$$


$$
    x^n = \prod_{k=0}^{r-1} \left ( 
        \prod_{j=0}^{v-1} G[j, I(jr+k)] 
    \right )^{2^k} 
$$

$$
    I(s) = (n_{a(h - 1) + s} \ldots n_{a + s} n_s), \quad n = \ldots 
$$

\begin{NumAlgo}\nspace\\

Вход: $x \in G$, $n \in \N$, $h,a,v,r,G[j,i]$. 

Выход: $x^n \in G$.

\begin{enumerate}
  \item $y \coleq 1$.
  \item Для $k \coleq r - 1$ по $0$ выполнить:
    \begin{enumerate}
        \item $y \coleq y^2$
        \item Для $j \coleq v - 1$ по $0$ выполнить:
        $$ I \coleq \sum_{s=0}^{h-1} n_{as + jr + k} 2^s$$
        \item $y \coleq y \cdot G[j, I]$ 
    \end{enumerate}
    \item Вернуть $y$.
\end{enumerate}
\end{NumAlgo}
\begin{Remark}[о сложности алгоритма]
$a + r - 2$ умножений и $v(2^h - 1)$ сохранённых значений. Если возведение в
квадрат быстрое, $v = 1$.
\end{Remark}

\end{document}
