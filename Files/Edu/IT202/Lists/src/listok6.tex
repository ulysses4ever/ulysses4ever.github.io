\begin{center}
{\bfseries\Large \textsc{\Llabel{} \Lnum}: нормальная форма Хомского}
\end{center}

\begin{Def}Пусть дана КС-грамматика $G=(\Sigma, N, \mathcal P, S \in N)$.
Говорят, что символ $X \in \Sigma \cup N$ 
\begin{enumerate}
    \item \emph{полезный в $G$}, если
    $$
        \exists \alpha, \beta \in (\Sigma \cup N)^{\ast} \:
        \exists w \in \Sigma^* \colon \quad
        S \Rightarrow^{\ast}_G \alpha X \beta \Rightarrow^{\ast}_G w. 
    $$
    \item \emph{порождающий в $G$}, если
    $$
        \exists w \in \Sigma^* \colon \quad X \Rightarrow^{\ast}_G w 
    $$
    \item  \emph{достижимый в $G$}, если
    $$
        \exists \alpha, \beta \in (\Sigma \cup N)^{\ast} \colon \quad 
        S \Rightarrow^{\ast}_G \alpha X \beta
    $$
\end{enumerate}
\emph{Бесполезным} называется любой символ, не являющийся полезным.
\end{Def}

%\begin{Remark}
%Каждый полезный символ свялется достижимым и порождающим.
%\end{Remark}

\begin{Thm} 
Если к КС грамматике $G$ \emph{последовательно} применить два преобразования:
\begin{enumerate}
  \item удалить символы, не являющиеся порождающими,
  \item удалить символы, не являющиеся достижимыми,
\end{enumerate}
то будет получена грамматика, не содержащая бесполезных символов.
\end{Thm}

\begin{Remark}
Порядок действий в \textbf{теореме} существенен.
\end{Remark}

\begin{Task} 
Удалить бесполезные символы в грамматиках с продукциями:

\begin{tabular}{cc} 
    (1) $
        \begin{array}{l}
            S \to 0 \mid A,\\
            A \to AB,\\
            B \to 1;
        \end{array}
    $
        &
    (2) $
        %\qquad\qquad
        \begin{array}{l}
            S \to AB \mid CA,\\
            A \to a,\\
            B \to BC \mid AB,\\
            C \to aB \mid \varepsilon.
        \end{array} 
    $
\end{tabular}
\end{Task}

\begin{Def}[Хомский, 1959]
Говорят, что КС-грамматика $G=(\Sigma, N, \mathcal P, S \in N)$ находится в
\emph{нормальной форме Хомского (НФХ)}, если она не содержит бесполезных символов
и каждая продукция грамматики имеет один из двух видов:
\begin{enumerate}
\renewcommand{\labelenumi}{\theenumi)}
  \item $A \to a$,
  \item $A \to BC$, 
\end{enumerate}
где $A,B,C \in N$, $a \in \Sigma$.
\end{Def}
\pagebreak

\textbf{Схема приведения грамматики к НФХ:}
\begin{enumerate}
  \item удалить $\varepsilon$-продукции;
  \item удалить цепные продукции (продукции вида $A \to B$);
  \item удалить бесполезные символы;
  \item привести грамматику к НФХ, используя метод «разбиения слов на слоги».
\end{enumerate}

\begin{Remark}
Порядок действий в \textbf{схеме} существенен.
\end{Remark}

\begin{Def}
Нетерминал $A \in N$ КС-грамматики $G=(\Sigma, N, \mathcal P, S \in N)$
называется \emph{$\varepsilon$-порождающим}, если существует вывод
$A \Rightarrow^{\ast}_G \varepsilon$. 
\end{Def}

\begin{Task} 
Привести к нормальной форме Хомского грамматики с продукциями:

\begin{tabular}{c|c} 
    (1) $\begin{array}{l}
    S \to ASB \mid \varepsilon,\\
    A \to aAS \mid a,\\
    B \to SbS \mid A \mid bb;
    \end{array}$&
    (2) $\begin{array}{l}
    S \to 0A0 \mid 1B1 \mid BB,\\
    A \to C,\\
    B \to S \mid A,\\
    C \to S \mid \varepsilon;
    \end{array}$\\[\bigskipamount] \hline
    (3) $\begin{array}{l}
    S \to AAA \mid B,\\
    A \to aA \mid B,\\
    B \to \varepsilon;
    \end{array}$ &
    (4) $\begin{array}{l}
    S \to aAa \mid bBb \mid \varepsilon,\\
    A \to C \mid a,\\
    B \to C \mid b,\\
    C \to CDE \mid \varepsilon,\\
    D \to A \mid B \mid ab.\\
    \end{array}$
\end{tabular}
\end{Task}