\twocolumn[\begin{center}
{\bfseries\Large \textsc{\Llabel{} \Lnum}}
\end{center}]

\textbf{1. Построение недетерминированных автоматов}\\
Построить автомат, распознающий 
\begin{enumerate}
  \item язык над $\{0,1\}$ из слов, заканчивающихся на $01$;
  \item язык, представляющий собой десятичную запись чисел, делящихся на $4$;
  \item язык над $\{a, b\}$, заданный регулярным выражением $(ab + aba)^\ast$; 
  \item язык над $\{0, 1, \ldots , 9\}$ из слов, в которых последняя цифра
  встречается ещё где-то в них;
  \item язык над $\{0, 1, \ldots , 9\}$ из слов, в которых последняя цифра
  больше нигде в них не встречается;
  \item язык над $\{0,1\}$ из слов, в которых содержится два $0$, разделённых
  символами, количество которых кратно $4$ (нуль символов также считать
  кратными четырём).
\end{enumerate}
\pagebreak

\textbf{2. Детерминизация конечных автоматов}
\begin{Thm}[о детерминизации конечных автоматов, М.О.~Рабин — Д.~Скотт, 1959]
Пусть задан недетерминированный конечный автомат (НКА) $$\mathcal A = (Q,
\Sigma, \delta, q_0 \in Q, F \subset Q).$$ Определим по нему детерминированный
конечный автомат (ДКА), используя \emph{«конструкцию подмножеств» (subset 
construction или powerset construction)}:
$$\widehat{\mathcal A} = (\widehat Q = 2^Q, \Sigma, \hat\delta, \{q_0\}, 
\widehat F),$$
где:
\begin{align*} 
\widehat F = &\{ \Omega \in \widehat Q \mid \Omega \cap F \neq \emptyset
\};\\
\hat\delta(\Omega, a) = &\bigcup_{q\in\Omega} \delta(q,a).
\end{align*}
Тогда
$$L(\mathcal A) = L(\widehat{\mathcal A}).$$
\end{Thm}
\begin{proof}См. лекции или предложенную электронную литературу.
\end{proof}

В конструкции подмножеств происходит экспоненциальный рост числа состояний
автомата, который можно попытаться избежать, с помощью вычисления
лишь достижимых состояний ДКА, используя

\textbf{«Ленивое вычисление» подмножеств:}
\begin{description}
  \item[База] Состояние $\{q_0\}$ достижимо.
  \item[Индукция] Если множество состояний $S$ достижимо, тогда для каждого $a
  \in \Sigma$ достижимо $\hat\delta(S, a)$.
\end{description}

Провести детерминизацию следующих НКА:
\begin{enumerate}
  \item распознающего язык над $\{0,1\}$ из слов, заканчивающихся на $01$;
  \item распознающего язык, заданный регулярным выражением $(ab + aba)^\ast$; 
  \item 
     \begin{tabular}{r||c|c}
     & $0$ & $1$\\
     \hline\hline
     ${}\to p$ & $\{p, q\}$ & $\{p\}$\\
     $q$ & $\{r\}$ & $\{r\}$\\
     $r$ & $\{s\}$ & $\emptyset$\\
     \boxed{s} & $\{s\}$ & $\{s\}$
    \end{tabular}
  \item 
     \begin{tabular}{r||c|c}
     & $0$ & $1$\\
     \hline\hline
     ${}\to p$ & $\{q, s\}$ & $\{q\}$\\
     \boxed{q} & $\{r\}$ & $\{q, r\}$\\
     $r$ & $\{s\}$ & $\{p\}$\\
     \boxed{s} & $\emptyset$ & $\{p\}$
    \end{tabular}
  \item
     \begin{tabular}{r||c|c}
     & $0$ & $1$\\
     \hline\hline
     ${}\to p$ & $\{p, q\}$ & $\{p\}$\\
     $q$ & $\{r, s\}$ & $\{t\}$\\
     $r$ & $\{p, r\}$ & $\{t\}$\\
     \boxed{s} & $\emptyset$ & $\emptyset$\\
     \boxed{t} & $\emptyset$ & $\emptyset$
    \end{tabular}
\end{enumerate}
\pagebreak

\textbf{3. Темы, связанные с регулярными языками, не затронутые в курсе:}
\begin{enumerate}
  \item\label{eps-nfa} $\varepsilon$-НКА;
  \item построение конечных автоматов по регулярным выражениям
  (использует~\eqref{eps-nfa});
  \item минимизация конечных автоматов;
  \item алгоритмическая разрешимость и алгоритмическая сложность вопросов о
  формальных языках — \begin{quote}пустота, принадлежность,\\эквивалентность 
  \end{quote}— применительно к регулярным языкам.
\end{enumerate}