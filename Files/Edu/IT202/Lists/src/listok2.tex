\twocolumn[\begin{center}
{\bfseries\Large \textsc{\Llabel{} \Lnum}}
\end{center}]
\textbf{1. Системы линейных уравнений с регулярными коэффициентами}

Решить системы уравнений:
\begin{equation}
    \begin{cases}
    X_1 = aX_1 + aX_2;\\
    X_2 = bX_2 + b.
    \end{cases}
\end{equation} 

\begin{equation}
    \begin{cases}
        X_1 = 1X_1 + 0X_2 + \es X_3; \\
        X_2 = 1X_1 + \es X_2 + 0X_3; \\
        X_3 = 0X_1 + \es X_2 + 1X_3. 
    \end{cases}
\end{equation} 

\begin{equation}
    \begin{cases}
        X_1 = a^{\ast}X_1 + (a+b)^{\ast}X_2;\\
        X_2 = (a+b^{\ast})X_1 + aX_2 + b^\ast.
    \end{cases}
\end{equation}

\begin{equation}
    \begin{cases}
        X_1 = (a+b)X_1 + \es X_2 + a^{\ast}X_3; \\
        X_2 = \es X_1 + aX_2 + a^{\ast}; \\
        X_3 = b^{\ast}X_1 + \es X_2 + a^{\ast}X_3.
    \end{cases}
\end{equation} 

Написать регулярные выражения для языков, заданных грамматиками cо
следующими продукциями:
\begin{equation}
\begin{array}{l}
S \to 1A \mid 2S; \\
A \to 0B \mid 0S \mid 1A; \\
B \to 1C \mid 2C; \\
C \to \varepsilon \mid 1S \mid 2A.
\end{array}
\end{equation}

\begin{equation}
\begin{array}{l}
S \to 0A \mid 1S \mid \varepsilon ;\\
A \to 0B \mid 1A;\\
B \to 0S \mid 1B.
\end{array}
\end{equation}

\newpage
\textbf{2. Доказательство нерегулярности формальных языков}
\begin{Thm}[«Лемма о накачке» / «Лемма о разрастании» / «Pumping Lemma»,
И.~Бар-Хил\-лел~— М.~Пе\-лис~— Э.~Ша\-мир, 1961] Пусть $L$ — регулярный язык.
Тогда существует такая константа $n\in \mathbb N$, что для любого слова $w \in L$,
такого что $|w|\geqslant n$, существует такое разбиение $w=xyz$ слова $w$, что:
\begin{enumerate}
  \item $y \neq \varepsilon$;
  \item $|xy| \leqslant n$;
  \item $\{ xy^kz \mid k \geqslant 0\} \subset L$.
\end{enumerate} 
\end{Thm}
\begin{proof}
${\otimes}$ — найти в одной из предложенных электронных книг.
\end{proof}

Доказать, что следующие языки нерегулярны:
\begin{enumerate}
  \item $\{0^n1^n \mid n \in \mathbb N\}$;
  \item язык из всех слов $w \in \{0,1\}^{\ast}$, содержащих одинаковое
  количество $0$ и $1$;
  \item $\{w w \mid w \in \{0, 1\}^\ast \}$;
  \item $\{0^n1^m \mid n \leqslant m \}$;
  \item $\{1^p \mid \text{$p$ — простое} \}$;
  \item $\{0^n10^n \mid n \in \mathbb N\}$;
  \item $\{1^{n^2} | n \in \mathbb N\}$;
  \item $\{1^{n!} | n \in \mathbb N\}$.
\end{enumerate}