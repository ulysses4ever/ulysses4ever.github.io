\twocolumn[\begin{center}
{\bfseries\Large \textsc{\Llabel{} \Lnum :} Нахождение языка конечного автомата}
\end{center}]

\textbf{Задача:} описать язык заданного конечного автомата регулярным
выражением.

\textbf{Способ 1.}
\begin{equation*} 
\begin{split}
\text{(Объ} &\text{ект реального мира}
\stackrel{0}{\mapsto})\\ 
&\text{конечный автомат}
\stackrel{1}{\mapsto}\\
&\text{граф переходов конечного автомата}
\stackrel{2}{\mapsto}\\
&\text{ПЛ-грамматика} 
\stackrel{3}{\mapsto}\\
&\text{система линейных уравнений}
\stackrel{4}{\mapsto}\\
&\text{решение системы.}
\end{split}
\end{equation*} 

\textbf{Способ 2. Исключение состояний.}
Метод исключения состояний подразумевает последовательное удаление вершин графа
переходов автомата, которое протоколируется с помощью записи на оставшихся дугах
регулярных выражений (можно считать, что в изначальном графе на дугах
простейшие регулярные выражения — однобуквенные).

\emph{Процедура исключения состояния $s$:} для каждых двух (необязательно
различных, но несовпадающих с $s$) состояний $p$ и $q$, таких что существует
фрагмент графа переходов автомата $p \xrightarrow{R_1} s \xrightarrow{R_2} q$,
где $R_1$, $R_2$ — некоторые регулярные выражения (метки дуг переходов),
прибавить к метке дуги $p \xrightarrow{R_3} q$ выражение $R_1R^{\ast}R_2$, где
$R$ это метка петли на вершине $s$ (если петля на $s$ и/или дуга $p \to q$
отсутствовали в исходном графе, то можно считать, что их метки равны
$\varnothing$) — таким образом получена дуга с меткой: $p \xrightarrow{R_3 +
R_1R^{\ast}R_2} q$. Удалить все просмотренные дуги $p \to s$ и $s \to q$,
инцидентные вершине $s$. После этого $s$ стала изолированной либо имеются только
вхоящие или только исходящие из неё дуги. Вершину $s$ можно удалить (с
входящими или выходящими из неё дугами, если таковые имеются).

\renewcommand{\labelenumi}{\theenumi .}
\emph{Алгоритм нахождения языка автомата методом исключения состояний.}
\begin{enumerate}
  \item\label{Final} Для каждого финального состояния $q\in F$, отличного от
  начального $q_0$, применять процедуру исключения состояний до тех пор, пока не останутся
  две вершины: $q_0$ и $q$. В результате получится подобный автомат:\\ 
  \begin{tikzpicture}[node distance=3cm,auto,%
                        every state/.style={thick},>=stealth']
    \node[state,initial,initial text=] (q_0) {$q_0$};
    \node[state,accepting] (q) [right of=q_0] {$q$};

    \path[->] 
        (q_0) edge [loop above] node 
            {$R$} ()
        (q_0) edge[bend left=40] node 
            {$S$} (q)
        (q) edge [loop above] node 
            {$U$} 
            ()
        (q) edge[bend left=40] node 
            {$T$} (q_0);
\end{tikzpicture}\\
Допускаемый им язык описывается так: $$(R+SU^{\ast}T)^{\ast}SU^{\ast}.$$

\item\label{InitIsFinal} Если начальное состояние $q_0$ является финальным
($q_0\in F$), применять процедуру исключения состояний, пока не останется
единственная вершина $q_0$. В результате получится подобный автомат:\\
\begin{tikzpicture}[node distance=3cm,auto,%
                    every state/.style={thick},>=stealth'] 
    \node[state,initial,initial text=,accepting] (q_0) {$q_0$};
    
    \path[->] 
        (q_0) edge [loop above] node 
            {$R$} ();
\end{tikzpicture}\\
Допускаемый им язык описывается так:$$R^{\ast}.$$

\item Язык исходного автомата определяется как сумма всех регулярных выражений,
полученных на шагах \eqref{Final}–\eqref{InitIsFinal}.

\end{enumerate}

\renewcommand{\labelenumi}{(\theenumi)}
Решите поставленную задачу каждым из двух способов для конечного автомата:
\begin{enumerate}
  \item моделирующего лампочку;
  \item моделирующего лампочку, которая сгорает на третьем включении, считая
  финальными состояния, когда лампочка выключена, но ещё не сгорела;
  \item 
    \begin{tabular}{r||c|c}
     & $0$ & $1$\\
     \hline\hline
     ${}\to q_0$ & $q_1$ & $q_0$\\
     $q_1$ & $q_2$ & $q_0$\\
     \boxed{q_2} & $q_2$ & $q_1$
    \end{tabular}
  \item 
    \begin{tabular}{r||c|c}
     & $0$ & $1$\\
     \hline\hline
     ${}\to q_0$ & $q_1$ & $q_2$\\
     $q_1$ & $q_0$ & $q_2$\\
     \boxed{q_2} & $q_1$ & $q_0$
    \end{tabular}
  \item 
    \begin{tabular}{r||c|c}
     & $0$ & $1$\\
     \hline\hline
     ${}\to\boxed{p}$ & $s$ & $p$\\
     $q$ & $p$ & $s$\\
     $r$ & $r$ & $q$\\
     $s$ & $q$ & $r$
    \end{tabular}
\end{enumerate}