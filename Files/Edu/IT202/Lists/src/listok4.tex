\begin{center}
{\bfseries\Large \textsc{Листок \Lnum}}
\end{center}

\begin{Thm}Класс регулярных языков замкнут относительно операций: объединения,
конкатенации, итерации, дополнения, пересечения, разности.
\end{Thm}
\begin{proof}Для регулярных языков $L_1$, $L_2$ рассмотрим описывающие
их регулярные выражения $\RE(L_1)$, $\RE(L_2)$. Язык объединения
(соответственно, конкатенации, итерации) описывается выражением $\RE(L_1) +
\RE(L_2)$ (соответственно, $\RE(L_1)\RE(L_2)$, $\RE(L_1)^\ast$), а значит,
регулярен.

Для регулярного языка $L$ построим распознающий его
детерминированный\footnote{Сделанное дальше утверждение, однако, неверно для
недетерминированного конечного автомата — постройте соответствующий контрпример.
\HW} конечный автомат $\mathcal A(L) = (Q, \Sigma, \delta, q_0, F)$. Легко
проверить, что автомат $\widetilde{\mathcal A}(L) = (Q, \Sigma, \delta, q_0, Q
\setminus F)$ допускает дополнение $\overline L = \Sigma^\ast \setminus L$ языка
$L$ \HW, которое является, таким образом, регулярным языком.

Поскольку справедливо $L_1 \cap L_2 = \overline{\overline L_1 \cup \overline
L_2}$, язык $L_1 \cap L_2$ регулярен по доказанному выше\footnote{Имеется
более полезная конструкция, которая строит ДКА, допускающий $L_1 \cap L_2$,
внутри которого «параллельно» работают $\mathcal A(L_1)$ и $\mathcal A(L_2)$ — попытайтесь придумать её или
разберите Теорему~4.8 раздела~4.2.1 книги Хопкрофта и др. \emph{Введение в
теорию автоматов\ldots}~\HW}. Аналогичное можно заключить из равенства $L_1
\setminus L_2 = L_1 \cap \overline{L_2}$.
\end{proof}

Доказать (не)регулярность:
\begin{enumerate}
  \item $\{0^n 1^m \mid n \neq m\}$;
  \item $\{ w \in \{0, 1\}^\ast \mid
  \text{$w$ содержит одинаковое число 0 и 1}\}$;
  \item языка из слов $w \in \{0,\ldots, 9\}^\ast$, которые
  являются десятичной записью чисел, делящихся на 2 или на 3, без
  лишних лидирующих нулей;
  \item $\{ 0^n 1^m 2^{n-m} \mid n \geqslant m \}$;
  \item $\{ a^n b a^m b a^{n+m} \mid n, m \in \N \}$.
\end{enumerate}

\textbf{Контрпример к достаточности леммы о накачке}\\
Для языка
$$L = \{a^i b^j c^k \mid i, j, k \in \N_0 \wedge (i=1 \Rightarrow j = k)\}$$
покажите, что
\begin{enumerate}
    \item $L$ нерегулярен, 
    \item $L$ удовлетворяет условию леммы о накачке. 
\end{enumerate}
  