\documentclass[12pt]{article}
\usepackage[utf8x]{inputenc}
\usepackage[english, russian]{babel}
\usepackage{ucs}
\usepackage[pdftex,backref=page,linkcolor=blue,draft=false,unicode, bookmarks=true,bookmarksnumbered=false,bookmarksopen=false,colorlinks=true]{hyperref}
\usepackage{cmap}
\usepackage{indentfirst}
\usepackage{amsmath}
\usepackage{amsfonts}
\usepackage{amssymb}
\usepackage{amsthm}
%\usepackage{url}
%\usepackage{listings}
\usepackage{tikz}
\usetikzlibrary{automata,shapes,arrows,positioning,calc}

%\usepackage{geometry}
%\geometry{left=2cm}
%\geometry{right=1.5cm}
%\geometry{top=1cm}
%\geometry{bottom=2cm}
\usepackage{sfedu-automata}

\begin{document}
\subsection*{Пример: автомат для языка $L_{ww^R}=\{ww^R \mid w\in \{a,b\}^*\}$}
Замечание: $w^R$ означает: «слово $w$, записанное задом—наперёд». Т.~о., $L_{ww^R}$ это язык палиндромов над алфавитом ${a,b}$.


Известно, что автомат построить можно, т.~к. ранее была построена КС-грамматика для $L_{ww^R}$ и приведено утверждение о совпадении МП- и КС-языков. В общем случае, понять, является ли язык автоматным сложно. Аналогичный вопрос для КС-языков может быть решён с помощью т.~н. \emph{леммы о накачке (разрастании)}. Она устанавливает некоторое свойство для скорости роста длины слов в любом КС-языке, опровергнув которое, можно утверждать, что имеем дело с не КС-языком. Таков, например, язык $\{a^nb^nc^n|n\geqslant0\}$.

\emph{Метод}. Автомат имеет два «рабочих» состояния, $q_0, q_1$: в первом он читает часть слова на ленте, относящуюся к $w$, а во втором — к $w^R$. В первом состоянии он каждую прочитанную букву кладёт на стек, чтобы потом можно было сравнить её с соответствующей буквой $w^R$. В завершающее состояние $q_2$ перейдём после $q_1$ при опустошении стека (при появлении маркера дна стека). Будем использовать графическое представление автомата.

\begin{tikzpicture}[node distance=3cm,auto,every state/.style={draw=blue!50,very thick,fill=blue!20}]
	\node[state,initial] (q_0) {$q_0$};
	\node[state] (q_1) [right of=q_0] {$q_1$};
	\node[state,accepting] (q_2) [right of=q_1] {$q_2$};
	
	\path[->] 
		(q_0) edge [loop below] node 
			{$\begin{array}{l}a, Z_0/aZ_0\\b, Z_0/bZ_0\\a, a/aa\\a, b/ba\\
			b, a/ba\\b, b/bb\end{array}$} ()
		(q_0) edge node 
			{$\begin{array}{l}\varepsilon, a/a\\\varepsilon, b/b\\
			\varepsilon, Z_0/Z_0\end{array}$} 
			(q_1)
		(q_1) edge [loop below] node 
			{$\begin{array}{l}a, a/\varepsilon\\b, b/\varepsilon\end{array}$} 
			() %(q_2) 
		(q_1) edge node {$\varepsilon, Z_0/Z_0$} (q_2);
\end{tikzpicture}

Опишем автомат с помощью функции перехода:
\[\begin{array}{l}
\delta(q_0,a, Z_0) = (q_0,aZ_0) \text{ — первая сверху метка петли на $q_0$}\\
\delta(q_0,\varepsilon,a) = (q_1, a) \text{ — первая сверху метка дуги $q_0q_1$}\\
\delta(q_1,a, a) = (q_1,\varepsilon) \text{ — первая сверху метка петли на $q_1$}\\
\ldots
\end{array}\]

Автомат имеет существенно недетерминированную природу. Например, находясь в $q_0$, он на каждом такте пытается угадать, достигнута ли середина слова: можно сказать, что автомат создаёт своего полного клона (с такой же конфигурацией) и каждый из двух идёт по своей дуге. Если кто-то один из множества клонов дойдёт до $q_2$, прочтя всё слово, то это слово будет допущено. Рассмотрим это на примере слова $abba$.
\begin{enumerate}
	\item Неудачный клон:
	$$(q_0, abba, Z_0) \vdash (q_1,abba,Z_0)\vdash(q_2,abba,Z_0)$$
	\item Удачный клон:
	\begin{multline*}
	(q_0, abba, Z_0)\vdash(q_0,bba,aZ_0)\vdash(q_0,ba,baZ_0)\vdash
		(q_1,a,aZ_0)\vdash\\\vdash(q_1,\varepsilon,Z_0)\vdash(q_2,\varepsilon,Z_0).
	\end{multline*}
\end{enumerate}

В определение МП-автомата была заложена недетерминированность: 
$$\delta: Q\times(\Sigma\cup\{\varepsilon\})\times\Gamma\to \mathbf{P_{fin}(}Q\times\Gamma^*\mathbf{)}.$$
Можно определить класс ДМП детерминированных МП-автоматов, которые совпадают с (Н)МП-автоматами во всём, кроме вида функции переходов:
$$\delta: Q\times(\Sigma\cup\{\varepsilon\})\times\Gamma\to Q\times\Gamma^*\mathbf,$$
причём, для всех $q\in Q$ выполнено условие:
$$\text{ \textbf{if} } \exists a\in\Sigma\; \exists Z\in\Gamma \mid 
\delta(q,a,Z)\neq\emptyset \text{ \textbf{then} } 
\delta(q,\varepsilon,Z)=\emptyset.$$

Для конечных автоматов выполнялось равенство $\LNFA=\LDFA$. Оно доказывалось с помощью детерминизации НКА-автомата. Функция переходов НКА выглядела так: $Q\times\Sigma\to P(Q)$, в строящемся ДКА в качестве множества состояний $Q'$ бралось $P(Q)$. Ясно, что эта конструкция не подойдёт для МП-автоматов, так как $\mathbf {P_{fin}}(Q\times\Gamma^*)$ не является конечным множеством. Оказывается, другую конструкцию придумать невозможно.

\begin{theorem} $\LPDA \neq \LDPDA$. \end{theorem}
\begin{stm} $L_{ww^R}\not\in\LDPDA$\end {stm}

Интуитивно понятно, что на практике желательно иметь дело с детерминированными объектами и алгоритмами. Например, в теории синтаксического анализа и компиляции используется класс языков, порождаемых т.н. $\LR(k)$-грамматиками, причём
$$ L(\text{КС})\varsupsetneq L(\LR(k)) = \LDPDA \varsubsetneq\LPDA.$$

\subsection*{Историческая справка}
\begin{itemize}
	\item МП-автоматы впервые были введены в работах Эттингера (Oet\-tin\-ger), 1961 и Шютценберже (Schutzenberger), 1963.
	\item Эквивалентность МПА- и КС-языков установлена Хомским (Chom\-sky) во внутреннем докладе MIT, 1961, и впервые опубликована в печати Эви (Evey), 1963.
	\item ДМПА введены Фишером, 1963 и Шютценберже, 1963.
\end{itemize}

\section*{Нормальные формы КС-грамматик}
Понятие нормальной формы встречается в различных разделах дискретной математики и информатики и имеет важное значение. Построение алгоритмов или доказательств зачастую делается более удобным, если входные данные имеют некоторую нормальную форму. До сих пор были известны (С)КНФ, (С)ДНФ, пренексная нормальная форма (только с внешними кванторами) и получаемая из неё при помощи операции сколемизации нормальная форма Сколема (конъюнктивная пренексная форма только с кванторами всеобщности). Пренексная нормальная форма использована Гёделем для доказательства теоремы о полноте исчисления предикатов, а НФ Сколема применяется в автоматическом доказательстве теорем.

\subsection*{Нормальная форма Хомского}
\begin{defenv}
	Говорят, что грамматика $G(N,\Sigma, \mathcal P, S)$ \emph{находится в нормальной форме Хомского} (НФХ), если каждое правило из $\mathcal P$ имеет один из следующих видов:
	\begin{enumerate}
		\item $A\to BC$, где $A,B,C\in N$;
		\item $A\to a$, где $A\in N$, $a\in\Sigma$.
	\end{enumerate}
	Кроме того, среди правил может быть $S\to\varepsilon$, при условии что $S$ не встречается в правых частях других правил.
\end{defenv}

\subsubsection*{Алгоритм приведения КС-грамматики к НФХ}
Вход: КС-грамматика $G=(\Sigma, N, \mathcal P, S)$.

Выход: КС-грамматика в НФХ $G'=(\Sigma', N', \mathcal P', S')$.

Метод: получение приведённого вида грамматики с последующим разбиением «длинных» слов в правых частях продукций на слоги (то есть заменой цепочек на новые нетерминалы).

\begin{enumerate}
	\item Получим по $G$ приведённую грамматику (используя известный алгоритм) —
	обозначим её также $G(\Sigma,N,\mathcal P, S)$.
	\item Если $S\to\varepsilon$ есть в $\mathcal P$, добавить $S\to\varepsilon$ к $\mathcal P'$.
	\item Все продукции вида $A \to a$ и $A\to BC$, где $A,B,C\in N$, $a\in \Sigma$, из $\mathcal P$ добавляем в $\mathcal P'$.
	\begin{note}Т.~к. в приведённой грамматике нет продукций вида $A\to B$ и $A\to \varepsilon$, то к этому моменту просмотрены все продукции с правыми частями длины $0$ и $1$. Рассмотрена также часть продукций с правой частью длины два. Оставшиеся продукции такого вида рассматривает шаг~\ref{Len2Bad}.\end{note}
	\item\label{Len2Bad} Для каждой непросмотренной продукции $A\to X_1X_2$ ($X_1,X_2\in\Sigma\cup N$) $\mathcal P$ добавим в $\mathcal P'$ продукцию $A\to X_1'X_2'$, где $X_i'$ это новый нетерминал, если $X_i\in\Sigma$, или совпадает с $X_i$, если $X_i\in N$.
	\begin{note}На данный момент просмотрены все продукции $\mathcal P$ длины не больше двух. Оставшиеся продукции рассматриваются на следующем шаге.\end{note}
	\item Для каждой продукции вида $A\to X_1X_2\ldots X_k$, $k\geqslant 3$ из $\mathcal P$ добавим в $\mathcal P'$ продукции: 
	$$\begin{array}{l}
		A\to X_1'\lbrack X_2X_3\ldots X_k\rbrack;\\
		\lbrack X_2X_3\ldots X_k\rbrack\to X_2'[X_3\ldots X_k];\\
		\ldots\\
		\lbrack X_{k-1}X_k\rbrack\to X_{k-1}'X_k'.
	\end{array}$$
	Где, по-прежнему, $X_i'$ это новый нетерминал, если $X_i\in\Sigma$, или совпадает с $X_i$, если $X_i\in N$.
	\item Для всех добавленных ранее символов $X_i'$ добавим в $\mathcal P'$ продукции $X_i'\to X_i$, если $X_i\in\Sigma$.
\end{enumerate}

\begin{theorem}
	Для любой КС-грамматики G существует эквивалентная ей КС-грамматика в НФХ.
\end{theorem}

Благодаря относительной простоте НФХ, она имеет большое практическое и теоретическое значение. Например, один из немногих общих алгоритмов разбора слов произвольной КС-грамматики, алгоритм Кока—Янгера—Касами, работает с грамматикой в НФХ. Асимптотическая сложность этого алгоритма $\Theta(n^3)$, что делает его плохо применим на практике. Именно потому при построении реальных языков чаще  используют не общие КС-грамматики, а их подкласс, $\LR(k)$, для которого существуют линейные алгоритмы разбора ($\Theta(n)$).

\subsection*{Нормальная форма Грейбах}
\begin{defenv}
	Говорят, что грамматика $G(N,\Sigma, \mathcal P, S)$ \emph{находится в нормальной форме Грейбах} (НФГ), если каждое правило из $\mathcal P$ имеет вид:
	$$ A\to a\gamma,$$
	где $A\in N$, $a\in\Sigma$, $\gamma\in(\Sigma\cup N)^*$. Допускается также правило $S\to \varepsilon$.
\end{defenv}

\begin{theorem}
	Для любой КС-грамматики G существует эквивалентная ей КС-грамматика в НФГ.
\end{theorem}
\begin{note}
	Алгоритм приведения произвольной КС-грамматики к НФГ весьма сложен и не рассматривается в курсе.
\end{note}

Одно из интересных теоретических приложений НФГ состоит в том, что будучи поданной на вход алгоритма построения МП-автомата по произвольной КС-грамматике (этот алгоритм ещё не был приведён, но утверждалось его существование), получается МП-автомат без $\varepsilon$-пе\-ре\-хо\-дов. Т.~о. автоматически получается доказательство утверждения об эквивалентности МП-автоматов и МП-автоматов без $\varepsilon$-переходов.

\section*{Варианты МП-автоматов}
\begin{defenv}
\emph{Расширенным автоматом с магазинной памятью} (РМПА) называют семёрку $(Q,\Sigma,\Gamma,\delta,q_o,Z_0,F)$, где все составляющие имеют такой же смысл, как и для МП-автоматов, но функция переходов $\delta$ имеет вид:
$$\delta:Q\times(\Sigma\cup\{\varepsilon\})\times\Gamma^*\to P_{fin}(Q\times\Gamma^*),$$
причём выполнено условие:
$$|\{(q,a,\gamma) \mid \delta(q,a,\gamma) \neq \emptyset \}| < \infty.$$
\end{defenv}

\begin{theorem}
	$$\LGPDA = \LPDA$$
\end{theorem}
\end{document}
