\documentclass[handout,envcountsect,ignorenonframetext,hyperref={pdftex,unicode,urlcolor=blue},
xcolor=dvipsnames]{beamer}

\usetheme{Madrid} %Warsaw 
\setbeamertemplate{items}[ball] 
\setbeamertemplate{blocks}[rounded][shadow=true] 
\setbeamertemplate{footline}[frame number]
%\setbeamercovered{transparent}
\setbeamertemplate{navigation symbols}{} %no nav symbols
\setbeamertemplate{bibliography item}[text]
\setbeamercolor{bibliography item}{fg=black}
\setbeamertemplate{qed symbol}{\vrule width1.5ex height1.5ex depth0pt}
%\setbeamertemplate{theorems}[numbered]

% font definitions, try \usepackage{ae} instead of the following
% three lines if you don't like this look
\usepackage{mathptmx}
%\usepackage[scaled=.90]{helvet}
%\usepackage{courier}
\parskip=2mm

\usepackage[utf8]{inputenc}
\usepackage[T2A]{fontenc}
\usepackage[russian,english]{babel}
\usepackage{cmap}    % searchable pdf
\usepackage{url}

\abovedisplayskip=.5\abovedisplayskip
\belowdisplayskip=.5\belowdisplayskip
\abovedisplayshortskip=.5\abovedisplayshortskip
\belowdisplayshortskip=.5\belowdisplayshortskip

\newcommand{\bstructure}[1]{{\bfseries \structure{#1}}}
\newcommand{\F}{\ensuremath{\mathbb{F}_{\tilde{q}}}}
\newcommand{\N}{\ensuremath{\mathbb{N}_0}}
\newcommand{\eqdef}{\ensuremath{\stackrel{\text{\upshape\tiny def}}{=}}}
\newcommand{\pt}[1]{\ensuremath{\boldsymbol{#1}}}
\newcommand{\leqp}{\mathrel{\leqslant_{\text{P}}}}
\newcommand{\nspace}{\hspace{0pt}}
\newcommand{\nbdash}{\nobreakdash-\nspace}
\newcommand{\FM}[2]{\ensuremath{\mathcal F(#1^{\pt #2})}}
\newcommand{\Ffail}{F_{\mathrm{fail}}}
\newcommand{\Ffall}{F_{\mathrm{fall}}}
\DeclareMathOperator{\Mon}{Mon}
\DeclareMathOperator{\LM}{LM}

\newtheorem{rutheorem}{Теорема}
\newtheorem{rulemma}{Лемма}
\newtheorem{ruremark}{Замечание}
\newtheorem{rucorollary}{Следствие}
\newtheorem{rudef}{Определение}
\newtheorem*{ruproof}{Доказательство}

\title{О многомерной версии алгоритма Берлекэмпа—Месси}

\author{Пеленицын~А.~М.\\ulysses4ever@gmail.com} 
\institute[ЮФУ]{
Кафедра алгебры и дискретной математики\\ 
Факультет математики, механики и компьютерных наук\\
Южный федеральный университет
}
\date{30 октября 2009~г.}

% This is only inserted into the PDF information catalog. Can be left
% out.
\subject{О многомерной версии алгоритма Берлекэмпа—Месси}

% If you have a file called "university-logo-filename.xxx", where xxx
% is a graphic format that can be processed by latex or pdflatex,
% resp., then you can add a logo as follows: 
%\pgfdeclareimage[height=1.5cm]{university-logo}{sfedu_logo2} %
%\logo{\pgfuseimage{university-logo}}

% Delete this, if you do not want the table of contents to pop up at
% the beginning of each subsection:
\AtBeginSubsection[]
{
\begin{frame}<beamer>
\frametitle{Содержание}
\tableofcontents[currentsection,currentsubsection]
\end{frame}
}

% If you wish to uncover everything in a step-wise fashion, uncomment
% the following command:
%\beamerdefaultoverlayspecification{<+->}
\begin{document}

\begin{frame}[plain]
\titlepage
\end{frame}

\begin{frame}
\frametitle{Содержание}
\tableofcontents[pausesubsections]
\end{frame} 

\section{Одномерный случай}
\subsection{Линейные рекуррентные последовательности} 
\begin{frame}{Определение линейной рекуррентной последовательности}
\onslide<+->{(Одномерная) \structure{Последовательность}: $u\colon \N\to\F$ ($\N
= \{0, 1, \ldots\}$).
}

\onslide<+->{$u$ — \structure{линейная рекуррентная последовательность} (ЛРП),
если существуют $\{a_i\}_{i=0}^{k-1}$, такие что:$$u_{n+k} = \sum_{i=0}^{k-1} a_i
u_{i+n},\quad n\in\N.$$}
\onslide<+->{Тогда
\begin{itemize}
    \item $k$ — \structure{порядок} ЛРП $u$,
    \item<+-> $\{a_i\}_{i=0}^{k-1}$ — \structure{закон рекурсии} ЛРП $u$. 
\end{itemize}
}

\onslide<+->{
Всем известный пример:
$$f_{n+2} = f_{n} + f_{n+1}$$}
\onslide<+->{Закон рекурсии: $a_0=1$, $a_1=1$, порядок — $2$. 
}
\end{frame}

\begin{frame}{Описание класса ЛРП}
\onslide<2->{\begin{rutheorem}
Класс ЛРП совпадает с классом периодических последовательностей.
\end{rutheorem}}
\onslide<3->{\begin{ruproof}
\begin{enumerate}
\item<4-> Пусть $u$ — периодическая. Существуют $p$ и $r$, т.~ч. $u_{n+p}=u_n$,
$n\geqslant r$. Значит $u$ — ЛРП с законом рекурсии $a_{r}=1$ и $a_i=0$, где
$i\in [0, p-1]_{\N}\setminus \{r\}$, порядка $p+r$.
\item<5-> Пусть $u$ — ЛРП порядка $k$ с законом рекурсии $\{a_i\}$.
\begin{itemize}
    \item<6-> $\bar{u}_n=(u_n, u_{n+1},\ldots,u_{n+k-1})$ — \structure{вектор
    $n$-го состояния}, он вполне определяет всю   
    последовательность; в частности, если 
    $\bar{u}_i=\bar{u}_j$, то $\bar{u}_{i+1}=\bar{u}_{j+1}$.
    \item<7-> В последовательности $\bar{u}_0, \bar{u}_1, \ldots$ лишь конечное
    число различных элементов, потому она периодическая.
\end{itemize}
\onslide<8->{Значит, и $u$ периодическая.\qed}
\end{enumerate}
\end{ruproof}}
\end{frame}

\subsection{Задача}
\begin{frame}{Минимальный многочлен I}
Для ЛРП $u$ существует более одного закона
рекурсии. Есть ли между ними связь?\onslide<2->{ — \alert{Да}, её можно описать
в алгебраических терминах.}

\onslide<3->{Пусть $\{a_i\}_{i=0}^{k-1}$ — закон рекурсии $u$. Назовём
\structure{характеристическим многочленом} $u$ нормированный
многочлен:$$f(x)=x^k - \sum_{i=0}^{k-1}a_ix^i.$$}

\onslide<4->{\begin{rutheorem}
Пусть $u$ — ЛРП, тогда существует единственный нормированный многочлен $m(x)$,
такой что любой характеристический многочлен $f(x)$ ЛРП $u$ делится на $m(x)$.
\end{rutheorem}}

\onslide<5->{
\begin{rucorollary}
    Множество характеристических многочленов ЛРП $u$ составляет все
    нормированные многочлены \alert{идеала} $(m(x))$.
\end{rucorollary}
}
\end{frame}

\begin{frame}{Минимальный многочлен II}
\onslide<1->{
Степень $m(x)$ называется \structure{линейной сложностью} ЛРП $u$.
}

\onslide<2->{
\bstructure{Как найти $m(x)$?}
}
\end{frame}

\begin{frame}{От теории к практике}
\onslide<1->{На практике нет возможности работать с бесконечными
последовательностями.}

\onslide<2->{На практике \structure{задача такова}: для данных
$\{u_i\}_{i=0}^m$\\найти $f(x)$ минимальной степени (обозначим её $k$), такой
что
\begin{equation}\sum_{i=0}^k f_i u_{i+n-k} = 0,\quad n\in[k,
m]_{\N}.\end{equation} ($f(x)=\sum_{i=0}^k f_i x^i$.)}

\onslide<3->{Похоже на СЛАУ?}

\onslide<4->{\structure{Решение} этой задачи — $f(x)$ — это
минимальный полином ЛРП $u$, первые $m$ членов которой совпадают с
заданными $\{u_i\}_{i=0}^m$. Закон~рекурсии~$u$:
$\{-\frac{f_i}{f_k}\}_{i=0}^{k-1}$.}
\end{frame}

\begin{frame}{Удобные обозначения}
\onslide<1->{Для $f(x)$ степени $k$, последовательности $u$ и $n\geqslant k$
\structure{введём обозначение}:
$$f[u]_{n}\eqdef\sum_{i=0}^k f_iu_{i+n-k}\quad(\in\F).$$}

\onslide<2->{На практике \structure{задача такова}: для данных
$\{u_i\}_{i=0}^m$\\найти $f(x)$ минимальной степени (обозначим её $k$), такой
что $$f[u]_n = 0,\quad n\in[k, m]_{\N}.$$}

\end{frame}

\subsection{Алгоритм}

\begin{frame}{Индукция}
\onslide<1->{Будем рассуждать \structure{индуктивно}.}\\
\onslide<2->{Пусть $f(x)$ — полином минимальной степени (обозначим её $k$),
такой что $$f[u]_n = 0,\quad k \leqslant n\leqslant p.$$
Как получить полином минимальной степени $f'(x)$  (обозначим её $k'$), такой что
$$f'[u]_n = 0,\quad k' \leqslant n\leqslant p+1?$$}

\onslide<3->{\begin{enumerate}
    \item<3-> $f[u]_{p+1}=0$ — нам повезло: $f'(x)\eqdef f(x)$.
    \item<4-> $f[u]_{p+1}\not=0$ — придётся потрудиться.
\end{enumerate}}
\end{frame}

\begin{frame}{Степень $f'(x)$}
\begin{rulemma}<2->[о нижней границе для степени $f'(x)$]
Для степени $k'$ полинома $f'(x)$ выполнено:
$$k'\geqslant p-k+1.$$
\end{rulemma}

\begin{rucorollary}<3->
Для степени $k'$ полинома $f'(x)$ выполнено
$$k'\geqslant \max {(p-k+1, k)}.$$
\end{rucorollary}
%Если $p-k\geqslant k$, то $k'>k$.
\begin{rucorollary}<4->
Если будет найден $h(x)$, такой что
\begin{enumerate}
    \item $h[u]_n=0,\quad n\leqslant p+1$,
    \item $\deg h = \max {(p-k+1, k)}$,
\end{enumerate}
то $f'(x)\eqdef h(x)$.
\end{rucorollary}

\end{frame}

\begin{frame}{«Формула Берлекэмпа»}
позволяет построить $h(x)$, такой что
\begin{enumerate}
    \item $h[u]_n=0,\quad n\leqslant p+1$,
    \item $\deg h = \max {(p-k+1, k)}$,
\end{enumerate}
на основе имеющегося $f(x)$ и некоторого полинома $g(x)$.

То есть
\begin{align*}
h(x)&=h(f,g),\\
f'(x)&\eqdef h(x).
\end{align*}
\end{frame}

\begin{frame}{Индукция}
\onslide<1->{Уточним и завершим шаг индукции.}

\onslide<2->{Пусть $f(x)$ — полином минимальной степени, такой что
$$f[u]_n = 0,\quad n\leqslant p,$$
и $g(x)$ подходящий для формулы Берлекэмпа полином.

Как получить $f'(x)$, $g'(x)$, такие что\ldots?
}

\onslide<3->{
Возможные варианты:
\begin{enumerate}
    \item<3-> $f[u]_{p+1}=0$ — тогда $f'(x)\eqdef f(x)$, $g'(x)\eqdef g(x)$.
    \item<4-> $f[u]_{p+1}\not=0$ — тогда $f'(x)=h(f,g)$, и\\
    если $k'=k$, то
    $g'(x)\eqdef g(x)$, иначе $g'(x)\eqdef f(x)$.
\end{enumerate}}
\end{frame}

\begin{frame}{«Формула Берлекэмпа»}
$$h(f,g)=x^{r-s}f(x) - \frac{d_p}{d_q}x^{r-p+q-t}g(x).$$
\onslide<2->{Обозначения. $s,t,p,q,r\in\N$, $d_p, d_q\in\F$.
\begin{itemize}
    \item<3-> $s=\deg f$, $t=\deg g$; 
    \item<4-> $p$ — текущий шаг, $q$ — таков, что $\forall m < q\colon g[u]_m =
    0$ и $g[u]_q\not=0$;
    \item<5-> $d_p=f[u]_p$, $d_q=g[u]_q$;   
    \item<6-> $r=\max {(s, p - q + t)}$. 
\end{itemize}}
\end{frame}

\begin{frame}{База индукции}
Инициализация: $f=1$.

$h_0=x^{p+1} - \frac{u_{p+1}}{u_p}$, если $p<m$,\\
$h_0=x^{m+1}$ иначе.
\end{frame}

\begin{frame}{Степень $h(f,g)$ (связь $f$ и $g$)}
$$h(f,g)=x^{r-s}f(x) - \frac{d_p}{d_q}x^{r-p+q-t}g(x),$$
где $r=\max {(s, p - q + t)}$.

\bstructure{Вопрос}: $\deg(h)\stackrel{?}{=}\max(s,p-s+1)$.
\begin{enumerate}
    \item<2-> $r=s$:
    $$\deg (h) = \deg (f - x^{s-p+q-t}g) = \max(s, s - p + q) \stackrel{p>q}{=}
    s.$$
    \item<3-> $r=p - q + t$:
    \begin{multline*}
    \deg(h)=\deg(x^{p-q+t-s}f - g) = \max(p-q+t,t) =\\
        \stackrel{p>q}{=}p-q+t
        \only<3|handout:0>{=?}
        \only<4|handout:0>{\stackrel{*}{=}?}
        \only<5-|handout:1>{\stackrel{*}{=}p-s+1,}
    \end{multline*}
    \onslide<4->(*) — по предположению индукции $s=q-t+1$.
\end{enumerate}

\end{frame}

\section{Многомерный случай}

\subsection{Последовательности и полиномы}
\begin{frame}{Основные определения}
\begin{itemize}
    \item<+-> $n$-мерная последовательность $u$: $u\colon\N^n\to\F$.
    \item<+-> Если $\pt m\in \N^n $, то $x^{\pt m} = x_1^{m_1}
    x_2^{m_2}\cdots x_n^{m_n}$.
    \item<+-> Полином $f(x)$ от $n$ переменных: $f(x) = \sum_{\pt i\in
    \Gamma_f}f_{\pt i}x^{\pt i}$.\\Конечное множество $\Gamma_f (\subset \N^n)$
    — «носитель» $f$.\\$f_{\pt i}\in \F$.
    \item<+-> \alert{Степень $f(x)$?}
\end{itemize}
\end{frame}

\begin{frame}{Мономиальный порядок}
%$(\{ x^{\pt m} \mid \pt m \in \N^n \}, \cdot) \cong 
%(\{ \pt m \in \N^n \}, {+})$
\structure{Мономиальный порядок} ${<}$ на множестве мономов 
$\Mon_n = \{ x^{\pt m} \mid \pt m \in \N^n\}$ это бинарное отношение, обладающее
свойствами:
\onslide<2->{\begin{enumerate}
    \item<2-> ${<}$ — полный порядок (линейный порядок, при котором любое
    $M\subset\Mon_n$ имеет наименьший элемент),
    \item<3-> для $u,v,w\in \Mon_n$ если $u<v$, то $uw<vw$.}
\end{enumerate}

\onslide<4->{
    \structure{Пример.}
    \begin{align*}
    x^{\pt m} < x^{\pt k} &&&\Longleftrightarrow &\onslide<5->{\Bigl(
         &(\sum_i m_i < \sum_i k_i) \onslide<6->{\, \vee\\ 
            &&&&&\, \bigl(\sum_i m_i = \sum_i k_i\bigr) \wedge \bigl(  \exists
            j\, \forall i>j\colon (m_i=k_i) \wedge (m_j < k_j)\bigr)
             }\Bigr)}
    \end{align*}
    }
    
\onslide<7->{
\structure{Зафиксируем ${<}$.} Тогда определена функция $\deg\colon \F[x] \to
\N^n$, $$\deg{(f)}=\max\nolimits_{<}\Gamma_f.$$
%Кроме того, можно определить функцию: 
}
\end{frame}

\begin{frame}{Порядки на $\N^n$}
\onslide<1->{Мономиальный порядок на $(\Mon_n, {\cdot})$
индуцирует \structure{полный порядок} на $(\N^n, {+})$, согласованный с
полугрупповой структурой.} Если $\pt n\in\N^n$, обозначим через $\pt n'\in\N^n$
точку, непосредственно следующую за $\pt n$ относительно этого порядка.

\onslide<2->{Определим ещё \structure{частичный порядок} ${\leqp}$ на $\N^n$:
$$\pt m \leqp \pt k \Leftrightarrow \forall i\colon m_i \leqslant k_i.$$}
\onslide<3->{В терминах мономов ${\leqp}$ означает делимость: если $\pt m
\leqp \pt k$, то корректно определён моном $x^{\pt k} / x^{\pt m} = x^{\pt k
- \pt m}$ ($\pt k - \pt m\in \N^n$)}.

\onslide<4->{
    Для $\pt p, \pt q\in\N^n$ введём \structure{обозначения для множеств точек}:
    \begin{align*}
    \Sigma_{\pt q} &= \{\pt m \in \N^n \mid \pt q \leqp \pt m \},\\
    \Sigma_{\pt q}^{\pt p} &= 
        \{\pt m \in \N^n \mid \pt q \leqp \pt m < \pt p\},\\
    \Gamma_{\pt p} &= \{\pt m \in \N^n \mid \pt m \leqp \pt p\}.
    \end{align*}
}
\end{frame}

\begin{frame}{Линейные рекуррентные последовательности}
\onslide<1->{$u$ называется \structure{линейной рекуррентной
последовательностью}, если существуют $\{f_{\pt i}\}_{\pt i \in \Gamma}$ ($\Gamma\subset\N^n$,
$|\Gamma|<\infty$, $\pt s=\max_<\Gamma$), такие что: 
$$\sum_{\pt i \in \Gamma} f_{\pt i} u_{\pt m + \pt i - \pt s} = 0\quad \forall
\pt m \in \Sigma_{\pt s}.$$}

\onslide<2->{Как и прежде, определяется \structure{характеристический полином}
$f(x)=\sum_{\pt i \in \Gamma}f_{\pt i}x^{\pt i}$ для ЛРП $u$ и вводится
обозначение:
$$f[u]_{\pt m}\eqdef\sum_{\pt i} f_{\pt i} u_{\pt i + \pt m - \pt
s} \quad \forall\pt m \in \Sigma_{\pt s},$$
где $\pt s = \deg f$.}
\onslide<3->{\begin{rutheorem}
Множество $I(u)$ характеристических полиномов ЛРП $u$ является \alert{идеалом} в
$\F[x]$.
\end{rutheorem}}
\end{frame}

\subsection{Задача}
\begin{frame}{Базовые сведения о $\F[x]$\nbdash идеалах I~\cite{CLO'S00}}
\begin{itemize}
    \item<+-> идеалы в $\F[x]$ \alert{не} являются главными;
    \item<+-> однако справедлива 
    \begin{rutheorem}[«Гильберта о базисе»]
    Любой идеал $I\subset\F[x]$ конечнопорождён, т.~е. существуют
    $\{f_i(x)\}_{i=1}^{k}$, такие что $$I = \{f_1g_1 + \ldots + f_kg_k \mid
    g_1(x),\ldots,g_k(x) \in \F[x]\} \eqdef \langle f_1,\ldots,f_k\rangle.$$
    \end{rutheorem}
\end{itemize}
\end{frame}

\begin{frame}{Базовые сведения о $\F[x]$\nbdash идеалах II~\cite{CLO'S00}}
\begin{itemize}
    \item<+-> базис \alert{существенно неединственен} (в частности, разные
    базисы могут содержать разное количество элементов);
    \item<+-> однако существуют «хорошие» базисы: базисы Грёбнера
    \begin{rudef}<+->
    Набор полиномов $\{f_i(x)\}_{i=1}^k\subset I$ называется \structure{базисом
    Грёбнера} идеала $I$, если:
    $$\forall g \in I \; \exists i\colon {\deg f_i} \leqp {\deg g}.$$
    В этом случае $I = \langle f_1,\ldots,f_k \rangle$.
    \end{rudef}
    \begin{rudef}<+->
    Нормированный базис Грёбнера $\{f_i(x)\}_{i=1}^k$ идеала $I$ называется
    \structure{минимальным базисом Грёбнера}, если: 
    $$\forall i \, \forall j\neq i\colon {\deg f_i} \not\leqp {\deg f_j}.$$
    \end{rudef}
\end{itemize}
\end{frame}

\begin{frame}{Задача}
\onslide<1->{Рассмотрим \structure{отрезок последовательности} $u^{\pt r}\colon
\Sigma_{\pt 0}^{\pt r} \to \F$. Для любого $f\in\F[x]$ условие
$$\forall \pt m \in \Sigma_{\deg f}^{\pt r}\colon f[u]_{\pt m}=0$$
будем записывать просто $f[u^{\pt r}]=0$ или кратко: $f[u]=0$.}

\begin{rudef}<2->
Множество нормированных полиномов $F=\{f_i(x)\}_{i=1}^{k}$
называется \structure{минимальным множеством} для отрезка последовательности 
$u^{\pt r}\colon \Sigma_{\pt 0}^{\pt r} \to \F$, если выполнены условия:
\begin{enumerate}
    \item<3-> $\forall i\colon f_i[u]=0$;
    \item<4-> $\forall g(x)\colon (g[u]=0) \to 
        (\exists i\colon \deg f_i \leqp \deg g)$;
    \item<5-> $\forall i \; \forall j\neq i\colon 
        {\deg f_i} \not\leqp {\deg f_j}.$
\end{enumerate}
\end{rudef}
\onslide<6->{\bstructure{Как найти минимальное множество для отрезка
$u^{\pt r}$?}}
\end{frame}

\begin{frame}{Класс минимальных множеств}
Множество минимальных множеств отрезка последовательности
$u^{\pt r}\colon \Sigma_{\pt 0}^{\pt r} \to \F$ обозначим $\FM u r$.
\end{frame}

\begin{frame}{Как выглядит $\deg(\{f(x)\mid f[u]=0\})$?}
Пусть $F\in \FM u r$. Обозначим:
    \begin{align*}
    \Sigma(u^{\pt r}) &\eqdef \bigcup_{f\in F}\Sigma_{\deg f},\\
    \Delta(u^{\pt r}) &\eqdef \Sigma_{\pt 0} \setminus \Sigma.
    \end{align*}
Для краткости можно писать $\Sigma(\pt r)$, $\Delta(\pt r)$.
\end{frame}

\subsection{Алгоритм}
\begin{frame}{Индукция}
\onslide<1->{Будем рассуждать \structure{индуктивно}.}

\onslide<2->{Пусть $\pt p < \pt r$, $F\in \FM u p$, $G\subset \F[x]$ ($|G|<
\infty$).}

\onslide<3->{Необходимо найти $F'\in \FM u {p'}$, $G'\subset \F[x]$ ($|G'|<
\infty$).}

\onslide<4->{Для каждого $f\in F$ есть две возможности:
\begin{enumerate}
    \item $f[u]_{\pt p} = 0$ — тогда $f\in F'$.
    \item $f[u]_{\pt p} \neq 0$ — \ldots
\end{enumerate}
}
\end{frame}

\begin{frame}{Связь $F$ и $G$}
\onslide<1->{$\forall g \in G \; \exists \pt q\colon g\in F(u^{\pt q}) \wedge
g[u]_{\pt q}\neq 0$.} 

\onslide<2->{\bstructure{Требование}: $\{\pt q - \deg g | g\in G\} =
\max_{{\leqp}}\Delta(u^{r})$}.
\end{frame}

\begin{frame}{Лемма о границе для степени $f'(x)\in F'$}
Обозначим $\Ffail = \{ f \in F \mid f[u]_{\pt p}\neq 0 \}$. 
\begin{rulemma}<+->
Пусть $f(x)\in \Ffail$, тогда \alert{не существует}
$f'(x) \in F'$, такого что:
$$\deg f' \leqp (\pt p - \deg f).$$ То есть $\deg f' \in \Sigma_{\pt 0}
\setminus \Gamma_{\pt p - \deg f}$.
% Пусть $f(x)$ таков, что 
% $$\forall \pt m \in \Sigma_{\deg f}^{\pt p}\colon f[u]_{\pt m}=0
% \quad \text{и} \quad f[u]_{\pt p} \neq 0$$
% Тогда не существует $f'(x)$, такого что
% $$\forall \pt m \in \Sigma_{\deg f'}^{\pt p}\colon f'[u]_{\pt m}=0
% \quad \text{и} \quad f'[u]_{\pt p} = 0$$
% со степенью $\deg f' \leqp \pt p - \deg f$.
\end{rulemma}
\begin{rucorollary}<+->
Пусть $\Gamma=\bigcup_{f\in \Ffail}\Gamma_{\pt p - \deg f}$, тогда
$$\deg(F') \subset (\Sigma(u^{\pt p}) \setminus \Gamma).$$
\end{rucorollary}
%\begin{ruremark}
%%Если $\pt p - \deg f \in \Delta(u^{\pt r})$, то 
%\end{ruremark}
\end{frame}

\begin{frame}{Степень $f\in\Ffail$ и формула Берлекэмпа}
\begin{enumerate}
    \item<1-> 
        \onslide<1->{Если $\pt p - \deg f\in\Delta(u^{\pt p})$, то 
        на степень $f'$ нет дополнительных ограничений и формула Берлекэмпа $h(f,g)$
        позволяет построить $f'\colon \deg f' = \deg f$.} 
        
        \onslide<2->{В качестве $g$ нужно взять такой элемент $G$, что $\pt p-\deg
        f \leqp \pt q - \deg g$.}
        
        \onslide<3->{
        $$h(f, g) = f - \frac {d_{\pt p}} {d_{\pt q}} 
        x^{\pt q - \deg g - (\pt p - \deg f)}g.$$}
    \item<4-> Если $\pt p - \deg f\not\in\Delta(u^{\pt p})$, \ldots
\end{enumerate}
\end{frame}

\begin{frame}{$\Ffall$}
\onslide<1->{Остались нерассмотренными $f\in\Ffail$, такие что $\pt p -
\deg f\not\in\Delta(u^{\pt p})$, обозначим их $\Ffall$.}

\onslide<2->{Для каждой пары $(f,g)$, где $f\in\Ffall$, $g\in G$,
\begin{itemize}
    \item<3-> если $\pt s' = \max (\deg f, \pt p - \pt q + \deg g)$ 
    минимальна по ${\leqp}$ в 
    $S' = \{\max (\deg f, \pt p - \pt q + \deg g) \mid 
    f \in \Ffall , \, g \in G\}$,
    \item<4-> то полином:
        $$ h(f, g) = x^{\pt s' - \deg f}f - \frac {d_{\pt p}} {d_{\pt q}}g $$
    добавляется в $F'$. 
\end{itemize}}
\end{frame}

\begin{frame}{Вырожденный случай}
Пусть $\hat{S} $ множество минимальных по ${\leqp}$ элементов в 
$\Sigma(u^{\pt p}) \setminus \Gamma_{\pt p}$.

Для каждого $\pt{\hat{s}} \in \hat{S}$, если не найдётся такого 
$\pt s' \in S'$, что $\pt s' \leqp \pt{\hat{s}}$, тогда для каждого 
$f\in \Ffall$, такого что $\deg f \leqp \pt{\hat{s}}$, полином
$$ h(f) = x^{\pt s' - \deg f}f $$
добавляется в $F'$. 
\end{frame}

%\section{Приложения}

\begin{frame}{Библиография} %[allowframebreaks]
    \begin{thebibliography}{Sakata90}
    \footnotesize
    \bibitem[Blahut86]{Blahut86} \emph{Блейхут Р.} Теория и практика кодов,
    контролирующих ошибки: Пер. с англ. / М.: Мир, 1986.
    \bibitem[CLO'S00]{CLO'S00}\emph{Кокс Д., Литтл Дж., О’Ши Д.} Идеалы,
    многообразия и алгоритмы. Введение в вычислительные аспекты алгебраической геометрии и 
    коммутативной алгебры. / М.: Мир, 2000. 
    \bibitem[KKMN94]{KKMN94}\emph{Kurakin~V. L., Kuzmin~A. S.,
    Mikhalev~A. V., Nechaev~A. A.} Linear recurring sequences over rings and
    modules. // I. of Math. Science. Contemporary Math. and it's Appl. Thematic surveys, vol. 10, 1994.
    \bibitem[LN88]{LidlNider}\emph{Лидл Р., Нидеррайтер Г.} Конечные поля: В
    2-х т. / М.:~Мир, 1988. 822 стр.
    \bibitem[Sakata88]{Sakata88}\emph{Sakata S.} Finding a minimal set of linear
    recurring relations capable of generating a given finite two–dimensional array // J. Symb. Comp. 1988.
Vol. 5. 1988. Pp. 321–337.
    \bibitem[Sakata90]{Sakata90}\emph{Sakata S.} Extension of the
    Berlekamp–Massey algorithm to N dimensions. // Inform. and Comput. 84,
     no. 2. 1990. Pp. 207–239.
    \bibitem[Sakata09]{Sakata09}\emph{Sakata S.} The BMS Algorithm // Chapter
    in Gröbner Bases, Coding, and Cryptography, Springer, 2009.
    %\bibitem[]{}    
    \end{thebibliography}
\end{frame}

\end{document}
