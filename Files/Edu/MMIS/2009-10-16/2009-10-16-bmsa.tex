\documentclass[envcountsect,ignorenonframetext,hyperref={pdftex,unicode,urlcolor=blue},
xcolor=dvipsnames]{beamer}

\usetheme{Madrid} %Warsaw 
\setbeamertemplate{items}[ball] 
\setbeamertemplate{blocks}[rounded][shadow=true] 
\setbeamertemplate{footline}[frame number]
%\setbeamercovered{transparent}
\setbeamertemplate{navigation symbols}{} %no nav symbols
\setbeamertemplate{bibliography item}[text]
\setbeamercolor{bibliography item}{fg=black}
\setbeamertemplate{qed symbol}{\vrule width1.5ex height1.5ex depth0pt}
%\setbeamertemplate{theorems}[numbered]

% font definitions, try \usepackage{ae} instead of the following
% three lines if you don't like this look
\usepackage{mathptmx}
%\usepackage[scaled=.90]{helvet}
%\usepackage{courier}
\parskip=2mm

\usepackage[utf8]{inputenc}
\usepackage[T2A]{fontenc}
\usepackage[russian,english]{babel}
\usepackage{cmap}    % searchable pdf
\usepackage{url}

\newcommand{\bstructure}[1]{{\bfseries \structure{#1}}}
\newcommand{\F}{\ensuremath{\mathbb{F}_{\tilde{q}}}}
\newcommand{\N}{\ensuremath{\mathbb{N}_0}}
\newcommand{\eqdef}{\stackrel{\text{\upshape\tiny def}}{=}}
\newcommand{\pt}[1]{\boldsymbol{#1}}

\newtheorem{rutheorem}{Теорема}
\newtheorem{rulemma}{Лемма}
\newtheorem{ruremark}{Замечание}
\newtheorem{rucorollary}{Следствие}
\newtheorem*{ruproof}{Доказательство}

\title[Model checking вычислений высших порядков]{О многомерной версии
алгоритма Берлекэмпа—Месси}

\author{Пеленицын~А.~М.\\ulysses4ever@gmail.com} 
\institute[ЮФУ]{
Кафедра алгебры и дискретной математики\\ 
Факультет математики, механики и компьютерных наук\\
Южный федеральный университет
}
\date{16 октября 2009~г.}

% This is only inserted into the PDF information catalog. Can be left
% out.
\subject{О многомерной версии алгоритма Берлекэмпа—Месси}

% If you have a file called "university-logo-filename.xxx", where xxx
% is a graphic format that can be processed by latex or pdflatex,
% resp., then you can add a logo as follows: 
%\pgfdeclareimage[height=1.5cm]{university-logo}{sfedu_logo2} %
%\logo{\pgfuseimage{university-logo}}

% Delete this, if you do not want the table of contents to pop up at
% the beginning of each subsection:
\AtBeginSubsection[]
{
\begin{frame}<beamer>
\frametitle{Содержание}
\tableofcontents[currentsection,currentsubsection]
\end{frame}
}

% If you wish to uncover everything in a step-wise fashion, uncomment
% the following command:
%\beamerdefaultoverlayspecification{<+->}
\begin{document}

\begin{frame}[plain]
\titlepage
\end{frame}

\begin{frame}
\frametitle{Содержание}
\tableofcontents[pausesubsections]
\end{frame} 

\section{Одномерный случай}
\subsection{Линейные рекуррентные последовательности} 
\begin{frame}{Определение линейной рекуррентной последовательности}
\onslide<+->{(Одномерная) \structure{Последовательность}: $u\colon \N\to\F$ ($\N
= \{0, 1, \ldots\}$).
}

\onslide<+->{$u$ — \structure{линейная рекуррентная последовательность} (ЛРП),
если существуют $\{a_i\}_{i=0}^{k-1}$, такие что:$$u_{n+k} = \sum_{i=0}^{k-1} a_i
u_{i+n},\quad n\in\N.$$}
\onslide<+->{Тогда
\begin{itemize}
    \item $k$ — \structure{порядок} ЛРП $u$,
    \item<+-> $\{a_i\}_{i=0}^{k-1}$ — \structure{закон рекурсии} ЛРП $u$. 
\end{itemize}
}

\onslide<+->{
Всем известный пример:
$$f_{n+2} = f_{n} + f_{n+1}$$}
\onslide<+->{Закон рекурсии: $a_0=1$, $a_1=1$, порядок — $2$. 
}
\end{frame}

\begin{frame}{Описание класса ЛРП}
\onslide<2->{\begin{rutheorem}
Класс ЛРП совпадает с классом периодических последовательностей.
\end{rutheorem}}
\onslide<3->{\begin{ruproof}
\begin{enumerate}
\item<4-> Пусть $u$ — периодическая. Существуют $p$ и $r$, т.~ч. $u_{n+p}=u_n$,
$n\geqslant r$. Значит $u$ — ЛРП с законом рекурсии $a_{r}=1$ и $a_i=0$, где
$i\in [0, p-1]_{\N}\setminus \{r\}$, порядка $p+r$.
\item<5-> Пусть $u$ — ЛРП порядка $k$ с законом рекурсии $\{a_i\}$.
\begin{itemize}
    \item<6-> $\bar{u}_n=(u_n, u_{n+1},\ldots,u_{n+k-1})$ — \structure{вектор
    $n$-го состояния}, он вполне определяет всю   
    последовательность; в частности, если 
    $\bar{u}_i=\bar{u}_j$, то $\bar{u}_{i+1}=\bar{u}_{j+1}$.
    \item<7-> В последовательности $\bar{u}_0, \bar{u}_1, \ldots$ лишь конечное
    число различных элементов, потому она периодическая.
\end{itemize}
\onslide<8->{Значит, и $u$ периодическая.\qed}
\end{enumerate}
\end{ruproof}}
\end{frame}

\subsection{Задача}
\begin{frame}{Минимальный многочлен I}
Для ЛРП $u$ существует более одного закона
рекурсии. Есть ли между ними связь?\onslide<2->{ — \alert{Да}, её можно описать
в алгебраических терминах.}

\onslide<3->{Пусть $\{a_i\}_{i=0}^{k-1}$ — закон рекурсии $u$. Назовём
\structure{характеристическим многочленом} $u$ нормированный
многочлен:$$f(x)=x^k - \sum_{i=0}^{k-1}a_ix^i.$$}

\onslide<4->{\begin{rutheorem}
Пусть $u$ — ЛРП, тогда существует единственный нормированный многочлен $m(x)$,
такой что любой характеристический многочлен $f(x)$ ЛРП $u$ делится на $m(x)$.
\end{rutheorem}}

\onslide<5->{
\begin{rucorollary}
    Множество характеристических многочленов ЛРП $u$ составляет все
    нормированные многочлены \alert{идеала} $(m(x))$.
\end{rucorollary}
}
\end{frame}

\begin{frame}{Минимальный многочлен II}
\onslide<1->{
Степень $m(x)$ называется \structure{линейной сложностью} ЛРП $u$.
}

\onslide<2->{
\bstructure{Как найти $m(x)$?}
}
\end{frame}

\begin{frame}{От теории к практике}
\onslide<1->{На практике нет возможности работать с бесконечными
последовательностями.}

\onslide<2->{На практике \structure{задача такова}: для данных
$\{u_i\}_{i=0}^m$\\найти $f(x)$ минимальной степени (обозначим её $k$), такой
что
\begin{equation}\sum_{i=0}^kf_iu_{i+n-k} = 0,\quad n\in[k,
m]_{\N}.\end{equation} ($f(x)=\sum_{i=0}^kf_ix^i$.)}

\onslide<3->{Похоже на СЛАУ?}

\onslide<4->{\structure{Решение} этой задачи — $f(x)$ — это
минимальный полином ЛРП $u$, первые $m$ членов которой совпадают с
заданными $\{u_i\}_{i=0}^m$. Закон~рекурсии~$u$:
$\{-\frac{f_i}{f_k}\}_{i=0}^{k-1}$.}
\end{frame}

\begin{frame}{Удобные обозначения}
\onslide<1->{Для $f(x)$ степени $k$, последовательности $u$ и $n\geqslant k$
\structure{введём обозначение}:
$$f[u]_{n}\eqdef\sum_{i=0}^k f_iu_{i+n-k}\quad(\in\F).$$}

\onslide<2->{На практике \structure{задача такова}: для данных
$\{u_i\}_{i=0}^m$\\найти $f(x)$ минимальной степени (обозначим её $k$), такой
что $$f[u]_n = 0,\quad n\in[k, m]_{\N}.$$}

\end{frame}

\subsection{Алгоритм}

\begin{frame}{Индукция}
\onslide<1->{Будем рассуждать \structure{индуктивно}.}\\
\onslide<2->{Пусть $f(x)$ — полином минимальной степени (обозначим её $k$),
такой что $$f[u]_n = 0,\quad k \leqslant n\leqslant p.$$
Как получить полином минимальной степени $f'(x)$  (обозначим её $k'$), такой что
$$f'[u]_n = 0,\quad k' \leqslant n\leqslant p+1?$$}

\onslide<3->{\begin{enumerate}
    \item<3-> $f[u]_{p+1}=0$ — нам повезло: $f'(x)\eqdef f(x)$.
    \item<4-> $f[u]_{p+1}\not=0$ — придётся потрудиться.
\end{enumerate}}
\end{frame}

\begin{frame}{Степень $f'(x)$}
\begin{rulemma}<2->[о нижней границе для степени $f'(x)$]
Для степени $k'$ полинома $f'(x)$ выполнено:
$$k'\geqslant p-k+1.$$
\end{rulemma}

\begin{rucorollary}<3->
Для степени $k'$ полинома $f'(x)$ выполнено
$$k'\geqslant \max {(p-k+1, k)}.$$
\end{rucorollary}
%Если $p-k\geqslant k$, то $k'>k$.
\begin{rucorollary}<4->
Если будет найден $h(x)$, такой что
\begin{enumerate}
    \item $h[u]_n=0,\quad n\leqslant p+1$,
    \item $\deg h = \max {(p-k+1, k)}$,
\end{enumerate}
то $f'(x)\eqdef h(x)$.
\end{rucorollary}

\end{frame}

\begin{frame}{«Формула Берлекэмпа»}
позволяет построить $h(x)$, такой что
\begin{enumerate}
    \item $h[u]_n=0,\quad n\leqslant p+1$,
    \item $\deg h = \max {(p-k+1, k)}$,
\end{enumerate}
на основе имеющегося $f(x)$ и некоторого полинома $g(x)$.

То есть
\begin{align*}
h(x)&=h(f,g),\\
f'(x)&\eqdef h(x).
\end{align*}
\end{frame}

\begin{frame}{Индукция}
\onslide<1->{Уточним и завершим шаг индукции.}

\onslide<2->{Пусть $f(x)$ — полином минимальной степени, такой что
$$f[u]_n = 0,\quad n\leqslant p,$$
и $g(x)$ подходящий для формулы Берлекэмпа полином.

Как получить $f'(x)$, $g'(x)$, такие что\ldots?
}

\onslide<3->{
Возможные варианты:
\begin{enumerate}
    \item<3-> $f[u]_{p+1}=0$ — тогда $f'(x)\eqdef f(x)$, $g'(x)\eqdef g(x)$.
    \item<4-> $f[u]_{p+1}\not=0$ — тогда $f'(x)=h(f,g)$, и\\
    если $k'=k$, то
    $g'(x)\eqdef g(x)$, иначе $g'(x)\eqdef f(x)$.
\end{enumerate}}
\end{frame}

\begin{frame}{«Формула Берлекэмпа»}
$$h(f,g)=x^{r-s}f(x) - \frac{d_p}{d_q}x^{r-p+q-t}g(x).$$
\onslide<2->{Обозначения. $s,t,p,q,r\in\N$, $d_p, d_q\in\F$.
\begin{itemize}
    \item<3-> $s=\deg f$, $t=\deg g$; 
    \item<4-> $p$ — текущий шаг, $q$ — таков, что $\forall m < q\colon g[u]_m =
    0$ и $g[u]_q\not=0$;
    \item<5-> $d_p=f[u]_p$, $d_q=g[u]_q$;   
    \item<6-> $r=\max {(s, p - q + t)}$. 
\end{itemize}}
\end{frame}

\begin{frame}{База индукции}
Инициализация: $f=1$.

$h_0=x^{p+1} - \frac{u_{p+1}}{u_p}$, если $p<m$,\\
$h_0=x^{m+1}$ иначе.
\end{frame}

\section{Многомерный случай}

\subsection{Последовательности и полиномы}

\subsection{Задача}

\subsection{Алгоритм}

\section{Приложения}

\begin{frame}{Библиография} %[allowframebreaks]
    \begin{thebibliography}{9}
    \footnotesize
    \bibitem[Blahut86]{Blahut86} \emph{Блейхут Р.} Теория и практика кодов,
    контролирующих ошибки: Пер. с англ. / М.: Мир, 1986.
    \bibitem[LN88]{LidlNider}\emph{Лидл Р., Нидеррайтер Г.} Конечные поля: В
    2-х т. / М.:~Мир, 1988. 822 стр.
    \bibitem[KKMN94]{KKMN94}\emph{V. L. Kurakin, A. S. Kuzmin, A. V. Mikhalev,
    A. A. Nechaev.} Linear recurring sequences over rings and modules. // I. of
    Math. Science. Contemporary Math. and it's Appl. Thematic surveys, vol. 10, 1994.
    \bibitem[Sakata88]{Sakata88}\emph{Sakata S.} Finding a minimal set of linear
    recurring relations capable of generating a given finite two–dimensional array // J. Symb. Comp. 1988.
Vol. 5. Pp. 321–337.
    \bibitem[Sakata09]{Sakata09}\emph{Sakata S.} The BMS Algorithm // Chapter
    in Gröbner Bases, Coding, and Cryptography, Springer, 2009.
    %\bibitem[]{}
    \bibitem[CLO'S00]{CLO'S00}\emph{Кокс Д., Литтл Дж., О’Ши Д.} Идеалы,
    многообразия и алгоритмы. Введение в вычислительные аспекты алгебраической геометрии и 
    коммутативной алгебры. / М.: Мир, 2000. 
    
    \end{thebibliography}
\end{frame}

\end{document}
