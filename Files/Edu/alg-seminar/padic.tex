\documentclass[12pt]{scrartcl}%,leqno scrartcl
\usepackage{cmap}
\usepackage[utf8x]{inputenc}
\usepackage[russian]{babel}
\usepackage[T2A]{fontenc}
\usepackage{amsmath}
\usepackage{amssymb}
\usepackage{amsthm}
\usepackage{graphicx}
\usepackage{tikz}
\usetikzlibrary{automata,shapes,arrows,positioning,calc}
\usepackage{clrscode}
\usepackage{sectsty}
\usepackage{verbatim}
\usepackage{ifthen}
\usepackage{forloop}
\usepackage{calc}
\usepackage{multicol}
\usepackage{indentfirst}
\usepackage{paralist}


%\sectionfont{\large}

\usepackage[linkbordercolor=white]{hyperref}%,menucolor=white,runbordercolor=white,urlbordercolor=white,
%colorlinks,urlcolor=black

% \topmargin=0cm
% %\headsep=10pt
% \oddsidemargin=0cm
% \textheight=21cm
% \textwidth=17cm

%\usepackage[vscale=0.77,hscale=0.8,centering]{geometry}%

\usepackage{fancyhdr}
\pagestyle{fancy}
\lfoot{\footnotesize ЮФУ, Мехмат}
%\rfoot{\footnotesize А.~Э.~Маевский}
\cfoot{--- \thepage{} ---}
\lhead{\itshape Семинар по алгебре}
\rhead{}

\parindent=1.25cm

\tolerance=1000

%\renewcommand{\thesubsubsection}{\arabic{\subsection}}
%\arabic{\subsubsection}
\renewcommand{\emptyset}{\varnothing}
\renewcommand{\leq}{\leqslant}
\newcommand{\es}{\emptyset}
\newcommand{\h}{\ensuremath{\hbar}}
\newcommand{\N}{\ensuremath{\mathbb N}}
\newcommand{\OO}{\ensuremath{\mathcal O}}
\newcommand{\PP}{\ensuremath{\mathfrak P}}
\newcommand{\Z}{\ensuremath{\mathbb Z}}
\newcommand{\Q}{\ensuremath{\mathbb Q}}
\newcommand{\F}{\ensuremath{\mathbb F}}
\newcommand{\A}{\ensuremath{\mathbb A}}
\newcommand{\R}{\ensuremath{\mathbb R}}
\renewcommand{\P}{\ensuremath{\mathbb P}}
\renewcommand{\C}{\ensuremath{\mathbb C}}
\newcommand{\NO}{\ensuremath{\mathbb N_0}}
\renewcommand{\phi}{\varphi}
\newcommand{\eps}{\varepsilon}
\renewcommand{\projlim}{\varprojlim}
\newcommand{\ol}{\overline}
\newcommand{\iso}{\cong}
\newcommand{\normsub}{\subset}
\newcommand{\normsup}{\supset}
\newcommand{\hto}{\hookrightarrow}
\renewcommand{\leq}{\leqslant}
\renewcommand{\geq}{\geqslant}
 
\DeclareMathOperator{\Aut}{Aut}
\DeclareMathOperator{\Hom}{Hom}
\DeclareMathOperator{\Endo}{End}
\DeclareMathOperator{\Id}{id}
\DeclareMathOperator{\chr}{char}
\DeclareMathOperator{\Gal}{Gal}
\DeclareMathOperator{\GL}{GL}
\DeclareMathOperator{\PGL}{PGL}

\renewcommand{\qed}{\hfill \vrule width1.5ex height1.5ex depth0pt}
\newcommand{\pseudoqed}{\hfill\Box}
\newcommand{\HW}{\ensuremath{{\otimes}}}
\newcommand{\eqdef}{\stackrel{\text{\upshape\tiny def}}{=}}
\newcommand{\coleq}{\ensuremath{\mathrel{\mathop:}=}}
\renewcommand{\labelenumi}{(\theenumi)}
\newcommand{\InsertList}[1]{%
\input{listok#1}
}
\newcommand{\nspace}{\hspace{0pt}}
\newcommand{\nbdash}{\nobreakdash-\nspace}
\newcommand{\ul}{\underline}
\newcommand{\rep}[1]{#1^*}

\newtheorem{Thm}{Теорема}[section]
\renewcommand{\theThm}{\arabic{Thm}}
\newtheorem{Lemma}{Лемма}[section]
\renewcommand{\theLemma}{\arabic{Lemma}}
\newtheorem{Cor}{Следствие}[section]
\renewcommand{\theCor}{\arabic{Cor}}

\theoremstyle{remark}
\newtheorem*{SketchOfProof}{Набросок доказательства}

\theoremstyle{definition}
\newtheorem{Ex}{Пример}
\newtheorem{Exec}{Упражнение}
\newtheorem*{Sol}{Решение}
\newtheorem{Prop}{Утверждение}
\newtheorem*{Remark}{Замечание}
\newtheorem{NumRemark}{Замечание}
\newtheorem*{Algo}{Алгоритм}
\newtheorem{NumAlgo}{Алгоритм}
\newtheorem{Def}{Определение}[section]
\renewcommand{\theDef}{\arabic{Def}}
\newtheorem*{Task}{Задача}

\title{Семинар по алгебре}
\author{рук. В.~А.~Стукопин}
\date{}

\begin{document}
\section{$p$-адические числа}
\subsection{Нормированные поля}
Определение абсолютного значения (нормы) на поле. Примеры: евклидова норма на \C{}
(\R, \Q), тривиальная норма. Единственная норма в конечном поле~— тривиальная.
$p$-адическая норма на $\Q$.

Неархимедовы нормы. Эквивалентное определение. Эквивалентность норм и теорема
Островского. Формула произведения:
\[
    \prod_{2 \leq p \leq \infty} |a|_p = 1.
\]

Последовательности Коши, полные поля, пополнение: определение и существование.

\subsection{Представление $p$-адических чисел в виде степенных рядов}
Свяжем с каждым «неархимедовым нормированным полем» несколько важных
алгебраических объектов.

\begin{Def}[+утверждение]
Пусть $(K, |\cdot|)$ — нормированное поле с неархимедовой нормой.
\begin{compactenum}
  \item Множество $\OO = \{ a \in K \mid |a| \leq 1 \}$ называется
  \emph{кольцом нормирования $(K, |\cdot|)$} и является подкольцом $K$.
  \item Множество $\PP = \{a \in K \mid |a| < 1\}$ называется \emph{идеалом
  нормирования $(K, |\cdot|)$} и является единственным максимальным идеалом
  $\OO$.
  \item Поле $k = \OO / \PP$ называется \emph{полем вычетов нормирования
  $(K, |\cdot|)$}.
\end{compactenum}
\end{Def}

\begin{Remark}
Кольцо с единственным максимальным идеалом называется \emph{локальным}.
\end{Remark}

\begin{Prop}
Для нормированного поля $(\Q, |\cdot|_p)$
\[
    \OO = \Z_{(p)} = \left \{ \frac{a}{s} \mid s \not \equiv 0 \bmod p \right
    \}, \qquad \PP = p\Z_{(p)}, \qquad
    k \cong \Z/p\Z.
\]
\end{Prop}
\begin{proof}
Первые два равенства очевидны. Для доказательства третьего рассмотрим
гомоморфизм $\Z \to \OO/\PP$, являющийся композицией вложения и канонического
гомоморфизма в факторкольцо. Его ядро равно $p\Z$ и значит, гомоморфизм $\Z/p\Z
\to \OO/\PP$ является вложением, остаётся показать его сюръективность.

Пусть $a/s \in \OO$, тогда $s$ обратимо в кольце $\Z/p\Z$, а значит существует
такое $b$, что $bs \equiv a \bmod {p\Z}$. Эта сравнимость продолжается до $bs
\equiv a \bmod {p\OO}$. Для $s$ существует обратный элемент $1/s$ в $R$, который
также принадлежит $R$, домножая последнее сравнение на него находим $b \equiv
a/s \bmod {p\OO}$.
\end{proof}

Оказывается, что в случае неархимедового нормирования поле вычетов нормирования
данного нормированного поля не меняется при пополнении этого поля. Для
доказательства этого сформулируем сначала одно важное свойство неархимедовых
норм.

\begin{Lemma}
Пусть $(K, |\cdot|)$ — нормированное поле с неархимедовой нормой, $a,b\in K$.
Тогда
\[
    |a| \neq |b| \Longrightarrow |a+b| = \max(|a|, |b|).
\]
\end{Lemma}
\begin{proof}
Пусть $|b| < |a|$ и предположим противное: $|a+b| < |a|$. Тогда
\[
    |a| = |a + b - b| \leq \max(|a+b|, |b|) < |a|.
\]
Противоречие.
\end{proof}


\begin{Thm}
Пусть $(K, |\cdot|)$ — нормированное поле с неархимедовой нормой, а $(\hat K,
{|\cdot|})$~— его пополнение, $\OO$, $\hat \OO$, $\PP$, $\hat \PP$, $k$, $\hat
k$ — соответствующие поля вычетов. Тогда $k \cong \hat k$.
\end{Thm}
\begin{proof}
Пусть $\alpha \in \hat K$ и последовательность $\{a_n\} \subset K$ сходится к
$\alpha$. Тогда начиная с некоторого номера $N\in\N$ для всех $n>N$ имеем
$|a_n - \alpha| < |\alpha|$, а значит $|a_n| = |a_n - \alpha + \alpha| =
|\alpha|$ (по предыдущей лемме). 

Пусть $\OO$, $\hat \OO$, $\PP$, $\hat \PP$~— соответствующие кольца и идеалы
нормирований. Для $\alpha \in \hat \hat \OO$ существует $a \in K$, такое что
$|a| = |\alpha|$, возьмём такое $a$. Очевидно, $a \in \OO$ и $|a-\alpha|\leq 1$,
а значит $a\equiv\alpha \bmod{\hat \PP}$ и $a$, таким образом, является
прообразом $\alpha$ про очевидном гомоморфизме $\OO/\PP \to \hat\OO / \hat\PP$.
\end{proof}

Прежде чем сформулировать основную теорему о представлении $p$-адических чисел
степенными рядами, докажем одно важное свойство рядов в полных неархимедовых
полях.

\begin{Lemma}
Пусть $(\hat K, |\cdot|)$~— полное нормированное поле с неархимедовой нормой и
$\{a_n\}\subset\hat K$. Тогда
\[
    \sum_n a_n \text{ сходится} \Longleftrightarrow a_n \to 0.
\]
\begin{proof}
Импликация ${\Rightarrow}$ известна из классического анализа. Пусть $a_n \to 0$.
Обозначим через $\{ b_n = \sum_0^n a_n \}$ последовательность частичных сумм
данного ряда, она является последовательностью Коши ввиду неравенства
\[
    |b_n - b_{n+k}| \leq \max\{ |a_n|, \ldots , |a_{n+k}| \},
\]
а значит, имеет предел в $\hat K$~— ввиду его полноты.
\end{proof}
\end{Lemma}
\begin{Cor}
Если $\{a_n\} \subset \hat\OO$ и $x\in \hat K$, то для сходимости ряда
$\sum a_n x^n$ достаточно, чтобы $x \in \hat \PP$.
\end{Cor}

\begin{Def}
Кольцо нормирования поля $\Q_p$ обозначается $\Z_p$ и имеет специальное
название — \emph{кольцо $p$-адических целых}. (Иногда, чтобы избежать путаницы,
кольцо $\Z$ называют \emph{кольцом рациональных целых}.)
\end{Def} 

\begin{Thm}
Каждое $a \in \Z_p$ имеет единственное представление в виде суммы ряда
\[
    a = \sum_{n=0}^{\infty} a_i p^i, \quad a_i\in\{0, \ldots , p-1\}.
\]
Каждое $a \in \Q_p$ имеет единственное представление в виде суммы ряда
\[
    a = \sum_{n=-k}^{\infty} a_i p^i, \quad k\in \N_0, a_i \in \{0, \ldots ,
        p-1\}.
\]
\end{Thm}

\end{document}