\documentclass[12pt]{article}%,leqno
\usepackage{cmap}
\usepackage[utf8x]{inputenc}
\usepackage[russian]{babel}
\usepackage[T2A]{fontenc}
\usepackage{amsmath}
\usepackage{amssymb}
\usepackage{amsthm}
\usepackage{graphicx}
\usepackage{tikz}
\usetikzlibrary{automata,shapes,arrows,positioning,calc}
\usepackage{clrscode}
\usepackage{sectsty}
\usepackage{verbatim}
\usepackage{ifthen}
\usepackage{forloop}
\usepackage{calc}
\usepackage{multicol}
\usepackage{indentfirst}

%\sectionfont{\large}

\usepackage[linkbordercolor=white]{hyperref}%,menucolor=white,runbordercolor=white,urlbordercolor=white,
%colorlinks,urlcolor=black

% \topmargin=0cm
% %\headsep=10pt
% \oddsidemargin=0cm
% \textheight=21cm
% \textwidth=17cm

\usepackage[vscale=0.77,hscale=0.8,centering]{geometry}%

\usepackage{fancyhdr}
\pagestyle{fancy}
\lfoot{\footnotesize ЮФУ, Мехмат}
%\rfoot{\footnotesize А.~Э.~Маевский}
\cfoot{--- \thepage{} ---}
\lhead{\itshape Семинар по алгебре}
\rhead{}

\parindent=1.25cm

\tolerance=1000

%\renewcommand{\thesubsubsection}{\arabic{\subsection}}
%\arabic{\subsubsection}
\renewcommand{\emptyset}{\varnothing}
\renewcommand{\leq}{\leqslant}
\newcommand{\es}{\emptyset}
\newcommand{\h}{\ensuremath{\hbar}}
\newcommand{\N}{\ensuremath{\mathbb N}}
\newcommand{\Z}{\ensuremath{\mathbb Z}}
\newcommand{\F}{\ensuremath{\mathbb F}}
\newcommand{\A}{\ensuremath{\mathbb A}}
\newcommand{\R}{\ensuremath{\mathbb R}}
\renewcommand{\P}{\ensuremath{\mathbb P}}
\renewcommand{\C}{\ensuremath{\mathbb C}}
\newcommand{\NO}{\ensuremath{\mathbb N_0}}
\renewcommand{\phi}{\varphi}
\newcommand{\eps}{\varepsilon}
\renewcommand{\projlim}{\varprojlim}
\newcommand{\ol}{\overline}
\newcommand{\iso}{\cong}
\newcommand{\normsub}{\subset}
\newcommand{\normsup}{\supset}
\newcommand{\hto}{\hookrightarrow}
\renewcommand{\leq}{\leqslant}
\renewcommand{\geq}{\geqslant}

\DeclareMathOperator{\Aut}{Aut}
\DeclareMathOperator{\Hom}{Hom}
\DeclareMathOperator{\Endo}{End}
\DeclareMathOperator{\Id}{id}
\DeclareMathOperator{\chr}{char}
\DeclareMathOperator{\Gal}{Gal}
\DeclareMathOperator{\GL}{GL}
\DeclareMathOperator{\PGL}{PGL}

\renewcommand{\qed}{\hfill \vrule width1.5ex height1.5ex depth0pt}
\newcommand{\pseudoqed}{\hfill\Box}
\newcommand{\HW}{\ensuremath{{\otimes}}}
\newcommand{\eqdef}{\stackrel{\text{\upshape\tiny def}}{=}}
\newcommand{\coleq}{\ensuremath{\mathrel{\mathop:}=}}
\renewcommand{\labelenumi}{(\theenumi)}
\newcommand{\InsertList}[1]{%
\input{listok#1}
}
\newcommand{\nspace}{\hspace{0pt}}
\newcommand{\nbdash}{\nobreakdash-\nspace}
\newcommand{\ul}{\underline}
\newcommand{\rep}[1]{#1^*}

\newtheorem{Thm}{Теорема}[section]
\renewcommand{\theThm}{\arabic{Thm}}
\newtheorem{Lemma}{Лемма}[section]
\renewcommand{\theLemma}{\arabic{Lemma}}

\theoremstyle{remark}
\newtheorem*{SketchOfProof}{Набросок доказательства}

\theoremstyle{definition}
\newtheorem{Ex}{Пример}
\newtheorem{Exec}{Упражнение}
\newtheorem*{Sol}{Решение}
\newtheorem{Prop}{Утверждение}
\newtheorem*{Remark}{Замечание}
\newtheorem{NumRemark}{Замечание}
\newtheorem*{Algo}{Алгоритм}
\newtheorem{NumAlgo}{Алгоритм}
\newtheorem{Def}{Определение}[section]
\renewcommand{\theDef}{\arabic{Def}}
\newtheorem*{Task}{Задача}

\title{Семинар по алгебре}
\author{рук. В.~А.~Стукопин}
\date{}

\begin{document}
\section{Введение в теорию представлений}
\begin{thebibliography}{9}
\bibitem{Serre}Ж.-П. Серр. Линейные представления конечных групп
\end{thebibliography}

\begin{Def}
Пусть $X$ — множество, $V$ — векторное пространство над $K$.
\emph{Представлением множества $X$} называется отображение $f\colon X \to L(V)$,
где $L(V)$ — множество линейных операторов в $V$.
\end{Def}
В такой общности от этого определения мало толку: обычно предполагается
некоторая структура на $X$ и уважение со стороны $f$ этой структуры.

Если есть два представления $f\colon X \to L(V)$ и $g\colon X \to L(V_1)$, то
интересно получить обратимое линейное отображение $U\colon V \to V_1$, такое что
\[
    f(x) = U g(x) U^{-1}.
\]
Если такое отображение существует, то представления $f$ и $g$ называются
эквивалентными.
\begin{Ex}
$X = \{ \star \}$. Тогда (для конечномерных $V$) задача сводится к описанию
классов сопряжённости матриц.
\end{Ex}

Уже в случае $X = \{ \star, \star\star \}$ сложно сказать что-то определённое. 

\begin{Def}
Пусть $V$ — конечномерное векторное пространство над $K$, $G$ — группа, $A$ —
алгебра над $K$. \emph{Представление группы $G$} это гомоморфизм групп $f \colon
G \to \GL(V)$, \emph{представление алгебры $A$} это гомоморфизм алгебр $g
\colon A \to L(V)$.
\end{Def}

\begin{Def}
Рассмотрим векторное пространство над $K$ с базисом $\{ e_g \}_{g \in G}$, где
$G$ — некоторая конечная группа. Групповой алгеброй группы $G$ называется
алгебра
\[
    K[G] = \{ \sum_{i = 1}^n x_i g_i \mid x_i \in K \},
\]
с естественным умножением таких конечных сумм. 
\end{Def}
В последнем определении можно снять требование конечности $G$, рассматривая в
$K[G]$ только конечные суммы соответствующего вида. Тогда в случае $G=\Z$, 
например, получится алгебра полиномов Лорана.

\subsection{Задача классификации представлений}
\begin{Def}
Пусть $A$ — линейный оператор в $V$. Подпространство $U \subset V$ называется
\emph{инвариантным} для оператора $A$, если $AU \subset U$.
\end{Def}
Если $\{ e_i \}_{i=1}^n$ — базис в $V$, а $\{ e_i \}_{i=1}^k$ — базис $U$ —
инвариантного подпространства для оператора $A$, то в матрице оператора в базисе
$\{e_i\}$ будет нулевой угол.
\begin{Def}
Представление $\rho \colon G \to \GL(V)$ называется \emph{неприводимым}, если у
операторов $\{ \rho(g) \}$ нет нетривиальных общих инвариантных подпространств.
\end{Def}

Напомним
\begin{Def}
Левый модуль $M$ над кольцом с единицей $R$ это абелева группа, для которой
задано отображение $R\times M \to M$, такое что:
\begin{enumerate}
  \item $em = m$;
  \item $\lambda(\mu m) = (\lambda\mu)m$;
  \item $\lambda(m_1 + m_2) = \lambda m_1 + \lambda m_2)$;
  \item $(\lambda_1 + \lambda_2)m = \lambda_1 m + \lambda_2 m$.
\end{enumerate}
\end{Def}
Пространство прибытия представления группы $G$ является левым $K[G]$-модулем.
Более того, от любого $K[G]$-модуля можно перейти к некоторому представлению.

Иногда представления записывают в терминах \emph{матричных элементов}:
представление $R\colon G \to \GL(n, K)$ можно мыслить как набор отображений:
$\{ r_{ij} \colon G \to K \}$.

\begin{Ex}
Пусть $X$ — конечное множество, а $S(X) = S_n$ — группа биекций $X$ (группа
перестановок из $n$ элементов, если $|X|=n$). Пусть $G$ — подгруппа $S_n$, а
$V$ — линейное пространство функций $\{v \colon X \to V\}$. Тогда $\rho \colon
G \to V$, действующее по формуле
\[
    (\rho(g)f)(x) = f(g^{-1}x),
\]
является представлением $G$ в $V$.
\end{Ex}

\begin{Def}
Представление $\rho \colon G \to \GL(V)$ называется \emph{вполне приводимым},
если
\[
    V = V_1 \oplus V_2,
\]
где $V_1, V_2$ — инвариантные подпространства $\{\rho(g)\}$ и $V_i \neq \{0\},
V$.
\end{Def}

% Можно ввести операцию суммы представлений.

В общем случае непонятно, является ли приводимое представление вполне
приводимым. Однако это верно для любой конечной группы и для любой компактной
группы.

Пусть $(V, (\cdot, \cdot))$ — конечномерное векторное пространство над $K$ с
некоторым скалярным произведением. Для представления $\rho$ \emph{конечной} группы $G$
построим \emph{инвариантное скалярное произведение}: 
\[
    \langle x, y\rangle = \langle \rho(g)x, \rho(g)y \rangle.
\]
Таковым является скалярное произведение, определённое по формуле:
\[
    \langle x, y\rangle = \frac {1} {|G|} \sum_{g\in G} (\rho(g)x, \rho(g)y).
\]
\begin{Exec}
Проверить предыдущее утверждение.
\end{Exec}

\begin{Exec}
Доказать, что из приводимости конечной группы следует полная приводимость.
(\emph{Указание}: воспользоваться конструкцией скалярного произведения $\langle
\cdot, \cdot\rangle$).
\end{Exec}

\section{Представления симметрической группы}
\begin{thebibliography}{9}
\bibitem{VO}\emph{А.~М.~Вершик}, \emph{А.~Ю.~Окуньков},
Новый подход к теории представлений симметрических групп. II.
\href{http://www.mathnet.ru/php/archive.phtml?wshow=paper&jrnid=znsl&paperid=840&option_lang=rus}{[mathnet.ru]}
\end{thebibliography}

\begin{Def}
Рассмотрим цепочку групп
\[
    \{ 1 \} = G(0) \subset G(1) \subset \ldots \subset G(n) \subset \ldots
\]
Обозначим $\rep{G(n)}$ — множество классов эквивалентности неприводимых
комплексных представлений группы $G(n)$.

\emph{Граф ветвления цепочки групп $\{G(i)\}$} это следующий граф: 
\begin{itemize}
  \item вершины графа: $\bigcup_{n \leq 0} \rep{G(n)}$;
  \item вершина $\lambda \in \rep{G(n)}$ соединяется с вершиной $\mu \in
  \rep{G(n-1)}$ $k$ рёбрами,  где $k=\dim \Hom_{G(n-1)}(V^\mu, V^\lambda)$.
\end{itemize}
\end{Def}

Имеет место разложение
\[
    V^\lambda = \bigoplus_{\mu \in \rep{G(n-1)}} V^{\mu}.
\]
Каждый $V^{\mu}$ можно разложить аналогичным образом. Продолжая этот процесс,
дойдём до разложения:
\[
    V^\lambda = \bigoplus_{T}V_T,
\]
где $V_T$ это одномерное комплексное пространство, $T$ пробегает все возможные
пути в графе ветвления от $V^\lambda$ к «одномерным листьям». В каждом
пространстве $V_T$ выделяется базисный вектор $v_T$.

\begin{Def}
Нормированный относительно $G(n)$-инвариантного скалярного произведения базис
$\{v_T\}$ в $V^\lambda$ называется \emph{базисом Гельфанда—Цетлина}.
\end{Def}

Обозначим $Z(n)$ центр групповой алгебры $\C[G(n)]$. Подалгебра, порождённая
линейными комбинациями элементов $Z(1)$, \ldots , $Z(n)$, называется
алгеброй Гельфанда—Цетлина.

\begin{Prop}
Имеет место \emph{фундаментальный изоморфизм}:
\[
    \C[G(n)] = \bigoplus_{\lambda} \Endo(V^\lambda).
\]
\end{Prop}

\begin{Prop}
\begin{enumerate} $GZ(n)$ имеет следующие характеризации:  
  \item $GZ(n)$ — это алгебра всех операторов, диагонализируемых в базисе Г—Ц.
  
  \item $GZ(n)$ — это максимальная коммутативная подалгебра $\C[G(n)]$. 
\end{enumerate}
\end{Prop}

\begin{Remark}
Любой вектор из базиса Г—Ц в любом неприводимом представлении группы
$G(n)$ однозначно определяется набором его собственных значений.
\end{Remark}

% \begin{Prop}
% Для любого семейства полупростых алгебр, подалгебра $GZ$ максимальна
% если в графе ветвления нет кратных дуг. 
% \end{Prop}

Пусть $M$ — $\C$-алгебра, $N$ — её подалгебра. Обозначим $Z(M,N)$ множество всех
элементов $M$, коммутирующих со всеми элементами $N$.

\begin{Prop}
Следующие утверждения эквивалентны.
\begin{enumerate}
  \item Ограничение конечномерных неприводимых представлений $M$ на её
  подалгебру $N$ имеет простую кратность (0 или 1).
  \item $Z(M,N)$ коммутативна.
\end{enumerate}
\end{Prop} 

\subsection{Элементы Юнга—Юциса—Мёрфи}
Далее полагаем $G(n) = S_n$. 

\begin{Def}
Элементом Юнга—Юциса—Мёрфи групповой алгебры $\C[S_n]$ называется элемент
\[
    X_i = (1i) + (2i) + \ldots (i-1\:i), \qquad 1 \leq i \leq n.
\]
\end{Def}

Заметим, что
\[
    X_i = \{ \text{сумма всех транспозиций в $S_i$} \} -
     \{ \text{сумма всех транспозиций в $S_{i-1}$} \},
\]
уменьшаемое в этой разности принадлежит $Z(i)$, а вычитаемое — $Z(i-1)$, а
значит, $X_i \in GZ(n)$ для $1 \leq i \leq n$.
\begin{Prop}
\[
    (A =) \sum_{\tau \in S_i} \tau \in Z(i),
\]
где $\tau$ пробегает все транспозиции в $S_i$. 
\end{Prop}
\begin{proof}
Пусть $g \in S_i$.
\[
    gAg^{-1} = \sum g\tau g^{-1} = \sum \tau = A,
\]
так как (1) сопряжение сохраняет цикловый тип перестановки, в частности,
переводит транспозиции в транспозиции, (2) сопряжение является автоморфизмом, то
есть действуя на разные перестановки, получаем разные перестановки. 
\end{proof}
\begin{Thm}
\[
    Z(n) \subset \langle Z(n-1), X_n \rangle.
\]
\begin{proof}
\[
    X_i = \sum_{i=1}^{n-1}(i,n) =  \sum_{i\neq j; i,j=1}^{n} (i,j) - 
    \sum_{i\neq j; i,j=1}^{n-1} (i,j).
\]
\[
    X_n^2 = \sum(i,n)(j,n) = \sum_{i\neq j; i,j=1}^{n}(i,j,n)  + (n-1)\Id,
\]
а значит, $A = \sum_{i\neq j; i,j=1}^{n}(i,j,n) \in \langle Z(n-1), X_n
\rangle$. Кроме того $B = \sum_{i\neq j \neq k; i,j,k=1}^{n}(i,j,k) \in Z(n-1)$,
и
\[
    C = \sum_{i\neq j \neq k; i,j,k=1}^n (i,j,k) = A + B \Longrightarrow C \in
    \langle Z(n-1), X_n \rangle.
\]
Аналогично можно рассмотреть $X_n\sum(i_1, \ldots, i_{k-1},n)$.

\end{proof}
\end{Thm}

\section{Деформации алгебр}

\begin{Def}
Пусть $A$ — ассоциативная алгебра. Деформацией алгебры $A$ называется алгебра
$\Lambda_\h = A[\h] / (\h^2) = \{ a + b\h \}$ с умножением, заданным следующим
образом:
\[
    a * b = ab + f(a,b)\h + O(\h^2),
\]
при этом (чтобы полученная алгебра также была ассоциативной) от $f$ требуется:
\[
    f(ab, c) + f(a,b)x - f(a, bc) - af(b,c) = 0.
\]
\end{Def}
\begin{Remark}
Функция $f$ является элементом \emph{когомологии Хохшильда}.
\end{Remark}

\end{document}