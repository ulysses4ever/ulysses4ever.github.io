\documentclass[a4paper]{article} %12pt,
\usepackage[utf8x]{inputenc}
\usepackage{ucs}
\usepackage[T2A]{fontenc}
\usepackage{cyrtimes}
\usepackage[russian,english]{babel}
%\usepackage{eufrak}
%\usepackage{amsmath}
%\usepackage{amsfonts}
%\usepackage{amssymb}
\author{Ulysses}
\begin{document}
Зададимся алфавитом $\Sigma$. Множество регулярных выражений над $\Sigma$ обозначим~$RE(\Sigma)$. Зафиксируем алфавит переменных: \[\Delta = \{X_i \mid i \in \overline{1,n}\}, \quad \Sigma \cap \Delta = \emptyset. \]
Далее потребуется обозначение: \[\Delta_j = \{X_i \mid i \in \overline{j,n}\},\; j\in \overline{1,n+1}.\] В частности, $\Delta_1 = \Delta$ и $\Delta_{n+1} = \emptyset$.
\par Пусть дана афинная система уравнений с регулярными коэффициентами:
\[\mathcal{E}: \left\{
X_i = \sum_{j=1}^n \alpha_{i,j} X_j + \alpha_{i,0} \qquad i \in \overline{1,n}
\right., \]
где $\alpha_{i,j} \in RE(\Sigma),\; i,j \in \overline{1,n}$. $i$-ое уравнение системы будем обозначать $\mathcal{E}[i]$. 

Обозначим операцию подстановки регулярного выражения $\gamma \in RE(\Omega)$ в $i$-ое уравнение системы $\mathcal{E}$ вместо всех вхождений переменной $X_j$ следующим образом:
\[\mathcal E[i] \gets \gamma / X_j.\]

Решением системы уравнений $\mathcal E$ называется набор $(\hat{x}_1,\ldots,\hat{x}_n)\in {RE(\Sigma)}^n$, такой что при подстановке: \[
\forall{i,j\in\overline{(1,n)}} \quad \mathcal E[i] \gets \hat{x}_j \ / X_j.
\] каждое $\mathcal E[i]$ прератится в верное регулярное тождество. 

\par Напомним, что решением афинного уравнения с регулярными коэффициентами: \[(1):X = \alpha X + \beta\] является регулярное выражение $\hat{x} = \alpha^*\beta$. Обозначим этот факт так: \[
(1):X = \alpha X + \beta \mapsto \hat{x} = \alpha^*\beta.
\]Теперь мы готовы изложить алгоритм, реализующий метод Гаусса для системы~$\mathcal E$.

\newtheorem{algos}{Алгоритм}
\begin{algos} Вход: система $\mathcal E$;\\
Выход: решение $(\hat{x}_1,\ldots,\hat{x}_n)$ системы $\mathcal E$;\\
алгоритм представлен последовательностью шагов:\\
St.1: $i=1$.\\
St.2:\\if $i=n$ then goto St.4;\\
else $\mathcal E[i]: X_i = \alpha X_i + \beta \mapsto \hat{x}_i = \alpha^*\beta, \quad \alpha, \beta \in RE(\Sigma \cup \Delta_{i+1})$;\\
for $j$ from $i+1$ to $n$ do $\mathcal E[j] \gets \hat{x}_i / X_i$.\\
St.3: $++i$, goto St.2;\\
St.4: $\mathcal E[i]: X_i = \alpha X_i + \beta \mapsto \hat{x}_i = \alpha^*\beta, \quad \alpha, \beta \in RE(\Sigma)$;\\
for $j$ from $i-1$ downto $1$ do: $\mathcal E[j] \gets \hat{x}_i / X_i$;\\
St.5: if $i > 1$ $--i$; goto St.4; else exit;
\end{algos}
По завершении работы алгоритма значения $\hat{x}_i$ составляют искомый набор.
\end{document}