\begin{enumerate}
    \itemsep=\myitemsep
    \item Расставить скобки, нарисовать дерево и выполнить полную редукцию
    терма:
    \[
        (\Lvar{uv} u (\Lvar{v} vv) v) (\Lvar{vu} uvv) v  
    \]
    Для каждой подстановки в процессе редукции указать номер используемого
    пункта определения операции подстановки.
    \item Выполнить редукцию, используя нормальный порядок и вызов по имени
    (при подстановке можно не указывать номер используемого правила)
    \[
        (\Lvar{xy} x (\Lx x)) ((\Lvar{xy} y (\Lx x))x) (\Ly yy (\Lx x))
            (\Ly yy(\Lx x))  
    \]
    \item Вычислить: $(\Lx \Lif {\isZero x} {4} {x + 3})2$. Все вычисления
    проводятся с помощью редукции соответствующих \L-термов.
    \item Вычислить в комбин\'{а}торной логике:
    $\clS \clS \clS \clS \clS \clS \clS$.
    \item Дать рекурсивное определение функции \texttt{fib}, вычисляющей
    $n$-ое число Фибоначчи ($\text{\texttt{fib} } 0 = \text{\texttt{fib}
    } 1 = 1$). Записать с помощью комбинатора неподвижной точки соответствующий
    \L-терм. Вычислить $\text{\ttfamily fib } 3$ (остаток от деления 4 на 3).
    \emph{Указание}: операцию вычитания можно не проделывать в \L-термах.
\end{enumerate}