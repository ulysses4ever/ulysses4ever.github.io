\documentclass[12pt]{article}%,leqno
\usepackage{cmap}
\usepackage[utf8x]{inputenc}
\usepackage[russian]{babel}
\usepackage[T2A]{fontenc}
\usepackage{amsmath}
\usepackage{amssymb}
\usepackage{amsthm}
\usepackage{graphicx}
\usepackage{tikz}
\usetikzlibrary{automata,shapes,arrows,positioning,calc}
\usepackage{clrscode}
\usepackage{sectsty}
\usepackage{verbatim}
\usepackage{ifthen}
\usepackage{forloop}
\usepackage{calc}
\usepackage{multicol}
\usepackage{indentfirst}

%\sectionfont{\large}

\usepackage[urlbordercolor=white,linkbordercolor=white]{hyperref}%,menucolor=white,runbordercolor=white
%colorlinks,urlcolor=black

% \topmargin=0cm
% %\headsep=10pt
% \oddsidemargin=0cm
% \textheight=21cm
% \textwidth=17cm

\usepackage[vscale=0.77,hscale=0.8,centering]{geometry}%

\usepackage{fancyhdr}
\pagestyle{fancy}
\lfoot{\footnotesize ЮФУ, Мехмат}
%\rfoot{\footnotesize А.~Э.~Маевский}
\cfoot{--- \thepage{} ---}
\lhead{\itshape Методы алгебраической геометрии в защите информации}
\rhead{}

\parindent=1.25cm

\tolerance=1000

%\renewcommand{\thesubsubsection}{\arabic{\subsection}}
%\arabic{\subsubsection}
\renewcommand{\emptyset}{\varnothing}
\renewcommand{\leq}{\leqslant}
\newcommand{\es}{\emptyset}
\newcommand{\N}{\ensuremath{\mathbb N}}
\newcommand{\F}{\ensuremath{\mathbb F}}
\newcommand{\NO}{\ensuremath{\mathbb N_0}}
\renewcommand{\phi}{\varphi}
\newcommand{\ol}{\overline}
\newcommand{\hto}{\hookrightarrow}
\renewcommand{\leq}{\leqslant}
\renewcommand{\geq}{\geqslant}
\DeclareMathOperator{\Aut}{Aut}
\DeclareMathOperator{\Id}{id}
\DeclareMathOperator{\chr}{char}
\DeclareMathOperator{\Gal}{Gal}
\renewcommand{\qed}{\hfill \vrule width1.5ex height1.5ex depth0pt}
\newcommand{\pseudoqed}{\hfill\Box}
\newcommand{\HW}{\ensuremath{{\otimes}}}
\newcommand{\eqdef}{\stackrel{\text{\upshape\tiny def}}{=}}
\newcommand{\coleq}{\ensuremath{\mathrel{\mathop:}=}}
\renewcommand{\labelenumi}{(\theenumi)}
\newcommand{\InsertList}[1]{%
\input{listok#1}
}
\newcommand{\nspace}{\hspace{0pt}}
\newcommand{\nbdash}{\nobreakdash-\nspace}
\newcommand{\ul}{\underline}

\newtheorem{Thm}{Теорема}[section]
\renewcommand{\theThm}{\arabic{Thm}}
\newtheorem{Lemma}{Лемма}[section]
\renewcommand{\theLemma}{\arabic{Lemma}}

\theoremstyle{remark}
\newtheorem*{SketchOfProof}{Набросок доказательства}

\theoremstyle{definition}
\newtheorem{Ex}{Пример}
\newtheorem{Exec}{Упражнение}
\newtheorem*{Sol}{Решение}
\newtheorem{Prop}{Утверждение}
\newtheorem*{Remark}{Замечание}
\newtheorem{NumRemark}{Замечание}
\newtheorem*{Algo}{Алгоритм}
\newtheorem{NumAlgo}{Алгоритм}
\newtheorem{Def}{Определение}[section]
\renewcommand{\theDef}{\arabic{Def}}
\newtheorem*{Task}{Задача}

\title{Методы алгебраической геометрии в защите информации}
\author{А.~Э.~Маевский}
\date{}

\begin{document}
\maketitle
\thispagestyle{fancy}

\section{Элементы теории Галуа}
\begin{center}
\pgfdeclareimage[height=8cm]{Galois}{Galois}%
\pgfuseimage{Galois}
\end{center}
\begin{Def}
Алгебраическое расширение $L/K$ называется \emph{нормальным}, если каждый
неприводимый над $K$ многочлен, имеющий в $L$ корень, разлагается над $L$ на
линейные множители.
\end{Def}

\begin{Def}\emph{Группой автоморфизмов поля $K$} называется множество
$$
    \Aut(K) \eqdef \{ \phi \colon K \to K \}
$$
с операцией композиции.
\end{Def}

\begin{Exec}
Доказать, что $\Aut(K)$ группа.
\end{Exec}

\begin{Exec}
Пусть $G$ — подгруппа в $\Aut(K)$. Доказать, что
$$
    K^G \eqdef \{ x \in K \mid \forall g \in G \colon g(x) = x \}
$$
подполе в $K$.
\end{Exec}

\begin{Def}
Группа автоморфизмов $L$ над $K$ определяется так:
$$
    \Aut(L / K) = \Aut_K(L) = \{ \phi \in \Aut(L) \mid \phi|_K = \Id_K \}.
$$
\end{Def}

\begin{Thm}
Пусть $L/K$ — нормальное алгебраическое. Тогда
$$
    \forall g \in \Aut(\overline L / K)\colon \: g|_L \in \Aut(L / K).
$$
\end{Thm}
\begin{proof}
Т.к. $L$ — алгебраическое, то
\[
    \forall \alpha \in L \; \exists m_{\alpha}(x) \in K[x] 
    \text{ — минимальный многочлен для $\alpha$.}
\]
Пусть $\alpha \in L$. Все корни $m_{\alpha}(x)$ принадлежат $\overline L$ и,
более того, $L$ (нормальность).

Пусть $g \in \Aut(\overline L / K)$. 
$$
    0 = g(\alpha) = g(m_\alpha (\alpha)) = \ldots = m_\alpha (g(\alpha)).
$$
Следовательно, $g(\alpha)$ — корень $m_\alpha(x)$, а значит, принадлежит
$\overline L$ и $L$.
\end{proof}

\begin{Thm}
Пусть $L/K$ — произвольное алгебраическое расширение. Тогда
$$
    \forall g \in \Aut(L/K) \colon \; 
    \exists \tilde{g} \in \Aut(\overline{L} / K)\colon \tilde{g}|_L = g.  
$$ 
\end{Thm}
\begin{proof}
По Т. о вложении алг. расширений $g \in \Aut(L / K)$ рассмотрим как вложение $g
\colon L \hto \overline L$, причём $g|_K = \sigma \colon K \hto \ol L$. Разным
$g$ соответствуют различные вложения.
\end{proof}

\begin{Thm}[о мощности $\Aut(L/K)$]
Пусть $L / K$ — конечное алгебраическое расширение, тогда
$$
    |\Aut(L/ K) | \leqslant [L:K]. 
$$
\end{Thm}
\begin{proof}
Пусть $\alpha \in \ol K$, существует минимальный над $K$ многочлен $\alpha$.
$$
    K(\alpha) \cong K[x] / (m_\alpha (x)).
$$
Пусть $d = \deg m_\alpha(x)$. Пусть $\alpha = \alpha_1, \alpha_2, \ldots ,
\alpha_s$ — различные корни $m_\alpha(x)$ в $\ol K$ ($s \leqs d$). Для
каждого $i \in [1,s]_\N$ существует единственный $\sigma_i \colon K(\alpha)
\hto \ol K$ и других вложений $\hat \sigma \colon K(\alpha) \hto \ol K$, $\hat
\sigma |_K = \sigma$ нет.

Это рассуждение продолжается по индукции для любого конечного алгебраического
расширения (которое, как известно, можно рассматривать как последовательность
простых расширений).
\end{proof}

\begin{Def}
Число $s$ различных вложений $K(\alpha)$ в $\ol K$ называется \emph{степенью
сепарабельности $\alpha$}. Обозначение: $\deg_s(K(\alpha) / K) = [K(\alpha) :
K]_s$
\end{Def}

\begin{Remark}
Степень сепарабельности $\alpha \in K$ совпадает с количеством корней
минимального многочлена $\alpha$ в $\ol K$.
\end{Remark}

\begin{Exec}
$L/K$ — алгебраическое расширение. $\phi\colon L \to L$ — гомоморфизм полей над
$K$. Тогда $\phi$ — изоморфизм.
\end{Exec}

\begin{Sol}
Гомоморфизм $\phi$ переводит каждый элемент $\alpha\in L$ в сопряжённый с ним.
Так как сопряжённых — конечное число, получается инъективный гомоморфизм
\emph{конечномерного} векторного пространства $K(\alpha)$, который (факт лин.
алгебры) является изоморфизмом.
\end{Sol}

\begin{Remark}
В случае трансцендентных расширений предыдущее утверждение, вообще говоря, не
верно.
\end{Remark}

Распространим понятие сепарабельности на произвольные алгебраические расширения.

\begin{Def}
$L/K$ — конечное алг. расширение, $\sigma\colon K \to \ol K$. \emph{Степень
сепарабельности $L$ над $K$} это количество различных вложений $\hat{\sigma}
\colon L \to \ol K$, таких что $\hat \sigma |_K = \sigma$. Расширение $L/K$ называется сепарабельным,
если $[L:K]_s = [L:K]$.
\end{Def}

\begin{Prop}
\begin{equation}\label{norm-sep-inequality}
    |\Aut(L/K)| \leq [L:K]_s \leq [L:K].
\end{equation}
\end{Prop}
Первое неравенство: два различных вложения $L \to \ol K$ могут не давать
два различных автоморфизма $L$, если образы вложений не совпадают.

В двойном неравенстве~\eqref{norm-sep-inequality} равенство слева — признак
нормальности расширения, справа — сепарабельности.

\begin{Thm}
\[
    [M:K]_s = [M:L]_s \cdot [L:K]_s
\] 
\end{Thm}
\begin{proof}
$\sigma\colon K \hto \ol K$. $\sigma_i\colon L \hto \ol K$, $\sigma_i|_K =
\sigma$. $m = [L:K]_s$, $n = [M:K]_s$, $t = [M:L]_s$. $\tau_{ij}\colon M \hto
\ol K$, $\tau_{ij}|_L = \sigma_i$ (Здесь мы используем $\ol L = \ol K$, иначе, в соответствии с опр.
сепарабельности, следовало бы писать $\tau_{ij}\colon M \hto \ol L$.)

Надо показать, что
\[
    \tau_{ij} \neq \tau_{lk} \quad \text{при } i\neq l, j\neq k.
\]
Если $\tau_{ij} = \tau_{lk}$, то $\sigma_i = \tau_{ij}|_L = \tau_{lk}|_L =
\sigma_l$, а значит, $i=l$. Но тогда $j=k$.

Имеем $[M:K]_s \geq t\cdot m$. Осталось показать обратное нестрогое неравенство.
Пусть $\rho \colon M \hto \ol K$, $\rho|_K=\sigma$. Но $\rho|_L=\sigma_i$ для
некоторого $i \in [1,m]_\N$. Следовательно $\rho = \tau_{ij}$ для некоторого
$j$.
\end{proof} 

\begin{Def}Многочлен $f(x) \in K[x]$ называется \emph{сепарабельным}, если все
его корни в $\ol K$ различны.
\end{Def}

\begin{Lemma}
Любой неприводимый многочлен над полем характеристики $0$ сепарабелен.
\end{Lemma}
\begin{proof}Производная имеет конечную степень (не равна $0$) и потому взаимно
проста с неприводимым многочленом.
\end{proof}

\begin{Lemma}
Пусть $f(x) \in K[x]$ неприводимый. $f(x) \in K[x]$ не сепарабелен тогда и
только тогда, когда $\chr K = p > 0$, $f(x) = g(x^{p^n})$ и $g(x)$ неприводим и сепарабелен над $K$.
\end{Lemma}

\begin{Prop}
Несепарабельные неприводимые многочлены могут существовать только в полях $K$,
таких что $\chr K = p > 0$ и $K^p \neq K$ ($K^p \eqdef \{a^p \mid a \in K\}$).
\end{Prop}
\begin{proof}
Несепарабельный неприводимый многочлен $f(x) \in K[x]$ имеет вид $f(x) =
g(x^{p^n})$, но в поле $K$, таком что $K^p = K$, можно извлекать корни $p$-ой степени, а значит
$f(x) = \left ( h(x) \right )^{p^n}$ что противоречит неприводимости.
\end{proof}

\begin{Ex}
Пусть $K=\F_p(t)$, $f(x) = x^p + t$ — неприводимый несепарабельный многочлен.
\end{Ex}

\begin{Lemma}
$K \subset L \subset M$ — произвольные алгебраические расширения. Если $M/K$
сепарабельно, то $M/L$ и $L/K$ сепарабельны.
\end{Lemma}

\begin{proof}
$[M:L]_s[L:K]_s = [M:K]_s = [M:K] = [M:L][L:K]$, но $[\cdot]_s\leq[\cdot]$,
следовательно, степени и степени сепарабельности промежуточных расширений равны.
\end{proof}

Рассмотрим подробнее несепрабельные элементы (алгебраические элементы,
минимальные многочлены которых несепарабельны).

\begin{Thm}
Пусть $\alpha$ — алгебраический и несепарабельный элемент над полем $K$
характеристики $p$, $m_\alpha(x) = g(x^{p^\mu})$. Тогда $\alpha^{p^\mu}$ сепарабельный и
\[
    [K(\alpha):K]_s = \deg g(x), \quad [K(\alpha):K] = p^\mu [K(\alpha):K]_s.
\]
\end{Thm}

\begin{proof}
\ldots
\[
    f(x) = g(x^{p^\mu}) = \prod_{i=1}^m (x^{p^\mu} - \alpha^{p^\mu})
     = \prod_{i=1}^m (x - \alpha)^{p^\mu}.
\]
\end{proof}

\begin{Def} 
$L/K$ — конечное алгебраическое расширение. \emph{Индекс несепарабельности}:
\[
    [L : K]_i = \frac {[L:K]} {[L:K]_s}.
\] 
Если $[L:K]_i$ максимально, то расширение называется \emph{чисто
несепарабельным}. Если $[K(\alpha):K]_i$ максимально, то $\alpha$ называется
\emph{чисто несепарабельным}. 
\end{Def}

\begin{Def}
Элемент $\alpha \in L$ называется \emph{чисто несепарабельным}, если
\end{Def}

\begin{Lemma}\label{pure-insep-elem-criteria}
Элемент $\alpha\in L$ чисто несепарабелен тогда и только тогда, когда
\[
    m_\alpha(x) = x^{p^\mu} - a, \quad a \in K.
\]
\end{Lemma}

\begin{Def}
Пусть $L/K$ — произвольное (необязательно конечное) алгебраическое расширение.
Оно называется \emph{чисто несепарабельным}, если для любого $\alpha \in L$:
\[
    m_\alpha(x) = x^{p^\mu} - a_\alpha \in K[x].
\]
\end{Def}

\begin{Exec}
Пусть $K \subset L \subset M$ — башня алгебраических расширений. Показать, что
$M/K$ чисто несепарабельно тогда и только тогда, когда чисто несепарабельны
$M/L$ и $L/K$.
\end{Exec}

\begin{Prop}
Пусть $L/K$ — алгебраическое расширение. Обозначим $L^s$ — множество всех
сепарабельных элементов $L$. Тогда $L^s$ — поле.
\end{Prop}

\begin{Prop}
$L^s/K$ — сепарабельно, а $L/ L^s$ — чисто несепарабельно.
\end{Prop}

\begin{proof}
Пусть $\alpha \in L$ — несепарабельный. $m_\alpha(x) = g(x^{p^\mu})$, $g(x) \in
K[x]$ — неприводимый и сепарабельный. $\alpha^{p^\mu} \in L^s$. $x^{p^\mu} -
\alpha^{p^\mu} \in L^s[x]$. По лемме~\eqref{pure-insep-elem-criteria} $\alpha$~—
чисто несепарабельным.
\end{proof}

\begin{Def}
Алгебраическое расширение $L/K$ называется \emph{расширением Галуа}, если оно
нормально и сепарабельно. \emph{Группой Галуа} $\Gal(L/K)$ расширения Галуа
$L/K$ называется его группа автоморфизмов.
\end{Def}

\begin{Thm}
Пусть $L/K$ — конечное расширение Галуа. Существует взаимно однозначное
соответствие между подгруппами $\Gal(L/K)$ и подполями между $L$ и $K$:
\[
    H \leftrightarrow L^H (=\{x\in L \mid Hx=x\}),
\]
при этом нормальным подгруппам соответствуют нормальные расширения $K$.
\end{Thm}

\begin{Def}
Поле $K$ называется совершенным, если любое его алгебраическое расширение
сепарабельно.
\end{Def}

\begin{Def}
Абсолютной группой Галуа поля $K$ называется группа
$\Aut\left((\ol K)^s / K\right)$.
\end{Def}

\begin{Exec}
$(\ol K)^s / K$ — нормально.
\end{Exec}
\end{document}
