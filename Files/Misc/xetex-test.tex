\documentclass{article}
\usepackage{xunicode}
\usepackage[no-sscript]{xltxtra}
\usepackage{polyglossia}
\usepackage{xecyr}
\usepackage{indentfirst}

\setmainfont{PT Serif}
\setsansfont{PT Sans}
\setmonofont{Inconsolata}

\setdefaultlanguage{russian}
\setotherlanguage{english}

\usepackage[colorlinks,urlcolor=cyan]{hyperref}
\usepackage{url}
\usepackage{fancyhdr}

\pagestyle{empty}

\renewcommand{\thesubsection}{\arabic{subsection}}

\begin{document}
\subsection{Комбинаторика на словах}
Относительно молодой раздел дискретной математики, изучающий свойства формальных языков и отдельных слов средствами комбинаторики. Основополагающим трудом в этой области является серия из трёх книг M.~Lothaire. Здесь встречается значительное число интересных алгоритмов. Например, алгоритмы для взвешенных автоматов (аналогичные алгоритмам для обычных конечных автоматов) и их использование для статистической обработки естественных языков (\textenglish{M.~Lothaire, Applied Combinatorics On Words, chap.~4}). Кроме реализации отдельных алгоритмов важной частью работы стало бы написание обзора развития дисциплины по первой книге M.~Lothaire, так как эта область, насколько мне известно, плохо представлена в русскоязычной литературе. 
\end{document}

