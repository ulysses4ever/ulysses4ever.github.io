\documentclass[a4paper,10pt]{article}

\lccode`\-=`\-
\defaulthyphenchar=127

\usepackage[english]{babel}%,russian
\usepackage[utf8]{inputenc}
\usepackage{latexsym}
\usepackage{amsfonts}
\usepackage{amssymb, amsmath}
\usepackage{amsthm} % for {proof}
\usepackage{indentfirst}
\usepackage{cmap} % copy text from зва
%\usepackage[T2A]{fontenc}
%\usepackage{geometry}
%\geometry{a4paper,margin=3cm}
\parindent=1.25cm

\emergencystretch=1em

% \usepackage{mdwlist} % list-related facilities
\newcommand{\CPP}
{C\nolinebreak[4]\hspace{-.05em}\raisebox{.22ex}{\footnotesize\bf ++}}

\usepackage[usenames]{color}
\usepackage{colortbl}


\usepackage[unicode,colorlinks,urlcolor=cyan]{hyperref}
\usepackage{url}
\usepackage{fancyhdr}

\usepackage{tikz}
\usetikzlibrary{automata,shapes,arrows,positioning,calc}

\usepackage{clrscode}

\usepackage{sectsty}
\usepackage{verbatim}
\usepackage{ifthen}
\usepackage{forloop}
\usepackage{calc}
\usepackage{fullpage}

%\pagestyle{empty}

\newcommand{\nspace}{\hspace{0pt}}
\newcommand{\nbdash}{\nobreakdash-\nspace}
\newcommand{\Rats}{\ensuremath{\mathbb Q}}
\newcommand{\Ints}{\ensuremath{\mathbb Z}}
\DeclareMathOperator{\Gal}{Gal}
% cyrillic enumerate enumerate
\renewcommand{\theenumii}{\asbuk{enumii}}

\newtheorem{thm}{Theorem}%[section]
% \renewcommand{\theThm}{\arabic{Thm}}
% \newtheorem{Lmm}{Лемма}
% \newtheorem*{Cor}{Следствие}
\theoremstyle{definition}
\newtheorem{defn}{Definition}
% \newtheorem*{Remark}{Замечание}
% \newtheorem*{Algo}{Алгоритм}
% \newtheorem{NumAlgo}{Алгоритм}
% \newcommand{\N}{\ensuremath{\mathbb N}}

\frenchspacing

\begin{document}
\begin{center}{\Large
Galois group of cubic
}\end{center}

\begin{defn} If $f(x)\in k[x]$, the \emph{Galois group of $f(x)$} is the Galois
group $\Gal(K/k)$ of a splitting field $K$ of $f(x)$.
\end{defn}

\begin{thm} For $f(x)\in k[x]$, the Galois group of $f(x)$ permutes the set of
roots of $f(x)$. Therefore, if the roots of $f(x)$ are
$\alpha_1,\ldots,\alpha_n\in K$, the Galois group of $f(x)$ is isomorphic to a
subgroup of $S_n$.
\end{thm}
\begin{proof}
$K=k(\alpha_1,\ldots,\alpha_n)$, so any automorphism $\sigma$ of
$K$ fixing $k$ is determined by the image of each $\alpha_i$. But $\sigma$ must
take each $\alpha_i$ to some $\alpha_j$ (where possibly $i=j$), since $\sigma$
is a homomorphism of $K$ and thus $f(\sigma(\alpha_i))=\sigma(f(\alpha_i))=0$.
Thus $\sigma$ permutes the roots of $f(x)$ and is determined by the resulting
permutation.
\end{proof}

We now restrict our attention to the case $k=\Rats$. If $f(x)\in\Rats[x]$ is a
cubic, its Galois group is a subgroup of $S_3$. We can then use the knowledge of
the group structure of $S_3$ to anticipate the possible Galois groups of a cubic
polynomial. There are six subgroups of $S_3$, and the three subgroups of order
$2$ are conjugate. This leaves four essentially different subgroups of $S_3$:
the trivial group, the group $\langle(1,2)\rangle$ that consists of a single
transposition, the group $A_3=\langle (1,2,3)\rangle$, and the full group $S_3$.
All four of these groups can in fact appear as the Galois group of a cubic.

Let $K$ be a splitting field of $f(x)$ over $\Rats$.

If $f(x)$ splits completely in $\Rats$, then $K=\Rats$ and so the Galois group
of $f(x)$ is trivial. So any cubic (in fact, a polynomial of any degree) that
factors completely into linear factors in $\Rats$ will have trivial Galois
group.

If $f(x)$ factors into a linear and an irreducible quadratic term, then
$K=\Rats(\sqrt{D})$, where $D$ is the discriminant of the quadratic. Hence
$[K:\Rats]=2$ and the order of $\Gal(K/\Rats)$ is $2$, so $\Gal(K/\Rats)\cong
\langle1,2\rangle\cong\Ints/2\Ints$; the nontrivial element of the Galois group
takes each root of the quadratic to its complex conjugate (i.e. it maps
$\sqrt{D}\mapsto -\sqrt{D}$). Thus any cubic that has exactly one rational root
will have Galois group isomorphic to $\langle 1,2\rangle\cong \Ints/2\Ints$.

The Galois groups $A_3$ and $S_3$ arise when considering irreducible cubics. Let
$f(x)$ is irreducible with roots $r_1, r_2, r_3$. Since $f$ is irreducible, the
roots are distinct. Thus $\Gal(K/\Rats)$ has at least $3$ elements, since the
image of $r_1$ may be any of the three roots. Since $\Gal(K/\Rats)\subset S_3$,
it follows that $\Gal(K/\Rats)\cong A_3$ or $\Gal(K/\Rats)\cong S_3$ and thus by
the fundamental theorem of Galois theory that $[K:\Rats]=3$ or $6$.

Now, the discriminant of $f$ is \[D=\prod_{i<j} (r_i-r_j)^2\] This is a
symmetric polynomial in the $r_i$. The coefficients of $f(x)$ are the elementary
symmetric polynomials in the $r_i$: if $f(x)=x^3+ax^2+bx+c$, then
\begin{gather*}
c=r_1r_2r_3\\
b=-(r_1r_2+r_1r_3+r_2r_3)\\
a=r_1+r_2+r_3
\end{gather*}
Thus $D$ can be written as a polynomial in the coefficients of $f$, so
$D\in\Rats$. $D\neq 0$ since $f(x)$ is irreducible and therefore has distinct
roots; also clearly $\sqrt{D}\in K$ and thus $\Rats(r_1,\sqrt{D})\subset K$. If
$f(x)=x^3+ax^2+bx+c$, then its discriminant $D$ is $18abc + a^2 b^2 - 4b^3 -
4a^3 c - 27 c^2$ (see the article on the discriminant for a longer discussion).

If $\sqrt{D}\notin\Rats$, it follows that $\sqrt{D}$ has degree $2$ over
$\Rats$, so that $[\Rats(r_1,\sqrt{D}):\Rats]=6$. Hence $K=\Rats(r_1,\sqrt{D})$,
so we can derive the splitting field for $f$ by adjoining any root of $f$ and
the square root of the discriminant. This can happen for either positive or
negative $D$, clearly. Note in particular that if $D<0$, then $\sqrt{D}$ is
imaginary and thus $K$ is not a real field, so that $f$ has one real and two
imaginary roots. So any cubic with only one real root has Galois group $S_3$.

If $\sqrt{D}\in\Rats$, then any element of $\Gal(K/\Rats)$ must fix $\sqrt{D}$.
But a transposition of two roots does not fix $\sqrt{D}$ - for example, the map
\[r_1\mapsto r_2, \qquad r_2\mapsto r_1, \qquad r_3\mapsto r_3\] takes
\[\sqrt{D}=(r_1-r_2)(r_1-r_3)(r_2-r_3)\mapsto
(r_2-r_1)(r_2-r_3)(r_1-r_3)=-\sqrt{D}\] Then $\Gal(K/\Rats)$ does not include
transpositions and so it must in this case be isomorphic to $A_3$. Thus
$[K:\Rats]=3$, so $K=\Rats(r_1)=\Rats(r_1,\sqrt{D})$ since $\sqrt{D}\in\Rats$.
This proves:

\begin{thm} Let $f(x)\in\Rats[x]$ be an irreducible cubic and $K$ its splitting
field. Then if $\alpha$ is any root of $f$, \[K=\Rats(\alpha,\sqrt{D})\] where
$D$ is the discriminant of $f$. Thus if $\sqrt{D}$ is rational, $[K:\Rats]=3$
and the Galois group is isomorphic to $A_3$, otherwise $[K:\Rats]=6$ and the
Galois group is isomorphic to $S_3$.
\end{thm}

Note that one consequence of all of this is that any irreducible cubic with
three real roots must have Galois group $A_3$ (and thus any irreducible cubic
with Galois group $S_3$ must have two complex roots), but the converse does not
hold.

One way of looking at the above analysis is that for a
``general'' polynomial of degree $n$, the Galois group is $S_n$. If the Galois
group of some polynomial is not $S_n$, there must be algebraic relations among
the roots that restrict the available set of permutations. In the case of a
cubic whose discriminant is a rational square, this relation is that $\sqrt{D}$,
which is a polynomial in the roots, must be preserved.

\begin{enumerate}
\renewcommand{\labelenumi}{Example \arabic{enumi}}

\item $f(x)=x^3-6x^2+11x-6$. By the rational root test, this polynomial has the three rational roots $1,2,3$, so it factors as $f(x)=(x-1)(x-2)(x-3)$ over $\Rats$. Its Galois group is therefore trivial.

\item $f(x)=x^3-x^2+x-1$. Again by the rational root test, this polynomial factors as $(x-1)(x^2+1)$, so its Galois group has two elements, and a splitting field $K$ for $f$ is derived by adjoining the square root of the discriminant of the quadratic: $K=\Rats(\sqrt{-1})$. The nontrivial element of the Galois group maps $\sqrt{-1}\leftrightarrow -\sqrt{-1}$.

\item $f(x)=x^3-2$. This polynomial has discriminant $-108=-3\cdot 6^2$. This is not a rational square, so\ the Galois group of $f$ over $\Rats$ is $S_3$, and the splitting field for $f$ is $\Rats(\sqrt[3]{2},\sqrt{-108})=\Rats(\sqrt[3]{2},\sqrt{-3})$. This is in agreement with what we already know, namely that the cube roots of $2$ are
\[\sqrt[3]{2}, \omega\sqrt[3]{2}, \omega^2\sqrt[3]{2}\]
where $\omega=\frac{-1+\sqrt{-3}}{2}$ is a primitive cube root of unity.

\item $f(x)=x^3-4x+2$. This is irreducible since it is Eisenstein at $2$ (or by the rational root test), and its discriminant is $202$, which is not a rational square. Thus the Galois group for this polynomial is also $S_3$; note, however, that $f$ has three real roots (since $f(0)>0$ but $f(1)<0$).

\item $f(x)=x^3-3x+1$. This is also irreducible by the rational root test. Its discriminant is $81$, which is a rational square, so the Galois group for this polynomial is $A_3$. Explicitly, the roots of $f(x)$ are
\[r_1=2\cos(2\pi/9), r_2=2\cos(8\pi/9), r_3=2\cos(14\pi/9)\]
and we see that
\begin{gather*}
\cos(14\pi/9)=\cos(4\pi/9)=2\cos^2(2\pi/9)-1\\
\cos(8\pi/9)=2\cos^2(4\pi/9)-1
\end{gather*}
Let's consider an automorphism of $K$ sending $r_1$ to $r_3$, i.e. sending $\cos(2\pi/9)\mapsto\cos(14\pi/9)$. Given the relations above, it is clear that this mapping uniquely determines the image of $r_3$ as well, since
\[r_3=2\cos^2(2\pi/9)-1\mapsto 2\cos^2(4\pi/9)-1=r_2\]
and thus we see how the relation imposed by the discriminant actually manifests itself in terms of restrictions on the permutation group.
\end{enumerate}
\end{document}