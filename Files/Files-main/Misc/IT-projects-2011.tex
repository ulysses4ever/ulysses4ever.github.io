\documentclass[a4paper,12pt]{article}

\lccode`\-=`\-
\defaulthyphenchar=127

\usepackage[utf8]{inputenc}
\usepackage{latexsym}
\usepackage{amsfonts}
\usepackage{amssymb, amsmath}
\usepackage{amsthm} % for {proof}
\usepackage{indentfirst}
\usepackage{cmap} % copy text from pdf
%\usepackage[T2A]{fontenc}
\usepackage{geometry}
\geometry{a4paper}

\usepackage[usenames]{color}
\usepackage{colortbl}

\usepackage[russian]{babel}

\usepackage[colorlinks,urlcolor=cyan]{hyperref}
\usepackage{url}
\usepackage{fancyhdr}


\pagestyle{fancy}

\newcommand{\nspace}{\hspace{0pt}}
\newcommand{\nbdash}{\nobreakdash-\nspace}

\renewcommand{\thesubsection}{\arabic{subsection}.}

\frenchspacing

\begin{document}
\subsection{Комбинаторика на словах}
Относительно молодой раздел дискретной математики, изучающий свойства формальных языков и отдельных слов средствами комбинаторики. Основополагающим трудом в этой области является серия из трёх книг M.~Lothaire. Здесь встречается значительное число интересных алгоритмов. Например, алгоритмы для взвешенных автоматов (аналогичные алгоритмам для обычных конечных автоматов) и их использование для статистической обработки естественных языков (M.~Lothaire, Applied Combinatorics On Words, chap.~4). Кроме реализации отдельных алгоритмов важной частью работы стало бы написание обзора развития дисциплины по первой книге M.~Lothaire, так как эта область, насколько мне известно, плохо представлена в русскоязычной литературе. 

\end{document}
