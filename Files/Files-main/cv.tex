%% start of file `template.tex'.
%% Copyright 2006-2012 Xavier Danaux (xdanaux@gmail.com).
%
% This work may be distributed and/or modified under the
% conditions of the LaTeX Project Public License version 1.3c,
% available at http://www.latex-project.org/lppl/.


\documentclass[11pt,a4paper]{moderncv}   % possible options include font size ('10pt', '11pt' and '12pt'), paper size ('a4paper', 'letterpaper', 'a5paper', 'legalpaper', 'executivepaper' and 'landscape') and font family ('sans' and 'roman')

%\usepackage{url}
\newcommand{\CPP}
{C\nolinebreak[4]\hspace{-.05em}\raisebox{.35ex}{\scriptsize\bfseries +\hspace{-.05em}+}}

\newcommand{\CSH}
{C\hspace{-.05em}{\raisebox{0.3ex}{\scriptsize\#}}}
%\usepackage{hyperref}
%\hypersetup{colorlinks=false,urlbordercolor=cyan,linkbordercolor=red}
%\usepackage[T2A]{fontenc}
%\usepackage{cmap}

%\hypersetup{urlcolor=blue}
\newcommand{\myhref}[2]{\textcolor{blue}{\href{#1}{#2}}}

% moderncv themes\footnotesize
\moderncvtheme[grey]{classic}
%\moderncvstyle{oldstyle}                        % style options are 'casual' (default), 'classic', 'oldstyle' and 'banking'
%\moderncvcolor{blue}                          % color options 'blue' (default), 'orange', 'green', 'red', 'purple', 'grey' and 'black'
\renewcommand{\familydefault}{\sfdefault}    % to set the default font; use '\sfdefault' for the default sans serif font, '\rmdefault' for the default roman one, or any tex font name
%\nopagenumbers{}                             % uncomment to suppress automatic page numbering for CVs longer than one page

% character encoding
\usepackage[utf8]{inputenc}                  % if you are not using xelatex ou lualatex, replace by the encoding you are using
%\usepackage{CJKutf8}                         % if you need to use CJK to typeset your resume in Chinese, Japanese or Korean

% adjust the page margins
\usepackage[scale=0.8]{geometry}
%\setlength{\hintscolumnwidth}{3cm}           % if you want to change the width of the column with the dates
%\setlength{\maketitlenamewidth}{10cm}        % for the 'classic' style, if you want to force the width allocated to your name and avoid line breaks. be careful though, the length is normally calculated to avoid any overlap with your personal info; use this at your own typographical risks...

% personal data
\firstname{Artem}
\familyname{Pelenitsyn}
\title{Curriculum Vitæ}               % optional, remove the line if not wanted
\address{50 South Huntington, apt. 22}{02130 Boston, MA, USA}    % optional, remove the line if not wanted
\mobile{+1~(857)~204~4460}
%\mobile{+7~(961)~290~2878}                     % optional, remove the line if not wanted
%\phone{+7~(863)~237~7831}                      % optional, remove the line if not wanted
%\fax{+3~(456)~789~012}                        % optional, remove the line if not wanted
%\email{ulysses4ever@gmail.com}                          % optional, remove the line if not wanted
\email{artem@ccs.neu.edu}
\homepage{mmcs.sfedu.ru/~ulysses}                    % optional, remove the line if not wanted
%\extrainfo{Current place of living: Rostov-na-Donu, Russia}            % optional, remove the line if not wanted
\photo{neu-2018.jpg}%[64pt][0.4pt]                  % '64pt' is the height the picture must be resized to, 0.4pt is the thickness of the frame around it (put it to 0pt for no frame) and 'picture' is the name of the picture file; optional, remove the line if not wanted
%\quote{Some quote (optional)}                 % optional, remove the line if not wanted

% to show numerical labels in the bibliography (default is to show no labels); only useful if you make citations in your resume
%\makeatletter
%\renewcommand*{\bibliographyitemlabel}{\@biblabel{\arabic{enumiv}}}
%\makeatother

% bibliography with mutiple entries
%\usepackage{multibib}
%\newcites{book,misc}{{Books},{Others}}
%----------------------------------------------------------------------------------
%            content
%----------------------------------------------------------------------------------
\begin{document}
%\begin{CJK*}{UTF8}{gbsn}                     % to typeset your resume in Chinese using CJK
\maketitle

\section{Education}
\cventry%
    {2003--2007}%
    {B.Sc. in Applied Mathematics and Computer Science}%
    {Southern Federal University}%
    {Rostov-na-Donu, Russia}%
    {\myhref%
        {http://mmcs.sfedu.ru/~ulysses/Edu/Diploma/scans/diploma-euro-supplement.pdf}%
        {link to the transcript}}  % arguments 3 to 6 can be left empty
    {Major: Foundations and Software Engineering for Computer Science}
\cventry%
    {2007--2009}%
    {M.Sc. in Applied Mathematics and Computer Science}%
    {Southern Federal University}%
    {Rostov-na-Donu, Russia}%
    {\myhref{http://mmcs.sfedu.ru/~ulysses/Edu/Diploma/scans/diploma-euro-supplement.pdf}{link to the transcript}}%
    {Major: Foundations and Software Engineering for Computer Science}

\section{Master thesis}
\cvitem{title}{\emph{BMS-algorithm and its application to decoding}}
\cvitem{supervisor}{Prof. V.M.~Deundyak}
%\cvitem{description}{Short thesis abstract}

\section{Research interests}

Staticaly typed functional programming.

Mix-in of languages with different paradigms, in particular by the way of eDSLs. Currently working on a Datalog-like environment on top of the Julia programming language.

\section{Experience (summary)}

\subsection{Occupation}

\cventry{2018--present}{PhD student}{Northeastern U.
}{Boston, USA}{}{%\newline{}
%Courses:
}
\cventry{Fall 2017}{Research assistant at \myhref{https://prl-prg.github.io}{Programming Research Laboratory}}{Czech Technical University}{Prague, Czech Republic}{}{%\newline{}
%Courses:
}
\cventry{Spring 2017}{Research assistant at \myhref{http://prl.ccs.neu.edu/}{Programming Research Laboratory}}{Northeastern University}{Boston, USA}{}{%\newline{}
%Courses:
}
\cventry{2010--2011, 2012--2017}{Assistant professor, lecturer}{Southern Federal University
}{Rostov-na-Donu, Russia}{}{%\newline{}
%Courses:
}

\section{Publications}

\subsection{Peer-reviewed International}

\cvlistitem{%
Julia Subtyping: a Rational Reconstruction (with F. Zappa Nardelli, J. Belyakova, B. Chung, J. Bezanson, J. Vitek) // 
In: Proc. ACM Program. Lang., Vol. 2, Issue OOPSLA, 2018.
\texttt{DOI: 10.1145/3276483} 
\myhref{https://www.di.ens.fr/~zappa/projects/lambdajulia/paper.pdf}{[PDF]}}

\cvlistitem{%
Functional Parser of Markdown Language Based on Monad Combining and Monoidal Source Stream Representation (with G.Lukyanov) // In: Itsykson V., Scedrov A., Zakharov V. (eds) Tools and Methods of Program Analysis. TMPA 2017. CCIS, vol 779, pp. 90--101. Springer, Cham. \texttt{DOI: 10.1007/978-3-319-71734-0\_8} \myhref{http://mmcs.sfedu.ru/~ulysses/Papers/2018-TMPA-effects-vs-transformers-in-parsing.pdf}{[PDF]}}

\cvlistitem{%
Associated Types and Constraint Propagation for Generic Programming in Scala // “Programming and Computer Software” (english trans. of “Programmirovanie”), 2015, No 4, pp. 224–230. \texttt{DOI: 10.1134/S0361768815040064} \myhref{http://mmcs.sfedu.ru/~ulysses/Papers/2015-PCS-Scala-generics.pdf}{[PDF]}.}

\subsection{Drafts}

\cvlistitem{%
Fuzzy-Testing A Subtyping Relation // 2018
\myhref{http://mmcs.sfedu.ru/~ulysses/Papers/2018-unpb-subtype-fuzzer.pdf}{[PDF]}}

\cvlistitem{%
Handling Recursion in Generic Programming Using Closed Type Families
(with A. Bolotina) // 2018
\myhref{http://mmcs.sfedu.ru/~ulysses/Papers/2018-unpb-dgp-recursion.pdf}{[PDF]}}


{\footnotesize
\subsection{Russian}

\cvlistitem{%
Buliding parsers with algebraic effects // Proceedings of the First Russian Conference on Programming Languages and Compilers (PLC'17), 2017, pp. 185–190. With G. Lukyanov.}

\cvlistitem{%
Pelenitsyn A. Generic and meta- programming approach to design of software implementation of decoder for a class of algebraic geometry codes // “Prikladnaya informatika” (Applied computer science), 2012, No 2(38), pp. 60–70. \myhref{http://mmcs.sfedu.ru/~ulysses/Papers/2012-PriklInf-metaprogramming-to-decoding.pdf}{[PDF]}, \myhref{http://staff.mmcs.sfedu.ru/~ulysses/Papers/2012-prepared-for-wgp.pdf}{link to the draft in English}.%
}

\cvlistitem{%
Pelenitsyn A. On exploiting one metaprogramming technique. Journal of the Ivanovo Mathematical Society, 2011, No. 1(8), pp.79–84. \myhref{http://mmcs.sfedu.ru/~ulysses/Papers/2011-Ivanovo-metaprogramming.pdf}{[PDF]}.%
}

\cvlistitem{%
Deundyak V., Pelenitsyn A. Operator-theoretic approach to Berlekamp--Massey Algorithm, // Izvestia vuzov (Universities' Bulletin), Sev.-Kav. Region (Caucasus Region), Estestvennie Nauki (Sciences), 2011, No. 3. Pp. 11–13.
\myhref{http://mmcs.sfedu.ru/~ulysses/Papers/2011-Izv-vuzov-BMSA-through-operator-theory.pdf}{[PDF]}.}

\cvlistitem{%
Mayevskiy A., Pelenitsyn A. Software Implementation of Algebraic-Geometry Codec using Sakata algorithm, // Izvestia Yufu (Southern Federal University Bulletin), Technology Sciences, 2008, No. 8, pp. 196–198.
\myhref{http://mmcs.sfedu.ru/~ulysses/Papers/2008-2-Izvestia-AGCodec-Sakata.pdf}{[PDF].}}

}

\subsection{In Conference Transactions (Russian)}

{\footnotesize
\cvlistitem{%
Pelenitsyn A. On Implementation of n-Dimensional BMS-algorithm Using Generic Programming // Transactions of Scientific School of I.B. Simonenko, 2010, pp. 197–203.
\myhref{http://mmcs.sfedu.ru/~ulysses/Papers/2010-2-Sbor-Simonenko-BMSA-impl.pdf}{[PDF] (in Russian)}.}

\cvlistitem{%
Mayevskiy A., Pelenitsyn A. Methodic Supply and IT-infrastructure for Teaching Low-Level Programming // Transactions of Scientific-Methodic Conference “Modern Information Technologies in Education”, 2010, pp. 210–212.
\myhref{http://mmcs.sfedu.ru/~ulysses/Papers/2010-1-SITO-abstract-preprint.pdf}{[PDF] (in Russian)}.}

\cvlistitem{%
Mayevskiy A., Pelenitsyn A. On Software Implementation of Algebraic-Geometry Codec using Sakata algorithm, // Transactions of X International Conference on Information Security and Safety, 2008, pp. 55–57.}

\cvlistitem{%
Pelenitsyn A. On Implementation of Decoder for a Class of Algebraic-Geometry Codes on Projectve Curves using Sakata algorithm, // Transactions of the Conference "Week of Science" in Southern Federal University, 2008, vol. 1, pp. 55–57.
\myhref{http://mmcs.sfedu.ru/~ulysses/Papers/2008-1-SciWeek-Decoder-Sakata-preprint.pdf}{[PDF] (in Russian)}.}

\cvlistitem{%
Bragilevsky V., Mihalkovich S., Pelenitsyn A. Building Web-portal for Information and Education purposes on Computing Department // Transactions of Scientific-Methodic Conference “Modern Information Technologies in Education”, 2008, pp. 48–49.
\myhref{http://mmcs.sfedu.ru/~ulysses/Papers/2008-3-SITO-IT-portal-preprint.pdf}{[PDF] (in Russian)}.}

}

\section{Conference Talks: Research}

\subsection{International}

\cventry{2018}%
    {ACM SIGPLAN Symposium on Scala, 2018}%
    {Student Talk ``Julia Subtyping Lessons Scala Could Learn''}
    {St. Louis, USA, 2018 (co-located with ICFP)}{}%
    {\myhref{%
        https://conf.researchr.org/event/scala-2018/scala-2018-papers-julia-subtyping-lessons-scala-could-learn-student-talk-}{Link to the abstract},
      \myhref{http://staff.mmcs.sfedu.ru/~ulysses/Talks/2018-lj-scalasymp/}{link to the slides}}


\cventry{2018}%
    {2nd Workshop on Machine Learning Techniques for Programming Languages}%
    {Talk ``Can We Learn Some PL Theory? How To Make Use of a Corpus of Subtype Checks''}
    {Amsterdam, The Netherlands, 2018 (co-located with ECOOP/ISSTA)}{}%
    {\myhref{%
https://conf.researchr.org/event/ecoop-issta-2018/ml4pl-2018-papers-can-we-learn-some-pl-theory-how-to-make-use-of-a-corpus-of-subtype-checks}{Link to the abstract}}

{\footnotesize
\subsection{Russian}

\cventry{2015}%
    {Scientific Conference “Modern Information Technologies and IT-Education”}%
    {talk “\protect\CPP{}17 Concepts in their relation to \protect\CPP{}0x ones”}
    {Lomonosov Moscov State University, Faculty of Computational Mathematics and Cybernetics}{}%
    {\myhref{http://conf.it-edu.ru/conference/2015/programm}{Link to the web-site} (in Russian),
\myhref{http://staff.mmcs.sfedu.ru/~ulysses/Papers/Talks/2015-SITITO-Cpp1z-concepts.pdf}{link to the slides (in Russian)}.}

\cventry{2012}%
    {Research and Pratice Conference: Free Open Source Software “FOSS Lviv 2012”}%
    {talk “Software Implementation of Decoder For a Class Of Error-Correcting Codes on Algebraic Curves: Designing on a Basis of Generic Metaprogramming Templates”}%
    {Ivan Franko National University of Lviv, Lviv, Ukraine}{}%
    {\myhref{http://conference.linux.lviv.ua/en/main}{Link to the web-site},
\myhref{http://mmcs.sfedu.ru/~ulysses/Papers/Talks/2012-FOSS-Lviv.pdf}{link to the slides (in Russian)}.}

\cventry{2008}%
    {Conference “Week of Science” in Southern Federal University}%
    {talk “On Implementation of Decoder for a Class of Algebraic-Geometry Codes on Projectve Curves using Sakata algorithm”}%
    {Rostov-na-Donu, Russia}{}%
    {\myhref{http://mmcs.sfedu.ru/~ulysses/Papers/Talks/2008-SFedU-SciWeek-slides.pdf}{Link to the slides (in Russian)}.}
}

\section{Seminar Talks}
{\footnotesize
  \subsection{In English}
  
  %% \cventry{year}%
  %%         {title}
  %%         {occasion}
  %%         {place}
  %%         {}
  %%         {\myhref{url}{Link title}}

  \cventry{2019}%
          {Parametricity Goes Gradual}
          {\myhref{http://www.ccs.neu.edu/home/amal/course/7480-s19/}{CS7480}}
          {Northeastern U., USA}
          {}
          {\myhref{http://staff.mmcs.sfedu.ru/~ulysses/Talks/2019-cs7480-gradual-parametricity.pdf}{Link to the slides}}
  
  \cventry{2019}%
          {A Spectrum of Type Soundness and Performance}
          {\myhref{http://www.ccs.neu.edu/home/amal/course/7480-s19/}{CS7480}}
          {Northeastern U., USA}
          {}
          {\myhref{http://staff.mmcs.sfedu.ru/~ulysses/Talks/2019-cs7480-gradual-spectrum.pdf}{Link to the slides}}

  \cventry{2018}%
          {Linear Haskell}
          {\myhref{https://prl-prg.github.io/}{PRL@PRG} seminar}
          {CVUT in Prague, Czechia}
          {}
          {\myhref{http://staff.mmcs.sfedu.ru/~ulysses/Talks/2018-linear-haskell/}{Link to the slides}}

          
\subsection{In Russian}
%\cventry{year--year}{Job title}{Employer}{City}{}{Description line 1\newline{}Description line 2}

\cventry{2018}%
    {Julia Subtyping: A Rational Reconstructions}
    {Programming Languages and Compilers seminar, MMCS, SFedU}
    {Rostov-on-Don, Russia}%
    {}%
    {\myhref{http://staff.mmcs.sfedu.ru/~ulysses/Talks/2018-lj-plc/}{Link to the slides}}

    
    \cventry{2016}
            {Functional Visitors}
            {Programming Languages and Compilers seminar, MMCS, SFedU}
            {Rostov-on-Don, Russia}
            {}
            {Link to the slides: \myhref{http://forum.mmcs.sfedu.ru/uploads/default/original/1X/87bd89c2abeef38c7ff835cecd2485c24b1e78c8.pdf}{Part I}, \myhref{http://forum.mmcs.sfedu.ru/uploads/default/original/1X/5fee3e64f4528aaf68fc4665b21ed3173a97497d.pdf}{Part II}}

\cventry{2016}{Seminar on Galois Theory}{}{Institute for Mathematics, Mechanics and Computer Science, Southern Federal University, Rostov-na-Donu}{\myhref{https://docs.google.com/document/d/1hCrg3VZDxYAHygG_DnrP_--k_qmGMrNFvtKxEOGR790/edit?usp=sharing}{Link to the syllabus}}{}

\cventry{2011}{Minicourse on Galois Theory}{Algebra seminar}{Faculty for Mathematics, Mechanics and Computer Science, Southern Federal University, Rostov-na-Donu}{}{}

\cventry{2011}%
    {Talks “Foundations for programming Languages”, “Automata and Formal Languages”}%
    {seminar for undergraduates “Introduction to Theoretical Computer Science”}{Faculty for Mathematics, Mechanics and Computer Science, Southern Federal University, Rostov-na-Donu}{}%
    {\myhref{http://mmcs.sfedu.ru/~ulysses/Papers/Talks/2011-ITCS-FPL.pdf}{Link to the slides (in Russian)}}

\cventry{2009}{Talk “Higher-Order Computations and Model Checking”}{Interchair seminar on Computer Science}{Faculty for Mathematics, Mechanics and Computer Science, Southern Federal University, Rostov-na-Donu}{}%
    {\myhref{http://mmcs.sfedu.ru/~ulysses/Papers/Talks/2009-10-19-hoc-model-checking.pdf}{Link to the slides (in Russian)}}

\cventry{2009}%
    {Talk “On multi-dimensional version of Berlekamp-Massey algorithm”}%
    {Seminar on Mathematical Methods in Information Safety and Security}%
    {Faculty for Mathematics, Mechanics and Computer Science, Southern Federal University, Rostov-na-Donu}%
    {}%
    {\myhref{http://mmcs.sfedu.ru/~ulysses/Papers/Talks/2009-10-16,30-bmsa.pdf}{Link to the slides (in Russian)}}

\cventry{2009}%
    {Talk “Inductive Data Types in Programming”}%
    {Seminar on Category Theory}%
    {Faculty for Mathematics, Mechanics and Computer Science, Southern Federal University, Rostov-na-Donu}%
    {}%
    {\myhref{http://mmcs.sfedu.ru/~ulysses/Papers/Talks/2009-03-19-CT-inducttypes.pdf}{Link to the slides (in Russian)}}


\cventry{2008}%
    {Talk “Spring Framework”}%
    {Rostov Java User Group}%
    {Computing Center of Southern Federal University, Rostov-na-Donu}%
    {}%
    {\myhref{http://mmcs.sfedu.ru/~ulysses/Papers/Talks/2008-RostovJUG-SpringFramework.zip}{Link to the slides (xul format -- to be run in Mozilla Firefox browser; in Russian)}}
}

\section{Conference Talks: Education, Technology, Popular Science}

\subsection{International}

\cventry{2014}%
    {Joint International Program For Scientific and Technology Cooperation}%
    {talk “Computer Science Projects Developed inside (in connection with) Department of Mathematics, Mechanics and Computer Sciences / SFedU”}
    {Sao Paulo, Rio de Janeiro, Fortaleza, Brasil}{}%
    {\myhref{http://sfedu.ru/www/sfedu$news$.show_full?p_nws_id=46741}{Info on university web-site} (in Russian),
\myhref{http://staff.mmcs.sfedu.ru/~ulysses/Papers/Talks/2014-BRICS-University-Forum.pdf}{link to the slides}.}

{\footnotesize
\subsection{Russian}

\cventry{2015}%
    {Scientific Conference “Modern Information Technologies in Education”}%
    {talk “Store and publication assignment infrastructure for Moodle LMS”}
    {Institute for Mathematics, Mechanics and Computer Science in honour of I.\,I.~Vorovich, Rostov-na-Donu, Russia}{}%
    {\myhref{http://inftech.uginfo.sfedu.ru/}{Link to the web-site} (in Russian),
\myhref{http://staff.mmcs.sfedu.ru/~ulysses/Papers/Talks/2015-SITO-Moodle-publication.pdf}{link to the slides (in Russian)}.}

\cventry{2010}%
    {Scientific-Methodic Conference “Modern Information Technologies in Education”}%
    {talk “Methodic Supply and IT-infrastructure for Teaching Low-Level Programming”}%
    {Computing Center of Southern Federal University, Rostov-na-Donu, Russia}{}%
    {\myhref{http://conf.sfedu.ru/inftech2010/}{Link to the web-site} (in Russian),
\myhref{http://mmcs.sfedu.ru/~ulysses/Papers/Talks/2010-SITO-Assembly-programming.pdf}{link to the slides (in Russian)}.}

\cventry{2008}%
    {International Conference on Information Security and Safety}%
    {talk “Building Web-portal for Information and Education purposes on Computing Department”}{Taganrog, Russia}{}%
    {\myhref{http://bit.tti.sfedu.ru/?q=en/node/15}{Link to the web-site}, or \myhref{http://bit.tti.sfedu.ru/?q=ru/node/6}{link to the expanded version in Russian}.}
}
\section{Personal awards, scholarships, etc.}
\cventry{2012}{Participation in all-russian final of international student olympiad “IT-planet”}{}{}{competition: “Oracle Java Olympic”}{\myhref{http://mmcs.sfedu.ru/~ulysses/Papers/Trainings/2012-ITPlanet-final.pdf}{Link to Diploma for Participation} (in Russian)}.

\cventry{2012}{Diploma for taking second place in regional stage of international student olympiad “IT-planet”}{}{}{competition: “Oracle Java Olympic”}{\myhref{http://mmcs.sfedu.ru/~ulysses/Papers/Trainings/2012-ITPlanet.jpg}{Link to Diploma scan} (in Russian)}

\cventry{2012}{Participation in the final stage of VI Open Programming Contest of Southern Federal University}{}{}{individual event}{\myhref{http://mmcs.sfedu.ru/~ulysses/Papers/Trainings/2012-TTI-olymp-participation.jpg}{Link to Diploma for Participation} (in Russian)}.

\cventry{2011}{Scholarship from foundation "Education and Science on the South of Russia"}{}{}{}{\myhref{http://mmcs.sfedu.ru/~ulysses/Papers/Trainings/2011-CI-cert.jpg}{Link to the scholarship statement scan} (in Russian)}

\cventry{2011}{Rector's commendation for participating in international accreditation of unversity teaching programmes}{Southern Federal University}{}{}{\myhref{http://mmcs.sfedu.ru/~ulysses/Papers/Trainings/2011-rector-commendation.pdf}{Link to scan of the commendation text} (in Russian)}

\cventry{2008}{Diploma for the best talk}{student session during annual “Week of Science”, Southern Federal University}{}{}
{\myhref%
{http://mmcs.sfedu.ru/~ulysses/Papers/Talks/2008-SFedU-SciWeek-diploma.jpg}
{Link to Diploma scan} (in Russian)}

\section{Community Service}

\subsection{Conference Organization}

\cvitem{\myhref{https://conf.researchr.org/committee/etaps-2019/etaps-2019-organizing-committee}{ETAPS '19}}{Web Co-Chair}

\cvitem{\myhref{https://conf.researchr.org/committee/ecoop-issta-2018/ml4pl-2018-papers-organizing-committee}{ML4PL '18}}{Organizer}

\cvitem{\myhref{http://plc.sfedu.ru/organizers.html}{PLC '17}}{Organizer}

\subsection{Book Translations (English to Russian)}

\cvlistitem{%
Dowek, Gilles, Levy, Jean-Jacques. Introduction to the Theory of Programming Languages. / Springer. 2011. Russian translation together with V.~Bragilevskiy. Published by DMK Press in 2013. \myhref{http://www.xn--d1amf.xn--p1ai/catalog/computer/programming/978-5-94074-913-4/}{Link to web page},
\myhref{http://books.google.ru/books?id=YR7RAAAAQBAJ&printsec=frontcover}{link to Google.Books preview.}
}

\cvlistitem{%
Bird, Richard. Pearls of Functional Algorithm Design. / Cambridge University Press. 2010. Russian translation together with V.~Bragilevskiy. Published by DMK Press in 2013. \myhref{http://www.xn--d1amf.xn--p1ai/catalog/computer/programming/978-5-94074-867-0/}{Link to web page},
\myhref{http://books.google.ru/books?id=JSjRAAAAQBAJ&printsec=frontcover}{link to Google.Books preview.}
}

\section{Experience (detailed)}

\subsection{Teaching (at \myhref{http://sfedu.ru/international/}{Southern Federal University}, mostly in Russian)}

    \cvlistitem {Quantum Computations (lectures in English) --- 2016 (fall).}
    \cvlistitem {Computer Architecture (lectures \& labs) --- 2013–2016 (spring).}
    \cvlistitem {Automata and Ciphers (lectures) --- 2013–2016 (fall).}
    \cvlistitem {Programming Basics labs --- 2008, 2010–2012, 2014--2016.}
    \cvlistitem {Programming Languages labs --- 2008, 2010, 2012--2015 (fall).}
    \cvlistitem {Functional Programming labs --- 2011 (spring).}
    \cvlistitem {Automata and Languages --- 2010 (spring).}
    \cvlistitem {Microprogramming/Assembler Programming labs --- 2009 (fall).}
    \cvlistitem {Geometry and Algebra --- 2009 (fall).}

\subsection{Supervising student projects}

    \cvlistitem {\textit{Structuring Effectful Computations} --- MSc~G.~Lukyanov,~2017,~\myhref{http://staff.mmcs.sfedu.ru/~ulysses/Edu/tutoring/2017/Lukyanov/text.pdf}{[PDF]}}
    \cvlistitem {\textit{Generic Programming and Zippers} --- A.~Bolotina,~2017}
    \cvlistitem {\textit{Generation of algebraic data types descriptions based on JSON data via Template Haskell} --- BSc~O.~Maroseev,~2016}
    \cvlistitem {\textit{Generation of type class instances based on instances of superclasses via GHC API} --- BSc~O.~Filippskaya,~2016}
    \cvlistitem {\textit{Functional parser for Markdown using monad combination and monoidal representation of input} --- BSc~G.~Lukianov,~2015}
    \cvlistitem {\textit{Deduction system for linear logic in Haskell} --- BSc V. Pankov, 2015}

\subsection{Summer schools and other extra trainings}
\cventry{2018}{Programming Languages Mentoring Workshop @ ICFP}%
    {}{St. Louis, USA, September 23rd 2018}%
    {}%
    {\myhref{https://icfp18.sigplan.org/track/PLMW-ICFP-2018}{Link to the program}.}

\cventry{2017}{Oregon Programming Languages Summer School}%
    {Univeristy of Oregon}{Eugene, USA, June 26th to July 8th 2017}%
    {}%
    {\myhref{https://www.cs.uoregon.edu/research/summerschool/summer17/}{Link to Official Web Page}.}

\cventry{2015}{Summer School on Generic and Effectful Programming}%
    {Department of Computer Science, Univeristy of Oxford}{St Anne's College, Oxford, 6th to 10th July 2015}%
    {}%
    {\myhref{http://staff.mmcs.sfedu.ru/~ulysses/Papers/Trainings/2015-oxford.pdf}{Link to Certificate of Attendance}. \myhref{https://www.cs.ox.ac.uk/projects/utgp/school/}{Link to Official Web Page}.}

\cventry{2011}{Summer School “Algebra and Geometry”}%
    {Laboratory of Algebraic Geometry in the National Research University Higher School of Economics, Teachers' Training University of Yaroslavl'}{Yaroslavl', Russia}%
    {}%
    {\myhref{http://mmcs.sfedu.ru/~ulysses/Papers/Trainings/2011-Yaroslavl.jpg}{Link to Certificate of Attendance} (in Russian). \myhref{http://bogomolov-lab.ru/SHKOLA/}{Link to Official Web Page} (in Russian).}

\cventry{2010}%
    {Microsoft Algorithms and Data Structures Summer School}%
    {Microsoft Research in Silicon Valey}{Saint-Petersburg, Russia}{}%
    {\myhref{http://mmcs.sfedu.ru/~ulysses/Papers/Trainings/2010-MIDAS-participation.jpg}{Link to Certificate of Attendance}. \myhref{http://logic.pdmi.ras.ru/midas/en/about}{Link to Official Web Page}.}

\cventry{2010}%
    {Winter School on Applied Mathematics and Computer Science}%
    {National Research University Higher School of Economics}%
    {Moscow province, Russia}{}%
    {\myhref{http://mmcs.sfedu.ru/~ulysses/Papers/Trainings/2010-HSE-math-school-participation.jpg}{Link to Certificate of Attendance} (in Russian).}

\cventry{2009}%
    {Marktoberdorf Summer School “Logics and Languages for Reliability and Security”}%
    {}{Marktoberdorf, Germany}{}%
    {\myhref{http://mmcs.sfedu.ru/~ulysses/Papers/Trainings/2009-Marktoberdorf-adoption.jpg}{Letter of Acceptance}. \myhref{http://asimod.in.tum.de/2009/index.shtml}{Link to Official Web Page}.}

%\cventry{year--year}{Job title}{Employer}{City}{}{Description line 1\newline{}Description line 2}

\subsection{Participation in MOOC}

{\footnotesize
\cventry{Coursera, 2013}
	{The Hardware/Software Interface}
	{Prof. J.D.~Noe}{}{}
	{\myhref{http://mmcs.sfedu.ru/~ulysses/Edu/coursera/hwsw.pdf}
	{Link to Certificate}}

\cventry{Coursera, 2012}
	{Quantum Mechanics andQuantum Computation}
	{Prof. U.~Vazirani}{}{}
	{\myhref{http://mmcs.sfedu.ru/~ulysses/Edu/coursera/qmqc.pdf}
	{Link to Certificate}}

\cventry{Coursera, 2012}
	{Functional Programming Principles
in Scala}
	{Prof. M.~Odersky}{}{}
	{\myhref{http://mmcs.sfedu.ru/~ulysses/Edu/coursera/progfun.pdf}
	{Link to Certificate}}

\cventry{Coursera, 2012}
	{Introduction to Logic}
	{Assoc.~Prof. M.~Genesereth}{}{}
	{\myhref{http://mmcs.sfedu.ru/~ulysses/Edu/coursera/intrologic.pdf}
	{Link to Certificate}}

\cventry{Coursera, 2012}
	{Compilers}
	{Prof. A.~Aiken}
	{}{}{\myhref{http://mmcs.sfedu.ru/~ulysses/Edu/coursera/compilers.pdf}
	{Link to Certificate}}

\cventry{Coursera, 2012}
	{Automata}
	{Prof. J.~Ullman}{}{}
	{\myhref{http://mmcs.sfedu.ru/~ulysses/Edu/coursera/automata.pdf}
	{Link to Certificate}}

\cventry{Coursera, 2012}
	{Cryptography I}
	{Prof. D.~Boneh}{}{}
	{\myhref{http://mmcs.sfedu.ru/~ulysses/Edu/coursera/crypto-I.pdf}
	{Link to Certificate}}

\cventry{Coursera, 2012}
	{Algorithms I}
	{Assoc.~Prof. T.~Roughgarden}{}{}
	{\myhref{http://mmcs.sfedu.ru/~ulysses/Edu/coursera/algo-I.pdf}
	{Link to Certificate}}
}

\subsection{GHC Contributions}

\cvitem{%
\myhref{https://gitlab.haskell.org/ghc/ghc/merge_requests/538/}{!538}
}{%
(Ongoing) Make \texttt{-threaded} the default
}

\cvitem{%
\myhref{https://gitlab.haskell.org/ghc/ghc/commit/bb3fa2d18686d0c08b57c66a90a9ea1b4e4482ee}{bb3fa2d1}
}{%
Remove some dependencies of the typechecker from the simplifier.
}

\cvitem{%
\myhref{https://gitlab.haskell.org/ghc/ghc/commit/c6f4eb4f8bc5e00024c74198ab9126bf1750db40}{c6f4eb4f}
}{%
Fix precision of asinh/acosh/atanh by making them primops
}

\cvitem{%
\myhref{https://gitlab.haskell.org/ghc/ghc/commit/8546afc502306de16b62c6386fe419753393cb12}{8546afc5}
}{%
Documentation: ``state transformer'' $\to$ ``state monad'' / ``ST'' (whichever is meant)
}

\cvitem{%
\myhref{https://gitlab.haskell.org/ghc/ghc/commit/14d88380ecb909e7032598aaad4efebb72561784}{14d88380}
}{%
Update Unicode tables to v. 12 of the standard
}

\subsection{Pet Projects}

\cvitem{\myhref{https://github.com/PRL-PRG/subtype-fuzzer}{subtype-fuzzer}}{A fuzzer to test a tricky subtype relation as found in the Julia programming language / Haskel, 2018}

\cvitem{\myhref{https://github.com/ulysses4ever/check-test}{chek-test}}{Remove groove from checking students' submissions / Haskell, 2016}

\cvitem{\myhref{http://code.google.com/p/cpp-mv-poly/}{cpp-mv-poly}}{\CPP-implementation of multivariate polynomials and the BMS-algrithm massively using \CPP{} templates}

\cvitem{\myhref{http://code.google.com/p/mmcs-entrance-2010/}{mmcs-entrance}}{Generation of entrance diagrams (in PNG) in MMCS/SFedU from oficial data (XLS) / Java, 2010}

\cvitem{\myhref{http://code.google.com/p/lj-comments-notifier/}{lj-comments-notifier}}{Notifications about new comments in some livejournal.com-based blog / Haskell, 2011}

\cvitem{\myhref{http://projecteuler.net/profile/ulysses4ever.png}{Project Euler}}{Link to the participant record / Haskell (mostly), \CPP}

\cvitem{\myhref{https://github.com/ulysses4ever}{Me @ GitHub}}{}

\section{Computer skills}
\cvitem{Programming languages}{\textbf{Haskell}, C, \CPP(14), Java, Scala, Julia, Pascal, \CSH}%{category 4}{XXX, YYY, ZZZ}
\cvitem{Markup, Scripting}{\textbf{\LaTeX}, Markdown, {\color{gray}HTML, CSS, JavaScript, PHP}, bash, Regular expressions}%{category 5}{XXX, YYY, ZZZ}
\cvitem{Environment}{Git, Make, Cabal, Stack, Emacs}
\cvitem{Operating systems}{\textbf{GNU/Linux family}, Windows family}%{category 6}{XXX, YYY, ZZZ}

%\cvitem{\myhref{}{}}{}

\section{Languages}
\cvitemwithcomment{Russian}{Native}{}
\cvitemwithcomment{English}{Advanced (IELTS exam band score 7.5 taken in 2012)}{}
%\cvitemwithcomment{Language 3}{Skill level}{Comment}

\section{Interests}
\cvitem{Classical literature}{Homer, Goethe, Joyce, Kafka, Camus, Sartr, Brodsky}
\cvitem{Art cinema}{Bergman, Fellini, Truffaut, Tarkovsky, Wenders, Kitano, von Trier}
%\cvitem{hobby 3}{Description}

\section{Extra info}
\cvitem{Gender}{Male}
\cvitem{Pronouns}{His/him}
\cvitem{Marital status}{Single}
\cvitem{Current place of living}{Boston, USA}
\cvitem{Citizenship, Homeland}{Russia}
%\cvitem{Name spelling in Russian}{Артём Михайлович Пеленицын}

%\section{Extra 1}
%\cvlistitem{Item 1}
%\cvlistitem{Item 2}
%\cvlistitem{Item 3}

%\renewcommand{\listitemsymbol}{-~}            % change the symbol for lists

%\section{Extra 2}
%\cvlistdoubleitem{Item 1}{Item 4}
%\cvlistdoubleitem{Item 2}{Item 5\cite{book1}}
%\cvlistdoubleitem{Item 3}{}

% Publications from a BibTeX file without multibib\renewcommand*{\bibliographyitemlabel}{\@biblabel{\arabic{enumiv}}}% for BibTeX numerical labels
%\nocite{*}
%\bibliographystyle{plain}
%\bibliography{publications}                   % 'publications' is the name of a BibTeX file

% Publications from a BibTeX file using the multibib package
%\section{Publications}
%\nocitebook{book1,book2}
%\bibliographystylebook{plain}
%\bibliographybook{publications}              % 'publications' is the name of a BibTeX file
%\nocitemisc{misc1,misc2,misc3}
%\bibliographystylemisc{plain}
%\bibliographymisc{publications}              % 'publications' is the name of a BibTeX file
\end{document}

