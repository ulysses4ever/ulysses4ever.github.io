\begin{center}
{\bfseries\Large \textsc{\Llabel{} \Lnum}}
\end{center}

\textbf{1. Другое определение нормальной формы Хомского.} Заметим, что
грамматика в НФХ не может порождать $\varepsilon$, однако мы знаем КС-языки,
содержащие $\varepsilon$. В действительности имеет место
\begin{Thm}
Пусть $G$ — КС-грамматика, а $G'$ — грамматика в НФХ, полученная из $G$
применением рассмотренного ранее алгоритма. Тогда
$$
    L(G') = L(G) \setminus \{\varepsilon\}.
$$
\end{Thm}

Можно сформулировать другое определение НФХ, которое с практической точки зрения
не хуже нашего, но позволяет выводить $\varepsilon$.
\begin{Def}[другое определение нормальной формы Хомского]
Говорят, что КС\nbdash грамматика $G$ находится в \emph{нормальной форме
Хомского}, если она не содержит бесполезных символов и каждая продукция грамматики имеет 
один из видов:
\begin{enumerate}
  \item $A \to a$,
  \item $A \to BC$,
  \item $S \to \varepsilon$,
\end{enumerate}
где $a \in \Sigma$, $A, B, C, S \in N$, $S$ — стартовый символ, \emph{не
встречающийся в правых частях продукций грамматики}.
\end{Def}

Чтобы получить НФХ в смысле последнего определения достаточно добавить в
алгоритм удаления $\varepsilon$\nbdash{}правил шаг 4:
\begin{quote}
Если $S \in \Gen_G(\varepsilon)$, то ввести в грамматику новый стартовый символ
$S'$ и две продукции $S' \to S \mid \varepsilon$.
\end{quote}
\HW{} Скорректировать решения заданий по получению НФХ так, чтобы ответом
служила грамматика в НФХ в смысле второго определения.

\textbf{2. Алгоритмические проблемы контекстно-свободных языков.} Тремя
основными проблемами теории формальных языков являются:
\begin{enumerate}
  \item \textit{проблема пустоты}: для данной грамматики $G$ определить $$L(G)
  \stackrel{?}{=} \es;$$
  \item \textit{проблема принадлежности}: для данных грамматики $G$ и слова $w
  \in \Sigma^*$ определить
  $$w \stackrel{?}{\in} L(G);$$
  \item \textit{проблема эквивалентности}: для данных грамматик $G_1, G_2$
  определить
  $$
    L(G_1) \stackrel{?}{=} L(G_2).
  $$
\end{enumerate}

Проверка пустоты КС\nbdash{}языка сводится к построению $\Gen_G(\Sigma)$ и
проверке $S \in \Gen_G(\Sigma)$. Рассмотрим один алгоритм, решающий проблему
принадлежности для КС-языков.

\begin{Algo}[Кок—Янгер—Касами, «CYK\nbdash{}алгоритм»]
\nspace\\
\textsc{Вход}: грамматика $G=(\Sigma, N, \mathcal P, S \in N)$ в НФХ,
слово $w \in \Sigma^*$.\\
\textsc{Выход}: да, $w \in L(G)$ / нет, $w \not \in L(G)$.\\
\textsc{Метод}: последовательное определение нетерминалов, выводящих
всевозможные подстроки $w$ всё большей длины.

Пусть $w = w_1 \ldots w_n$. Для всех $1 \leqslant i \leqslant j \leqslant n$
определим множество 
$$
    N_{ij} = \{ A \in N \mid A \Rightarrow^*_G w_i \ldots w_j \}.
$$
Очевидно, что $w \in L(G) \Leftrightarrow S \in N_{1n}$. Приведём алгоритм
построения множеств $N_{ij}$.
\begin{codebox}
\zi\For $i \gets 1$ \To $n$
\zi     \Do
        $N_{ii} \gets \{ A \in N \mid A \to w_i \in \mathcal P  \}$ 
        \Comment Подстроки $w$ длины $1$
        \End
\zi\For $s \gets 2$ \To $n$ \Comment Цикл по длине подстроки
\zi     \Do
        \For $i \gets 1$ \To $n - s + 1$ \Comment Цикл по месту начала подстроки
\zi         $j \gets i + s - 1$ 
            \Comment Позиция конца подстроки с началом в $w_i$ длины $s$
\zi         $N_{ij} \gets \{ A \in N \mid A \to BC \in \mathcal P; \;
                \exists k \in [i, j-1]_{\mathbb Z} \colon B \in N_{ik}, \; 
                C \in N_{k+1 j} \} $
        \End
\end{codebox}
\end{Algo}
\begin{Remark}
Алгоритм удобно выполнять, заполняя таблицу с $N_{ij}$ в ячейках.
\end{Remark}
Используя CYK\nbdash{}алгоритм,
\begin{enumerate}
    \item 
    для грамматики $G$ с продукциями:\\
    \begin{tabular}{ccc} 
        $S \to AB$, &
        $A \to BB \mid a$,&
        $B \to AB \mid b$
    \end{tabular}\\
    определить, принадлежат ли $L(G)$ строки: (а) $aabbb$, (б) $babab$,
    (в) $b^7$;
    
    \item
    для грамматики $G$ с продукциями:\\
    \begin{tabular}{cccc} 
        $S \to AB \mid BC$,&
        $A \to BA \mid a$,&
        $B \to CC \mid b$,&
        $C \to AB \mid a$
    \end{tabular}\\
    определить, принадлежат ли $L(G)$ строки: (а) $ababa$, (б) $baaab$, (в) $aabab$.
\end{enumerate}

\begin{NumRemark}[о применении CYK-алгоритма к решению задачи синтаксического
анализа] Несложная модификация CYK\nbdash{}алгоритма позволяет в случае $w \in
L(G)$ давать на выходе вывод $w$ в $G$. С точки зрения теории синтаксического анализа
CYK\nbdash{}алгоритм проводит \emph{восходящий (bottom—up) анализ}.
\end{NumRemark}

\begin{NumRemark}[о сложности CYK-алгоритма]
Нетрудно видеть, что сложность CYK\nbdash{}алгоритма может быть оценена как
$O(n^3 \cdot |\mathcal P|)$, что ограничивает применение алгоритма на практике.
Чаще всего в приложениях рассматривается подкласс КС\nbdash{}грамматик,
\emph{детерминированные КС\nbdash{}грамматики} (по-другому, $LL(k)$- и
$LR(k)$\nbdash{}грамматики), для которых существуют линейные алгоритмы разбора
(сложность $O(n)$).
\end{NumRemark} 

\begin{Proposition}
Проблема эквивалентности КС\nbdash{}грамматик является неразрешимой.
\end{Proposition}