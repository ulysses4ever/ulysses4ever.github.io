\newcommand{\Sig}{\ensuremath{\Sigma}}

\begin{center}
{\bfseries\Large \textsc{Листок \Lnum :}  грамматики,
регулярные выражения}
\end{center}

\textbf{1. Порождающие грамматики, иерархия Хомского.}\\
Для каждой грамматики, встречающейся в заданиях, следует указать её тип (в
иерархии Хомского). Написать грамматику, порождающую:
\begin{enumerate}
 \item язык $\Sig^{\ast}$, где (a) $\Sig = \{0, 1\}$; (b) \Sig{}~— произвольный
 (конечный) алфавит;
 
 \item произвольный конечный язык $L = \{\omega_i\}^n_{i=1}$;
 
 \item $\{a^{+}b^{+}\}, \{a^nb^n \mid n \in \N\}, \{a^nb^na^m \mid m, n \in
     \N\}$; $\{ a^nb^nc^n \mid n \in \N \}$ — ${\otimes}$\footnote{ Задания,
     отмеченные ${\otimes}$,~— для самостоятельного выполнения.};
 
 \item множество правильных скобочных последовательностей («язык Дика») с
     одним типом скобок;\\
     множество правильных скобочных последовательностей («язык Дика») с
     двумя типами скобок~— ${\otimes}$;
  \item арифметическую прогрессию $\{a + nd \mid n \in \NO\}$, $d > 0$, $0
    \leqslant a < d$ (имея в виду изоморфизм моноидов $(\NO, +) \cong
    (\{|\}^{\ast}, {\cdot})$, где ${\cdot}$ означает операцию конкатенации);
    язык, являющийся объединением конечного числа арифметических прогрессий.
\end{enumerate}
     
\textbf{2. Регулярные выражения.}\\
Написать регулярное выражение для
\begin{enumerate}
    \item языка над $\{a, b, c\}$ из всех слов, содержащих хотя бы один символ
    $a$;
    \item языка над $\{a, b, c\}$ из всех слов, содержащих хотя бы один символ
    $a$ и хотя бы один символ $b$;
    \item языка над $\{0, 1\}$ из всех слов, в которых третий с правого края
    символ равен $1$;
    \item языка над $\{0, 1\}$ из всех слов, в которых нет двух подряд идущих единиц;
    \item языка над $\{0, 1\}$ из всех слов, в которых любая пара смежных нулей,  
     расположена левее любой пары смежных единиц;
    \item языка над $\{0, 1\}$ из всех слов с чередующимися нулями и единицами. 
\end{enumerate}
 