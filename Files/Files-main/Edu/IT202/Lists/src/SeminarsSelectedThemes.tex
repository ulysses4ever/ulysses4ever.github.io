% файл должен быть включён в index.tex:
% % файл должен быть включён в index.tex:
% % файл должен быть включён в index.tex:
% % файл должен быть включён в index.tex:
% \input{SeminarsSelectedThemes}

\lfoot{\scriptsize ЮФУ, Мехмат, ИТ}
\rfoot{}
\lhead{\footnotesize Теория автоматов и формальных языков}
\rhead{\footnotesize\emph{весна 2009/2010}}


\begin{center}
{\bfseries\Large Некоторые вопросы, рассмотренные на практических занятиях}
\end{center}
%\begin{center}
%{\large Подробное описание}
%\end{center}
\section{Лемма о накачке}
\begin{Thm}[«Лемма о накачке» / «Лемма о разрастании»] Пусть $L$ — регулярный язык.
Тогда существует такая константа $n\in \mathbb N$, что для любого слова $w \in L$,
такого что $|w|\geqslant n$, существует такое разбиение $w=xyz$ слова $w$, что:
\begin{enumerate}
  \item\label{ynotempty} $y \neq \varepsilon$;
  \item\label{shortprefix} $|xy| \leqslant n$;
  \item\label{pumping} $\{ xy^kz \mid k \geqslant 0\} \subset L$.
\end{enumerate} 
\end{Thm}
\begin{proof}
Так как $L$ регулярный язык, то существует детерминированный конечный автомат
$\mathcal A=(Q, \Sigma, \delta, q_0, F)$ с $|Q|=N$ состояниями, распознающий
язык $L$. Пусть $w \in L$ и $|w|=N+1$. Подадим на вход автомату $\mathcal A$
слово $w$. Очевидно, существует состояние $q \in Q$, в котором автомат окажется 
дважды, читая это слово (принцип Дирихле / принцип голубятни). Разобьём слово
$w$ на три части $w=xyz$, так что: $$
    (q_0, xyz) \vdash^*  (q, yz) \stackrel{\bigtriangleup}{\vdash^*} (q, z)
    \vdash^* (q_F, \varepsilon), 
$$
где $q_F \in F$. Покажем, что для любого целого $k \geqslant 0$ автомат
распознает слово $xy^kz$. Действительно, последовательность переходов при чтении
цепочек $x$ и $z$ остаётся такой же, как для слова $w$. Часть $y^k$ читается
$k$\nbdash{}кратным повтором последовательности переходов $\bigtriangleup$. Таким
образом, $\{ xy^kz \mid k \geqslant 0\} \subset L$ и выполнено
условие~\eqref{pumping}.

Части $x, y$ слова $w$ удовлетворяют
условиям~\eqref{ynotempty}–\eqref{shortprefix} по построению. Полагая
$n=N+1$, получаем выполненными все условия теоремы.
\end{proof}

\section{Удаление бесполезных символов}
\begin{Def}Пусть дана КС-грамматика $G=(\Sigma, N, \mathcal P, S \in N)$.
Говорят, что символ $X \in \Sigma \cup N$ 
\begin{enumerate}
    \item \emph{полезный в $G$}, если
    $$
        \exists \alpha, \beta \in (\Sigma \cup N)^{\ast} \:
        \exists w \in \Sigma^* \colon \quad
        S \Rightarrow^{\ast}_G \alpha X \beta \Rightarrow^{\ast}_G w. 
    $$
    \item \emph{порождающий в $G$}, если
    $$
        \exists w \in \Sigma^* \colon \quad X \Rightarrow^{\ast}_G w 
    $$
    \item  \emph{достижимый в $G$}, если
    $$
        \exists \alpha, \beta \in (\Sigma \cup N)^{\ast} \colon \quad 
        S \Rightarrow^{\ast}_G \alpha X \beta
    $$
\end{enumerate}
\emph{Бесполезным} называется любой символ, не являющийся полезным.
\end{Def}

\begin{Thm} 
Если к КС\nbdash{}грамматике $G$, такой что $L(G) \neq \es$,
\emph{последовательно} применить два преобразования:
\begin{enumerate}
  \item удалить символы, не являющиеся порождающими,
  \item удалить символы, не являющиеся достижимыми,
\end{enumerate}
то будет получена грамматика, не содержащая бесполезных символов.
\end{Thm}
\begin{proof}
Обозначим грамматику, получившуюся после первого шага, $G_1 = (N_1, \Sigma_1,
\mathcal P_1, S)$ и грамматику, получившуюся после второго шага, $G_2 = (N_2,
\Sigma_2, \mathcal P_2, S)$ (символ $S$ не будет удaлён из грамматики после
первого шага благодаря требованию $L(G) \neq \es$ и после второго шага~— по
определению достижимого символа). Пусть $X \in N_2 \cup \Sigma_2$. Покажем, что
$X$~— полезный в $G_2$ символ. Для этого необходимо найти $\alpha, \beta \in
(N_2 \cup \Sigma_2)^*$ и $w' \in \Sigma_2^*$, такие что
\begin{equation}\label{isxusefull}
    S \Rightarrow^{*}_{G_2} \alpha X \beta \Rightarrow^{*}_{G_2} w'.
\end{equation}
Так как $X$ не был удалён после первого шага, то он является порождающим в
исходной грамматике $G$, то есть существует вывод $X \Rightarrow^*_G w$ для
некоторого $w \in \Sigma^*$. Очевидно, что каждый символ этого вывода рано или
поздно переходит в один из символов $w$ или в $\varepsilon$, то есть также
является порождающим в $G$, а значит, попадает в $G_1$. Потому этот вывод можно
целиком перенести в $G_1$:
\begin{equation}\label{xgen}
    X \Rightarrow^*_{G_1} w,
\end{equation}
где $w \in  \Sigma_1^*$.

Так как $X$ не был удалён после второго шага, то он достижим в грамматике
$G_1$, то есть существует вывод $S \Rightarrow^{*}_{G_1} \alpha X \beta$
для некоторых $\alpha, \beta \in (N_1 \cup \Sigma_1)^*$. Очевидно, что каждый
символ в этом выводе получен из $S$ применением правил грамматики $G_1$, то
есть является достижимым в $G_1$, а значит, попадает в $G_2$. Потому этот вывод
можно целиком перенести в $G_2$:
\begin{equation}\label{xreach}
    S \Rightarrow^{*}_{G_2} \alpha X \beta,
\end{equation}
где $\alpha, \beta \in (N_2 \cup \Sigma_2)^*$.

Так как $X$ достижим в $G_1$, то и каждый символ вывода~\eqref{xgen} достижим в
$G_1$, а значит, этот вывод целиком переносится в $G_2$:
\begin{equation}\label{xreachgsec}
    X \Rightarrow^*_{G_2}w, 
\end{equation}
где $w \in  \Sigma_2^*$.

Так как $\alpha, \beta \in (N_2 \cup \Sigma_2)^*$, то к цепочкам $\alpha$ и
$\beta$ применимы те же рассуждения, что к символу $X$, то есть можно получить
результат, аналогичный~\eqref{xreachgsec}:
\begin{align}
\exists u \in \Sigma_2^*\colon \alpha \Rightarrow^{*}_{G_2} u,\\
\exists v \in \Sigma_2^*\colon \beta \Rightarrow^{*}_{G_2} v.
\label{alphabetagen}
\end{align}
Объединяя результаты~\eqref{xreach}–\eqref{alphabetagen}, получаем:
$$
    S \Rightarrow^{*}_{G_2} \alpha X \beta \Rightarrow^{*}_{G_2} uwv,
$$
то есть выполнено~\eqref{isxusefull}.
\end{proof}

\section{Нахождение языка конечного автомата методом исключения состояний}
Метод исключения состояний подразумевает последовательное удаление вершин графа
переходов автомата, которое протоколируется с помощью записи на оставшихся дугах
регулярных выражений (можно считать, что в изначальном графе на дугах
простейшие регулярные выражения — однобуквенные или пустые множества для
отсутствующих дуг).

\textbf{Процедура исключения состояния $s$:} для каждых двух (необязательно
различных, но несовпадающих с $s$) состояний $p$ и $q$, таких что существует
фрагмент графа переходов автомата $p \xrightarrow{R_1} s \xrightarrow{R_2} q$,
где $R_1$, $R_2$ — некоторые регулярные выражения (метки дуг переходов),
прибавить к метке дуги $p \xrightarrow{R_3} q$ выражение $R_1R^{\ast}R_2$, где
$R$ это метка петли на вершине $s$ (если петля на $s$ и/или дуга $p \to q$
отсутствовали в исходном графе, то можно считать, что их метки равны
$\varnothing$) — таким образом получена дуга с меткой: $p \xrightarrow{R_3 +
R_1R^{\ast}R_2} q$. Удалить все просмотренные дуги $p \to s$ и $s \to q$,
инцидентные вершине $s$. После этого $s$ стала изолированной или
(неориентированно) висячей и её можно удалить (с входящими или выходящими из неё
дугами, если таковые имеются).

\begin{Algo}[нахождение языка автомата методом исключения состояний]
\nspace\\
\textsc{Вход}: Конечный автомат $\mathcal A = (Q, \Sigma, \delta, q_0, F)$,
заданный графом переходов.\\
\textsc{Выход}: Регулярное выражение, описывающее язык  автомата.
\begin{enumerate}
\renewcommand{\labelenumi}{Шаг \theenumi{}.}
  \item\label{Final} Для каждого финального состояния $q\in F$, отличного от
  начального $q_0$, применять процедуру исключения состояний до тех пор, пока не останутся
  две вершины: $q_0$ и $q$. В результате получится подобный автомат: 
  \begin{center}
  \begin{tikzpicture}[node distance=3cm,auto,%
                        every state/.style={thick},>=stealth']
    \node[state,initial,initial text=] (q_0) {$q_0$};
    \node[state,accepting] (q) [right of=q_0] {$q$};

    \path[->] 
        (q_0) edge [loop above] node 
            {$R$} ()
        (q_0) edge[bend left=40] node 
            {$S$} (q)
        (q) edge [loop above] node 
            {$U$} 
            ()
        (q) edge[bend left=40] node 
            {$T$} (q_0);
    \end{tikzpicture}
    \end{center}
    Допускаемый им язык описывается так:
    $$(R+SU^{\ast}T)^{\ast}SU^{\ast}.$$
    
    \item\label{InitIsFinal} Если начальное состояние $q_0$ является финальным
    ($q_0\in F$), применять процедуру исключения состояний, пока не останется
    единственная вершина $q_0$. В результате получится подобный автомат:
    \begin{center}
    \begin{tikzpicture}[node distance=3cm,auto,%
                        every state/.style={thick},>=stealth'] 
        \node[state,initial,initial text=,accepting] (q_0) {$q_0$};
        
        \path[->] 
            (q_0) edge [loop above] node 
                {$R$} ();
    \end{tikzpicture}
    \end{center}
    Допускаемый им язык описывается так: $R^{\ast}$.

    \item Язык исходного автомата определяется как сумма всех регулярных выражений,
    полученных на шагах \eqref{Final}–\eqref{InitIsFinal}.
    \end{enumerate}
\end{Algo}

\section{Свойства замкнутости класса регулярных языков}
\begin{Thm}Класс регулярных языков замкнут относительно операций: объединения,
конкатенации, итерации, дополнения, пересечения, разности.
\end{Thm}
\begin{proof}Для регулярных языков $L_1$, $L_2$ рассмотрим описывающие
их регулярные выражения $\RE(L_1)$, $\RE(L_2)$. Язык объединения
(соответственно, конкатенации, итерации) описывается выражением $\RE(L_1) +
\RE(L_2)$ (соответственно, $\RE(L_1)\RE(L_2)$, $\RE(L_1)^\ast$), а значит,
регулярен.

Для регулярного языка $L$ построим распознающий его детерминированный конечный
автомат $\mathcal A = (Q, \Sigma, \delta, q_0, F)$ ($L=L(\mathcal A)$). Легко
видеть, что автомат $\widetilde{\mathcal A} = (Q, \Sigma, \delta, q_0, Q
\setminus F)$ допускает дополнение $\overline L = \Sigma^\ast \setminus L$ языка
$L$, которое является, таким образом, регулярным языком.

Поскольку справедливо $L_1 \cap L_2 = \overline{\overline L_1 \cup \overline
L_2}$, язык $L_1 \cap L_2$ регулярен по доказанному выше. Аналогичное можно 
заключить из равенства $L_1 \setminus L_2 = L_1 \cap \overline{L_2}$.
\end{proof}

% \section{Алгоритм Кока—Янгера—Касами решения проблемы принадлежности для
% КС-языков}%\nbdash{}
% \begin{Algo}[Кок—Янгер—Касами, «CYK\nbdash{}алгоритм»]
% \nspace\\
% \textsc{Вход}: грамматика $G=(\Sigma, N, \mathcal P, S \in N)$ в НФХ,
% слово $w \in \Sigma^*$.\\
% \textsc{Выход}: да, $w \in L(G)$ / нет, $w \not \in L(G)$.\\
% \textsc{Метод}: последовательное определение нетерминалов, выводящих
% всевозможные подстроки $w$ всё большей длины.
% 
% Пусть $w = w_1 \ldots w_n$. Для всех $1 \leqslant i \leqslant j \leqslant n$
% определим множество 
% $$
%     N_{ij} = \{ A \in N \mid A \Rightarrow^*_G w_i \ldots w_j \}.
% $$
% Очевидно, что $w \in L(G) \Leftrightarrow S \in N_{1n}$. Приведём алгоритм
% построения множеств $N_{ij}$.
% \begin{codebox}
% \zi\For $i \gets 1$ \To $n$
% \zi     \Do
%         $N_{ii} \gets \{ A \in N \mid A \to w_i \in \mathcal P  \}$ 
%         \Comment Подстроки $w$ длины $1$
%         \End
% \zi\For $s \gets 2$ \To $n$ \Comment Цикл по длине подстроки
% \zi     \Do
%         \For $i \gets 1$ \To $n - s + 1$ \Comment Цикл по месту начала подстроки
% \zi         $j \gets i + s - 1$ 
%             \Comment Позиция конца подстроки с началом в $w_i$ длины $s$
% \zi         $N_{ij} \gets \{ A \in N \mid A \to BC \in \mathcal P; \;
%                 \exists k \in [i, j-1]_{\mathbb Z} \colon B \in N_{ik}, \; 
%                 C \in N_{k+1 j} \} $
% \end{codebox}
% \end{Algo}
% \begin{NumRemark}
% Алгоритм удобно выполнять, заполняя таблицу с $N_{ij}$ в ячейках. Ячейки таблицы
% расположены в системе координат $(i,s)$, в позиции $(i,s)$ находится множество
% $N_{i, i+s-1}$.
% \end{NumRemark}
% 
% \begin{NumRemark}[о сложности CYK-алгоритма]
% Нетрудно видеть, что сложность CYK\nbdash{}алгоритма может быть оценена как
% $O(n^3 \cdot |\mathcal P|)$.
% \end{NumRemark} 


\lfoot{\scriptsize ЮФУ, Мехмат, ИТ}
\rfoot{}
\lhead{\footnotesize Теория автоматов и формальных языков}
\rhead{\footnotesize\emph{весна 2009/2010}}


\begin{center}
{\bfseries\Large Некоторые вопросы, рассмотренные на практических занятиях}
\end{center}
%\begin{center}
%{\large Подробное описание}
%\end{center}
\section{Лемма о накачке}
\begin{Thm}[«Лемма о накачке» / «Лемма о разрастании»] Пусть $L$ — регулярный язык.
Тогда существует такая константа $n\in \mathbb N$, что для любого слова $w \in L$,
такого что $|w|\geqslant n$, существует такое разбиение $w=xyz$ слова $w$, что:
\begin{enumerate}
  \item\label{ynotempty} $y \neq \varepsilon$;
  \item\label{shortprefix} $|xy| \leqslant n$;
  \item\label{pumping} $\{ xy^kz \mid k \geqslant 0\} \subset L$.
\end{enumerate} 
\end{Thm}
\begin{proof}
Так как $L$ регулярный язык, то существует детерминированный конечный автомат
$\mathcal A=(Q, \Sigma, \delta, q_0, F)$ с $|Q|=N$ состояниями, распознающий
язык $L$. Пусть $w \in L$ и $|w|=N+1$. Подадим на вход автомату $\mathcal A$
слово $w$. Очевидно, существует состояние $q \in Q$, в котором автомат окажется 
дважды, читая это слово (принцип Дирихле / принцип голубятни). Разобьём слово
$w$ на три части $w=xyz$, так что: $$
    (q_0, xyz) \vdash^*  (q, yz) \stackrel{\bigtriangleup}{\vdash^*} (q, z)
    \vdash^* (q_F, \varepsilon), 
$$
где $q_F \in F$. Покажем, что для любого целого $k \geqslant 0$ автомат
распознает слово $xy^kz$. Действительно, последовательность переходов при чтении
цепочек $x$ и $z$ остаётся такой же, как для слова $w$. Часть $y^k$ читается
$k$\nbdash{}кратным повтором последовательности переходов $\bigtriangleup$. Таким
образом, $\{ xy^kz \mid k \geqslant 0\} \subset L$ и выполнено
условие~\eqref{pumping}.

Части $x, y$ слова $w$ удовлетворяют
условиям~\eqref{ynotempty}–\eqref{shortprefix} по построению. Полагая
$n=N+1$, получаем выполненными все условия теоремы.
\end{proof}

\section{Удаление бесполезных символов}
\begin{Def}Пусть дана КС-грамматика $G=(\Sigma, N, \mathcal P, S \in N)$.
Говорят, что символ $X \in \Sigma \cup N$ 
\begin{enumerate}
    \item \emph{полезный в $G$}, если
    $$
        \exists \alpha, \beta \in (\Sigma \cup N)^{\ast} \:
        \exists w \in \Sigma^* \colon \quad
        S \Rightarrow^{\ast}_G \alpha X \beta \Rightarrow^{\ast}_G w. 
    $$
    \item \emph{порождающий в $G$}, если
    $$
        \exists w \in \Sigma^* \colon \quad X \Rightarrow^{\ast}_G w 
    $$
    \item  \emph{достижимый в $G$}, если
    $$
        \exists \alpha, \beta \in (\Sigma \cup N)^{\ast} \colon \quad 
        S \Rightarrow^{\ast}_G \alpha X \beta
    $$
\end{enumerate}
\emph{Бесполезным} называется любой символ, не являющийся полезным.
\end{Def}

\begin{Thm} 
Если к КС\nbdash{}грамматике $G$, такой что $L(G) \neq \es$,
\emph{последовательно} применить два преобразования:
\begin{enumerate}
  \item удалить символы, не являющиеся порождающими,
  \item удалить символы, не являющиеся достижимыми,
\end{enumerate}
то будет получена грамматика, не содержащая бесполезных символов.
\end{Thm}
\begin{proof}
Обозначим грамматику, получившуюся после первого шага, $G_1 = (N_1, \Sigma_1,
\mathcal P_1, S)$ и грамматику, получившуюся после второго шага, $G_2 = (N_2,
\Sigma_2, \mathcal P_2, S)$ (символ $S$ не будет удaлён из грамматики после
первого шага благодаря требованию $L(G) \neq \es$ и после второго шага~— по
определению достижимого символа). Пусть $X \in N_2 \cup \Sigma_2$. Покажем, что
$X$~— полезный в $G_2$ символ. Для этого необходимо найти $\alpha, \beta \in
(N_2 \cup \Sigma_2)^*$ и $w' \in \Sigma_2^*$, такие что
\begin{equation}\label{isxusefull}
    S \Rightarrow^{*}_{G_2} \alpha X \beta \Rightarrow^{*}_{G_2} w'.
\end{equation}
Так как $X$ не был удалён после первого шага, то он является порождающим в
исходной грамматике $G$, то есть существует вывод $X \Rightarrow^*_G w$ для
некоторого $w \in \Sigma^*$. Очевидно, что каждый символ этого вывода рано или
поздно переходит в один из символов $w$ или в $\varepsilon$, то есть также
является порождающим в $G$, а значит, попадает в $G_1$. Потому этот вывод можно
целиком перенести в $G_1$:
\begin{equation}\label{xgen}
    X \Rightarrow^*_{G_1} w,
\end{equation}
где $w \in  \Sigma_1^*$.

Так как $X$ не был удалён после второго шага, то он достижим в грамматике
$G_1$, то есть существует вывод $S \Rightarrow^{*}_{G_1} \alpha X \beta$
для некоторых $\alpha, \beta \in (N_1 \cup \Sigma_1)^*$. Очевидно, что каждый
символ в этом выводе получен из $S$ применением правил грамматики $G_1$, то
есть является достижимым в $G_1$, а значит, попадает в $G_2$. Потому этот вывод
можно целиком перенести в $G_2$:
\begin{equation}\label{xreach}
    S \Rightarrow^{*}_{G_2} \alpha X \beta,
\end{equation}
где $\alpha, \beta \in (N_2 \cup \Sigma_2)^*$.

Так как $X$ достижим в $G_1$, то и каждый символ вывода~\eqref{xgen} достижим в
$G_1$, а значит, этот вывод целиком переносится в $G_2$:
\begin{equation}\label{xreachgsec}
    X \Rightarrow^*_{G_2}w, 
\end{equation}
где $w \in  \Sigma_2^*$.

Так как $\alpha, \beta \in (N_2 \cup \Sigma_2)^*$, то к цепочкам $\alpha$ и
$\beta$ применимы те же рассуждения, что к символу $X$, то есть можно получить
результат, аналогичный~\eqref{xreachgsec}:
\begin{align}
\exists u \in \Sigma_2^*\colon \alpha \Rightarrow^{*}_{G_2} u,\\
\exists v \in \Sigma_2^*\colon \beta \Rightarrow^{*}_{G_2} v.
\label{alphabetagen}
\end{align}
Объединяя результаты~\eqref{xreach}–\eqref{alphabetagen}, получаем:
$$
    S \Rightarrow^{*}_{G_2} \alpha X \beta \Rightarrow^{*}_{G_2} uwv,
$$
то есть выполнено~\eqref{isxusefull}.
\end{proof}

\section{Нахождение языка конечного автомата методом исключения состояний}
Метод исключения состояний подразумевает последовательное удаление вершин графа
переходов автомата, которое протоколируется с помощью записи на оставшихся дугах
регулярных выражений (можно считать, что в изначальном графе на дугах
простейшие регулярные выражения — однобуквенные или пустые множества для
отсутствующих дуг).

\textbf{Процедура исключения состояния $s$:} для каждых двух (необязательно
различных, но несовпадающих с $s$) состояний $p$ и $q$, таких что существует
фрагмент графа переходов автомата $p \xrightarrow{R_1} s \xrightarrow{R_2} q$,
где $R_1$, $R_2$ — некоторые регулярные выражения (метки дуг переходов),
прибавить к метке дуги $p \xrightarrow{R_3} q$ выражение $R_1R^{\ast}R_2$, где
$R$ это метка петли на вершине $s$ (если петля на $s$ и/или дуга $p \to q$
отсутствовали в исходном графе, то можно считать, что их метки равны
$\varnothing$) — таким образом получена дуга с меткой: $p \xrightarrow{R_3 +
R_1R^{\ast}R_2} q$. Удалить все просмотренные дуги $p \to s$ и $s \to q$,
инцидентные вершине $s$. После этого $s$ стала изолированной или
(неориентированно) висячей и её можно удалить (с входящими или выходящими из неё
дугами, если таковые имеются).

\begin{Algo}[нахождение языка автомата методом исключения состояний]
\nspace\\
\textsc{Вход}: Конечный автомат $\mathcal A = (Q, \Sigma, \delta, q_0, F)$,
заданный графом переходов.\\
\textsc{Выход}: Регулярное выражение, описывающее язык  автомата.
\begin{enumerate}
\renewcommand{\labelenumi}{Шаг \theenumi{}.}
  \item\label{Final} Для каждого финального состояния $q\in F$, отличного от
  начального $q_0$, применять процедуру исключения состояний до тех пор, пока не останутся
  две вершины: $q_0$ и $q$. В результате получится подобный автомат: 
  \begin{center}
  \begin{tikzpicture}[node distance=3cm,auto,%
                        every state/.style={thick},>=stealth']
    \node[state,initial,initial text=] (q_0) {$q_0$};
    \node[state,accepting] (q) [right of=q_0] {$q$};

    \path[->] 
        (q_0) edge [loop above] node 
            {$R$} ()
        (q_0) edge[bend left=40] node 
            {$S$} (q)
        (q) edge [loop above] node 
            {$U$} 
            ()
        (q) edge[bend left=40] node 
            {$T$} (q_0);
    \end{tikzpicture}
    \end{center}
    Допускаемый им язык описывается так:
    $$(R+SU^{\ast}T)^{\ast}SU^{\ast}.$$
    
    \item\label{InitIsFinal} Если начальное состояние $q_0$ является финальным
    ($q_0\in F$), применять процедуру исключения состояний, пока не останется
    единственная вершина $q_0$. В результате получится подобный автомат:
    \begin{center}
    \begin{tikzpicture}[node distance=3cm,auto,%
                        every state/.style={thick},>=stealth'] 
        \node[state,initial,initial text=,accepting] (q_0) {$q_0$};
        
        \path[->] 
            (q_0) edge [loop above] node 
                {$R$} ();
    \end{tikzpicture}
    \end{center}
    Допускаемый им язык описывается так: $R^{\ast}$.

    \item Язык исходного автомата определяется как сумма всех регулярных выражений,
    полученных на шагах \eqref{Final}–\eqref{InitIsFinal}.
    \end{enumerate}
\end{Algo}

\section{Свойства замкнутости класса регулярных языков}
\begin{Thm}Класс регулярных языков замкнут относительно операций: объединения,
конкатенации, итерации, дополнения, пересечения, разности.
\end{Thm}
\begin{proof}Для регулярных языков $L_1$, $L_2$ рассмотрим описывающие
их регулярные выражения $\RE(L_1)$, $\RE(L_2)$. Язык объединения
(соответственно, конкатенации, итерации) описывается выражением $\RE(L_1) +
\RE(L_2)$ (соответственно, $\RE(L_1)\RE(L_2)$, $\RE(L_1)^\ast$), а значит,
регулярен.

Для регулярного языка $L$ построим распознающий его детерминированный конечный
автомат $\mathcal A = (Q, \Sigma, \delta, q_0, F)$ ($L=L(\mathcal A)$). Легко
видеть, что автомат $\widetilde{\mathcal A} = (Q, \Sigma, \delta, q_0, Q
\setminus F)$ допускает дополнение $\overline L = \Sigma^\ast \setminus L$ языка
$L$, которое является, таким образом, регулярным языком.

Поскольку справедливо $L_1 \cap L_2 = \overline{\overline L_1 \cup \overline
L_2}$, язык $L_1 \cap L_2$ регулярен по доказанному выше. Аналогичное можно 
заключить из равенства $L_1 \setminus L_2 = L_1 \cap \overline{L_2}$.
\end{proof}

% \section{Алгоритм Кока—Янгера—Касами решения проблемы принадлежности для
% КС-языков}%\nbdash{}
% \begin{Algo}[Кок—Янгер—Касами, «CYK\nbdash{}алгоритм»]
% \nspace\\
% \textsc{Вход}: грамматика $G=(\Sigma, N, \mathcal P, S \in N)$ в НФХ,
% слово $w \in \Sigma^*$.\\
% \textsc{Выход}: да, $w \in L(G)$ / нет, $w \not \in L(G)$.\\
% \textsc{Метод}: последовательное определение нетерминалов, выводящих
% всевозможные подстроки $w$ всё большей длины.
% 
% Пусть $w = w_1 \ldots w_n$. Для всех $1 \leqslant i \leqslant j \leqslant n$
% определим множество 
% $$
%     N_{ij} = \{ A \in N \mid A \Rightarrow^*_G w_i \ldots w_j \}.
% $$
% Очевидно, что $w \in L(G) \Leftrightarrow S \in N_{1n}$. Приведём алгоритм
% построения множеств $N_{ij}$.
% \begin{codebox}
% \zi\For $i \gets 1$ \To $n$
% \zi     \Do
%         $N_{ii} \gets \{ A \in N \mid A \to w_i \in \mathcal P  \}$ 
%         \Comment Подстроки $w$ длины $1$
%         \End
% \zi\For $s \gets 2$ \To $n$ \Comment Цикл по длине подстроки
% \zi     \Do
%         \For $i \gets 1$ \To $n - s + 1$ \Comment Цикл по месту начала подстроки
% \zi         $j \gets i + s - 1$ 
%             \Comment Позиция конца подстроки с началом в $w_i$ длины $s$
% \zi         $N_{ij} \gets \{ A \in N \mid A \to BC \in \mathcal P; \;
%                 \exists k \in [i, j-1]_{\mathbb Z} \colon B \in N_{ik}, \; 
%                 C \in N_{k+1 j} \} $
% \end{codebox}
% \end{Algo}
% \begin{NumRemark}
% Алгоритм удобно выполнять, заполняя таблицу с $N_{ij}$ в ячейках. Ячейки таблицы
% расположены в системе координат $(i,s)$, в позиции $(i,s)$ находится множество
% $N_{i, i+s-1}$.
% \end{NumRemark}
% 
% \begin{NumRemark}[о сложности CYK-алгоритма]
% Нетрудно видеть, что сложность CYK\nbdash{}алгоритма может быть оценена как
% $O(n^3 \cdot |\mathcal P|)$.
% \end{NumRemark} 


\lfoot{\scriptsize ЮФУ, Мехмат, ИТ}
\rfoot{}
\lhead{\footnotesize Теория автоматов и формальных языков}
\rhead{\footnotesize\emph{весна 2009/2010}}


\begin{center}
{\bfseries\Large Некоторые вопросы, рассмотренные на практических занятиях}
\end{center}
%\begin{center}
%{\large Подробное описание}
%\end{center}
\section{Лемма о накачке}
\begin{Thm}[«Лемма о накачке» / «Лемма о разрастании»] Пусть $L$ — регулярный язык.
Тогда существует такая константа $n\in \mathbb N$, что для любого слова $w \in L$,
такого что $|w|\geqslant n$, существует такое разбиение $w=xyz$ слова $w$, что:
\begin{enumerate}
  \item\label{ynotempty} $y \neq \varepsilon$;
  \item\label{shortprefix} $|xy| \leqslant n$;
  \item\label{pumping} $\{ xy^kz \mid k \geqslant 0\} \subset L$.
\end{enumerate} 
\end{Thm}
\begin{proof}
Так как $L$ регулярный язык, то существует детерминированный конечный автомат
$\mathcal A=(Q, \Sigma, \delta, q_0, F)$ с $|Q|=N$ состояниями, распознающий
язык $L$. Пусть $w \in L$ и $|w|=N+1$. Подадим на вход автомату $\mathcal A$
слово $w$. Очевидно, существует состояние $q \in Q$, в котором автомат окажется 
дважды, читая это слово (принцип Дирихле / принцип голубятни). Разобьём слово
$w$ на три части $w=xyz$, так что: $$
    (q_0, xyz) \vdash^*  (q, yz) \stackrel{\bigtriangleup}{\vdash^*} (q, z)
    \vdash^* (q_F, \varepsilon), 
$$
где $q_F \in F$. Покажем, что для любого целого $k \geqslant 0$ автомат
распознает слово $xy^kz$. Действительно, последовательность переходов при чтении
цепочек $x$ и $z$ остаётся такой же, как для слова $w$. Часть $y^k$ читается
$k$\nbdash{}кратным повтором последовательности переходов $\bigtriangleup$. Таким
образом, $\{ xy^kz \mid k \geqslant 0\} \subset L$ и выполнено
условие~\eqref{pumping}.

Части $x, y$ слова $w$ удовлетворяют
условиям~\eqref{ynotempty}–\eqref{shortprefix} по построению. Полагая
$n=N+1$, получаем выполненными все условия теоремы.
\end{proof}

\section{Удаление бесполезных символов}
\begin{Def}Пусть дана КС-грамматика $G=(\Sigma, N, \mathcal P, S \in N)$.
Говорят, что символ $X \in \Sigma \cup N$ 
\begin{enumerate}
    \item \emph{полезный в $G$}, если
    $$
        \exists \alpha, \beta \in (\Sigma \cup N)^{\ast} \:
        \exists w \in \Sigma^* \colon \quad
        S \Rightarrow^{\ast}_G \alpha X \beta \Rightarrow^{\ast}_G w. 
    $$
    \item \emph{порождающий в $G$}, если
    $$
        \exists w \in \Sigma^* \colon \quad X \Rightarrow^{\ast}_G w 
    $$
    \item  \emph{достижимый в $G$}, если
    $$
        \exists \alpha, \beta \in (\Sigma \cup N)^{\ast} \colon \quad 
        S \Rightarrow^{\ast}_G \alpha X \beta
    $$
\end{enumerate}
\emph{Бесполезным} называется любой символ, не являющийся полезным.
\end{Def}

\begin{Thm} 
Если к КС\nbdash{}грамматике $G$, такой что $L(G) \neq \es$,
\emph{последовательно} применить два преобразования:
\begin{enumerate}
  \item удалить символы, не являющиеся порождающими,
  \item удалить символы, не являющиеся достижимыми,
\end{enumerate}
то будет получена грамматика, не содержащая бесполезных символов.
\end{Thm}
\begin{proof}
Обозначим грамматику, получившуюся после первого шага, $G_1 = (N_1, \Sigma_1,
\mathcal P_1, S)$ и грамматику, получившуюся после второго шага, $G_2 = (N_2,
\Sigma_2, \mathcal P_2, S)$ (символ $S$ не будет удaлён из грамматики после
первого шага благодаря требованию $L(G) \neq \es$ и после второго шага~— по
определению достижимого символа). Пусть $X \in N_2 \cup \Sigma_2$. Покажем, что
$X$~— полезный в $G_2$ символ. Для этого необходимо найти $\alpha, \beta \in
(N_2 \cup \Sigma_2)^*$ и $w' \in \Sigma_2^*$, такие что
\begin{equation}\label{isxusefull}
    S \Rightarrow^{*}_{G_2} \alpha X \beta \Rightarrow^{*}_{G_2} w'.
\end{equation}
Так как $X$ не был удалён после первого шага, то он является порождающим в
исходной грамматике $G$, то есть существует вывод $X \Rightarrow^*_G w$ для
некоторого $w \in \Sigma^*$. Очевидно, что каждый символ этого вывода рано или
поздно переходит в один из символов $w$ или в $\varepsilon$, то есть также
является порождающим в $G$, а значит, попадает в $G_1$. Потому этот вывод можно
целиком перенести в $G_1$:
\begin{equation}\label{xgen}
    X \Rightarrow^*_{G_1} w,
\end{equation}
где $w \in  \Sigma_1^*$.

Так как $X$ не был удалён после второго шага, то он достижим в грамматике
$G_1$, то есть существует вывод $S \Rightarrow^{*}_{G_1} \alpha X \beta$
для некоторых $\alpha, \beta \in (N_1 \cup \Sigma_1)^*$. Очевидно, что каждый
символ в этом выводе получен из $S$ применением правил грамматики $G_1$, то
есть является достижимым в $G_1$, а значит, попадает в $G_2$. Потому этот вывод
можно целиком перенести в $G_2$:
\begin{equation}\label{xreach}
    S \Rightarrow^{*}_{G_2} \alpha X \beta,
\end{equation}
где $\alpha, \beta \in (N_2 \cup \Sigma_2)^*$.

Так как $X$ достижим в $G_1$, то и каждый символ вывода~\eqref{xgen} достижим в
$G_1$, а значит, этот вывод целиком переносится в $G_2$:
\begin{equation}\label{xreachgsec}
    X \Rightarrow^*_{G_2}w, 
\end{equation}
где $w \in  \Sigma_2^*$.

Так как $\alpha, \beta \in (N_2 \cup \Sigma_2)^*$, то к цепочкам $\alpha$ и
$\beta$ применимы те же рассуждения, что к символу $X$, то есть можно получить
результат, аналогичный~\eqref{xreachgsec}:
\begin{align}
\exists u \in \Sigma_2^*\colon \alpha \Rightarrow^{*}_{G_2} u,\\
\exists v \in \Sigma_2^*\colon \beta \Rightarrow^{*}_{G_2} v.
\label{alphabetagen}
\end{align}
Объединяя результаты~\eqref{xreach}–\eqref{alphabetagen}, получаем:
$$
    S \Rightarrow^{*}_{G_2} \alpha X \beta \Rightarrow^{*}_{G_2} uwv,
$$
то есть выполнено~\eqref{isxusefull}.
\end{proof}

\section{Нахождение языка конечного автомата методом исключения состояний}
Метод исключения состояний подразумевает последовательное удаление вершин графа
переходов автомата, которое протоколируется с помощью записи на оставшихся дугах
регулярных выражений (можно считать, что в изначальном графе на дугах
простейшие регулярные выражения — однобуквенные или пустые множества для
отсутствующих дуг).

\textbf{Процедура исключения состояния $s$:} для каждых двух (необязательно
различных, но несовпадающих с $s$) состояний $p$ и $q$, таких что существует
фрагмент графа переходов автомата $p \xrightarrow{R_1} s \xrightarrow{R_2} q$,
где $R_1$, $R_2$ — некоторые регулярные выражения (метки дуг переходов),
прибавить к метке дуги $p \xrightarrow{R_3} q$ выражение $R_1R^{\ast}R_2$, где
$R$ это метка петли на вершине $s$ (если петля на $s$ и/или дуга $p \to q$
отсутствовали в исходном графе, то можно считать, что их метки равны
$\varnothing$) — таким образом получена дуга с меткой: $p \xrightarrow{R_3 +
R_1R^{\ast}R_2} q$. Удалить все просмотренные дуги $p \to s$ и $s \to q$,
инцидентные вершине $s$. После этого $s$ стала изолированной или
(неориентированно) висячей и её можно удалить (с входящими или выходящими из неё
дугами, если таковые имеются).

\begin{Algo}[нахождение языка автомата методом исключения состояний]
\nspace\\
\textsc{Вход}: Конечный автомат $\mathcal A = (Q, \Sigma, \delta, q_0, F)$,
заданный графом переходов.\\
\textsc{Выход}: Регулярное выражение, описывающее язык  автомата.
\begin{enumerate}
\renewcommand{\labelenumi}{Шаг \theenumi{}.}
  \item\label{Final} Для каждого финального состояния $q\in F$, отличного от
  начального $q_0$, применять процедуру исключения состояний до тех пор, пока не останутся
  две вершины: $q_0$ и $q$. В результате получится подобный автомат: 
  \begin{center}
  \begin{tikzpicture}[node distance=3cm,auto,%
                        every state/.style={thick},>=stealth']
    \node[state,initial,initial text=] (q_0) {$q_0$};
    \node[state,accepting] (q) [right of=q_0] {$q$};

    \path[->] 
        (q_0) edge [loop above] node 
            {$R$} ()
        (q_0) edge[bend left=40] node 
            {$S$} (q)
        (q) edge [loop above] node 
            {$U$} 
            ()
        (q) edge[bend left=40] node 
            {$T$} (q_0);
    \end{tikzpicture}
    \end{center}
    Допускаемый им язык описывается так:
    $$(R+SU^{\ast}T)^{\ast}SU^{\ast}.$$
    
    \item\label{InitIsFinal} Если начальное состояние $q_0$ является финальным
    ($q_0\in F$), применять процедуру исключения состояний, пока не останется
    единственная вершина $q_0$. В результате получится подобный автомат:
    \begin{center}
    \begin{tikzpicture}[node distance=3cm,auto,%
                        every state/.style={thick},>=stealth'] 
        \node[state,initial,initial text=,accepting] (q_0) {$q_0$};
        
        \path[->] 
            (q_0) edge [loop above] node 
                {$R$} ();
    \end{tikzpicture}
    \end{center}
    Допускаемый им язык описывается так: $R^{\ast}$.

    \item Язык исходного автомата определяется как сумма всех регулярных выражений,
    полученных на шагах \eqref{Final}–\eqref{InitIsFinal}.
    \end{enumerate}
\end{Algo}

\section{Свойства замкнутости класса регулярных языков}
\begin{Thm}Класс регулярных языков замкнут относительно операций: объединения,
конкатенации, итерации, дополнения, пересечения, разности.
\end{Thm}
\begin{proof}Для регулярных языков $L_1$, $L_2$ рассмотрим описывающие
их регулярные выражения $\RE(L_1)$, $\RE(L_2)$. Язык объединения
(соответственно, конкатенации, итерации) описывается выражением $\RE(L_1) +
\RE(L_2)$ (соответственно, $\RE(L_1)\RE(L_2)$, $\RE(L_1)^\ast$), а значит,
регулярен.

Для регулярного языка $L$ построим распознающий его детерминированный конечный
автомат $\mathcal A = (Q, \Sigma, \delta, q_0, F)$ ($L=L(\mathcal A)$). Легко
видеть, что автомат $\widetilde{\mathcal A} = (Q, \Sigma, \delta, q_0, Q
\setminus F)$ допускает дополнение $\overline L = \Sigma^\ast \setminus L$ языка
$L$, которое является, таким образом, регулярным языком.

Поскольку справедливо $L_1 \cap L_2 = \overline{\overline L_1 \cup \overline
L_2}$, язык $L_1 \cap L_2$ регулярен по доказанному выше. Аналогичное можно 
заключить из равенства $L_1 \setminus L_2 = L_1 \cap \overline{L_2}$.
\end{proof}

% \section{Алгоритм Кока—Янгера—Касами решения проблемы принадлежности для
% КС-языков}%\nbdash{}
% \begin{Algo}[Кок—Янгер—Касами, «CYK\nbdash{}алгоритм»]
% \nspace\\
% \textsc{Вход}: грамматика $G=(\Sigma, N, \mathcal P, S \in N)$ в НФХ,
% слово $w \in \Sigma^*$.\\
% \textsc{Выход}: да, $w \in L(G)$ / нет, $w \not \in L(G)$.\\
% \textsc{Метод}: последовательное определение нетерминалов, выводящих
% всевозможные подстроки $w$ всё большей длины.
% 
% Пусть $w = w_1 \ldots w_n$. Для всех $1 \leqslant i \leqslant j \leqslant n$
% определим множество 
% $$
%     N_{ij} = \{ A \in N \mid A \Rightarrow^*_G w_i \ldots w_j \}.
% $$
% Очевидно, что $w \in L(G) \Leftrightarrow S \in N_{1n}$. Приведём алгоритм
% построения множеств $N_{ij}$.
% \begin{codebox}
% \zi\For $i \gets 1$ \To $n$
% \zi     \Do
%         $N_{ii} \gets \{ A \in N \mid A \to w_i \in \mathcal P  \}$ 
%         \Comment Подстроки $w$ длины $1$
%         \End
% \zi\For $s \gets 2$ \To $n$ \Comment Цикл по длине подстроки
% \zi     \Do
%         \For $i \gets 1$ \To $n - s + 1$ \Comment Цикл по месту начала подстроки
% \zi         $j \gets i + s - 1$ 
%             \Comment Позиция конца подстроки с началом в $w_i$ длины $s$
% \zi         $N_{ij} \gets \{ A \in N \mid A \to BC \in \mathcal P; \;
%                 \exists k \in [i, j-1]_{\mathbb Z} \colon B \in N_{ik}, \; 
%                 C \in N_{k+1 j} \} $
% \end{codebox}
% \end{Algo}
% \begin{NumRemark}
% Алгоритм удобно выполнять, заполняя таблицу с $N_{ij}$ в ячейках. Ячейки таблицы
% расположены в системе координат $(i,s)$, в позиции $(i,s)$ находится множество
% $N_{i, i+s-1}$.
% \end{NumRemark}
% 
% \begin{NumRemark}[о сложности CYK-алгоритма]
% Нетрудно видеть, что сложность CYK\nbdash{}алгоритма может быть оценена как
% $O(n^3 \cdot |\mathcal P|)$.
% \end{NumRemark} 


\lfoot{\scriptsize ЮФУ, Мехмат, ИТ}
\rfoot{}
\lhead{\footnotesize Теория автоматов и формальных языков}
\rhead{\footnotesize\emph{весна 2009/2010}}


\begin{center}
{\bfseries\Large Некоторые вопросы, рассмотренные на практических занятиях}
\end{center}
%\begin{center}
%{\large Подробное описание}
%\end{center}
\section{Лемма о накачке}
\begin{Thm}[«Лемма о накачке» / «Лемма о разрастании»] Пусть $L$ — регулярный язык.
Тогда существует такая константа $n\in \mathbb N$, что для любого слова $w \in L$,
такого что $|w|\geqslant n$, существует такое разбиение $w=xyz$ слова $w$, что:
\begin{enumerate}
  \item\label{ynotempty} $y \neq \varepsilon$;
  \item\label{shortprefix} $|xy| \leqslant n$;
  \item\label{pumping} $\{ xy^kz \mid k \geqslant 0\} \subset L$.
\end{enumerate} 
\end{Thm}
\begin{proof}
Так как $L$ регулярный язык, то существует детерминированный конечный автомат
$\mathcal A=(Q, \Sigma, \delta, q_0, F)$ с $|Q|=N$ состояниями, распознающий
язык $L$. Пусть $w \in L$ и $|w|=N+1$. Подадим на вход автомату $\mathcal A$
слово $w$. Очевидно, существует состояние $q \in Q$, в котором автомат окажется 
дважды, читая это слово (принцип Дирихле / принцип голубятни). Разобьём слово
$w$ на три части $w=xyz$, так что: $$
    (q_0, xyz) \vdash^*  (q, yz) \stackrel{\bigtriangleup}{\vdash^*} (q, z)
    \vdash^* (q_F, \varepsilon), 
$$
где $q_F \in F$. Покажем, что для любого целого $k \geqslant 0$ автомат
распознает слово $xy^kz$. Действительно, последовательность переходов при чтении
цепочек $x$ и $z$ остаётся такой же, как для слова $w$. Часть $y^k$ читается
$k$\nbdash{}кратным повтором последовательности переходов $\bigtriangleup$. Таким
образом, $\{ xy^kz \mid k \geqslant 0\} \subset L$ и выполнено
условие~\eqref{pumping}.

Части $x, y$ слова $w$ удовлетворяют
условиям~\eqref{ynotempty}–\eqref{shortprefix} по построению. Полагая
$n=N+1$, получаем выполненными все условия теоремы.
\end{proof}

\section{Удаление бесполезных символов}
\begin{Def}Пусть дана КС-грамматика $G=(\Sigma, N, \mathcal P, S \in N)$.
Говорят, что символ $X \in \Sigma \cup N$ 
\begin{enumerate}
    \item \emph{полезный в $G$}, если
    $$
        \exists \alpha, \beta \in (\Sigma \cup N)^{\ast} \:
        \exists w \in \Sigma^* \colon \quad
        S \Rightarrow^{\ast}_G \alpha X \beta \Rightarrow^{\ast}_G w. 
    $$
    \item \emph{порождающий в $G$}, если
    $$
        \exists w \in \Sigma^* \colon \quad X \Rightarrow^{\ast}_G w 
    $$
    \item  \emph{достижимый в $G$}, если
    $$
        \exists \alpha, \beta \in (\Sigma \cup N)^{\ast} \colon \quad 
        S \Rightarrow^{\ast}_G \alpha X \beta
    $$
\end{enumerate}
\emph{Бесполезным} называется любой символ, не являющийся полезным.
\end{Def}

\begin{Thm} 
Если к КС\nbdash{}грамматике $G$, такой что $L(G) \neq \es$,
\emph{последовательно} применить два преобразования:
\begin{enumerate}
  \item удалить символы, не являющиеся порождающими,
  \item удалить символы, не являющиеся достижимыми,
\end{enumerate}
то будет получена грамматика, не содержащая бесполезных символов.
\end{Thm}
\begin{proof}
Обозначим грамматику, получившуюся после первого шага, $G_1 = (N_1, \Sigma_1,
\mathcal P_1, S)$ и грамматику, получившуюся после второго шага, $G_2 = (N_2,
\Sigma_2, \mathcal P_2, S)$ (символ $S$ не будет удaлён из грамматики после
первого шага благодаря требованию $L(G) \neq \es$ и после второго шага~— по
определению достижимого символа). Пусть $X \in N_2 \cup \Sigma_2$. Покажем, что
$X$~— полезный в $G_2$ символ. Для этого необходимо найти $\alpha, \beta \in
(N_2 \cup \Sigma_2)^*$ и $w' \in \Sigma_2^*$, такие что
\begin{equation}\label{isxusefull}
    S \Rightarrow^{*}_{G_2} \alpha X \beta \Rightarrow^{*}_{G_2} w'.
\end{equation}
Так как $X$ не был удалён после первого шага, то он является порождающим в
исходной грамматике $G$, то есть существует вывод $X \Rightarrow^*_G w$ для
некоторого $w \in \Sigma^*$. Очевидно, что каждый символ этого вывода рано или
поздно переходит в один из символов $w$ или в $\varepsilon$, то есть также
является порождающим в $G$, а значит, попадает в $G_1$. Потому этот вывод можно
целиком перенести в $G_1$:
\begin{equation}\label{xgen}
    X \Rightarrow^*_{G_1} w,
\end{equation}
где $w \in  \Sigma_1^*$.

Так как $X$ не был удалён после второго шага, то он достижим в грамматике
$G_1$, то есть существует вывод $S \Rightarrow^{*}_{G_1} \alpha X \beta$
для некоторых $\alpha, \beta \in (N_1 \cup \Sigma_1)^*$. Очевидно, что каждый
символ в этом выводе получен из $S$ применением правил грамматики $G_1$, то
есть является достижимым в $G_1$, а значит, попадает в $G_2$. Потому этот вывод
можно целиком перенести в $G_2$:
\begin{equation}\label{xreach}
    S \Rightarrow^{*}_{G_2} \alpha X \beta,
\end{equation}
где $\alpha, \beta \in (N_2 \cup \Sigma_2)^*$.

Так как $X$ достижим в $G_1$, то и каждый символ вывода~\eqref{xgen} достижим в
$G_1$, а значит, этот вывод целиком переносится в $G_2$:
\begin{equation}\label{xreachgsec}
    X \Rightarrow^*_{G_2}w, 
\end{equation}
где $w \in  \Sigma_2^*$.

Так как $\alpha, \beta \in (N_2 \cup \Sigma_2)^*$, то к цепочкам $\alpha$ и
$\beta$ применимы те же рассуждения, что к символу $X$, то есть можно получить
результат, аналогичный~\eqref{xreachgsec}:
\begin{align}
\exists u \in \Sigma_2^*\colon \alpha \Rightarrow^{*}_{G_2} u,\\
\exists v \in \Sigma_2^*\colon \beta \Rightarrow^{*}_{G_2} v.
\label{alphabetagen}
\end{align}
Объединяя результаты~\eqref{xreach}–\eqref{alphabetagen}, получаем:
$$
    S \Rightarrow^{*}_{G_2} \alpha X \beta \Rightarrow^{*}_{G_2} uwv,
$$
то есть выполнено~\eqref{isxusefull}.
\end{proof}

\section{Нахождение языка конечного автомата методом исключения состояний}
Метод исключения состояний подразумевает последовательное удаление вершин графа
переходов автомата, которое протоколируется с помощью записи на оставшихся дугах
регулярных выражений (можно считать, что в изначальном графе на дугах
простейшие регулярные выражения — однобуквенные или пустые множества для
отсутствующих дуг).

\textbf{Процедура исключения состояния $s$:} для каждых двух (необязательно
различных, но несовпадающих с $s$) состояний $p$ и $q$, таких что существует
фрагмент графа переходов автомата $p \xrightarrow{R_1} s \xrightarrow{R_2} q$,
где $R_1$, $R_2$ — некоторые регулярные выражения (метки дуг переходов),
прибавить к метке дуги $p \xrightarrow{R_3} q$ выражение $R_1R^{\ast}R_2$, где
$R$ это метка петли на вершине $s$ (если петля на $s$ и/или дуга $p \to q$
отсутствовали в исходном графе, то можно считать, что их метки равны
$\varnothing$) — таким образом получена дуга с меткой: $p \xrightarrow{R_3 +
R_1R^{\ast}R_2} q$. Удалить все просмотренные дуги $p \to s$ и $s \to q$,
инцидентные вершине $s$. После этого $s$ стала изолированной или
(неориентированно) висячей и её можно удалить (с входящими или выходящими из неё
дугами, если таковые имеются).

\begin{Algo}[нахождение языка автомата методом исключения состояний]
\nspace\\
\textsc{Вход}: Конечный автомат $\mathcal A = (Q, \Sigma, \delta, q_0, F)$,
заданный графом переходов.\\
\textsc{Выход}: Регулярное выражение, описывающее язык  автомата.
\begin{enumerate}
\renewcommand{\labelenumi}{Шаг \theenumi{}.}
  \item\label{Final} Для каждого финального состояния $q\in F$, отличного от
  начального $q_0$, применять процедуру исключения состояний до тех пор, пока не останутся
  две вершины: $q_0$ и $q$. В результате получится подобный автомат: 
  \begin{center}
  \begin{tikzpicture}[node distance=3cm,auto,%
                        every state/.style={thick},>=stealth']
    \node[state,initial,initial text=] (q_0) {$q_0$};
    \node[state,accepting] (q) [right of=q_0] {$q$};

    \path[->] 
        (q_0) edge [loop above] node 
            {$R$} ()
        (q_0) edge[bend left=40] node 
            {$S$} (q)
        (q) edge [loop above] node 
            {$U$} 
            ()
        (q) edge[bend left=40] node 
            {$T$} (q_0);
    \end{tikzpicture}
    \end{center}
    Допускаемый им язык описывается так:
    $$(R+SU^{\ast}T)^{\ast}SU^{\ast}.$$
    
    \item\label{InitIsFinal} Если начальное состояние $q_0$ является финальным
    ($q_0\in F$), применять процедуру исключения состояний, пока не останется
    единственная вершина $q_0$. В результате получится подобный автомат:
    \begin{center}
    \begin{tikzpicture}[node distance=3cm,auto,%
                        every state/.style={thick},>=stealth'] 
        \node[state,initial,initial text=,accepting] (q_0) {$q_0$};
        
        \path[->] 
            (q_0) edge [loop above] node 
                {$R$} ();
    \end{tikzpicture}
    \end{center}
    Допускаемый им язык описывается так: $R^{\ast}$.

    \item Язык исходного автомата определяется как сумма всех регулярных выражений,
    полученных на шагах \eqref{Final}–\eqref{InitIsFinal}.
    \end{enumerate}
\end{Algo}

\section{Свойства замкнутости класса регулярных языков}
\begin{Thm}Класс регулярных языков замкнут относительно операций: объединения,
конкатенации, итерации, дополнения, пересечения, разности.
\end{Thm}
\begin{proof}Для регулярных языков $L_1$, $L_2$ рассмотрим описывающие
их регулярные выражения $\RE(L_1)$, $\RE(L_2)$. Язык объединения
(соответственно, конкатенации, итерации) описывается выражением $\RE(L_1) +
\RE(L_2)$ (соответственно, $\RE(L_1)\RE(L_2)$, $\RE(L_1)^\ast$), а значит,
регулярен.

Для регулярного языка $L$ построим распознающий его детерминированный конечный
автомат $\mathcal A = (Q, \Sigma, \delta, q_0, F)$ ($L=L(\mathcal A)$). Легко
видеть, что автомат $\widetilde{\mathcal A} = (Q, \Sigma, \delta, q_0, Q
\setminus F)$ допускает дополнение $\overline L = \Sigma^\ast \setminus L$ языка
$L$, которое является, таким образом, регулярным языком.

Поскольку справедливо $L_1 \cap L_2 = \overline{\overline L_1 \cup \overline
L_2}$, язык $L_1 \cap L_2$ регулярен по доказанному выше. Аналогичное можно 
заключить из равенства $L_1 \setminus L_2 = L_1 \cap \overline{L_2}$.
\end{proof}

% \section{Алгоритм Кока—Янгера—Касами решения проблемы принадлежности для
% КС-языков}%\nbdash{}
% \begin{Algo}[Кок—Янгер—Касами, «CYK\nbdash{}алгоритм»]
% \nspace\\
% \textsc{Вход}: грамматика $G=(\Sigma, N, \mathcal P, S \in N)$ в НФХ,
% слово $w \in \Sigma^*$.\\
% \textsc{Выход}: да, $w \in L(G)$ / нет, $w \not \in L(G)$.\\
% \textsc{Метод}: последовательное определение нетерминалов, выводящих
% всевозможные подстроки $w$ всё большей длины.
% 
% Пусть $w = w_1 \ldots w_n$. Для всех $1 \leqslant i \leqslant j \leqslant n$
% определим множество 
% $$
%     N_{ij} = \{ A \in N \mid A \Rightarrow^*_G w_i \ldots w_j \}.
% $$
% Очевидно, что $w \in L(G) \Leftrightarrow S \in N_{1n}$. Приведём алгоритм
% построения множеств $N_{ij}$.
% \begin{codebox}
% \zi\For $i \gets 1$ \To $n$
% \zi     \Do
%         $N_{ii} \gets \{ A \in N \mid A \to w_i \in \mathcal P  \}$ 
%         \Comment Подстроки $w$ длины $1$
%         \End
% \zi\For $s \gets 2$ \To $n$ \Comment Цикл по длине подстроки
% \zi     \Do
%         \For $i \gets 1$ \To $n - s + 1$ \Comment Цикл по месту начала подстроки
% \zi         $j \gets i + s - 1$ 
%             \Comment Позиция конца подстроки с началом в $w_i$ длины $s$
% \zi         $N_{ij} \gets \{ A \in N \mid A \to BC \in \mathcal P; \;
%                 \exists k \in [i, j-1]_{\mathbb Z} \colon B \in N_{ik}, \; 
%                 C \in N_{k+1 j} \} $
% \end{codebox}
% \end{Algo}
% \begin{NumRemark}
% Алгоритм удобно выполнять, заполняя таблицу с $N_{ij}$ в ячейках. Ячейки таблицы
% расположены в системе координат $(i,s)$, в позиции $(i,s)$ находится множество
% $N_{i, i+s-1}$.
% \end{NumRemark}
% 
% \begin{NumRemark}[о сложности CYK-алгоритма]
% Нетрудно видеть, что сложность CYK\nbdash{}алгоритма может быть оценена как
% $O(n^3 \cdot |\mathcal P|)$.
% \end{NumRemark} 
