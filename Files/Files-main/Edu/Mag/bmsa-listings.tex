\subsection*{BMSAlgorithm.h}
\begin{lstlisting}
#ifndef _BMSALGORITHM_H
#define	_BMSALGORITHM_H

#include <algorithm>
#include <cassert>
#include <iostream>
#include <map>
#include <set>
#include <utility>
#include <vector>

#include "PolySet.h"
#include "Polynomial.h"
#include "Sequence.h"
#include "DeltaSet.h"

namespace bmsa {

using std::vector;
using std::pair;
using namespace boost::lambda;
using namespace boost::tuples;

template<typename T, typename S = T> // T = Poly Coef Type
                                     // S = Seq Elem Type
class Algorithm {
    typedef std::list<Point> PointCollection;
    typedef std::list<PolySubsetEntry<T,S> > PolySubset;

    Sequence<S> const & u;

    Point n; // current step of the algorithm

    PolySet<T,S> F; // current minimal set
    PolyWithAuxInfoSet<T,S> G;

    PolySubset FN;
    PolySubset FV;

    vector<typename Polynomial<T,S>::Coef> discr; // current
                                              // discrepancies
    vector<size_t> degreeInvariantSuppliers;
    DeltaSet newDeltaSet;

public:
    Algorithm(Sequence<S> const & u_) : u(u_),
            F(ElemTypeTraits<typename Polynomial<T,S>::Coef>::id()) {}

    PolySet<T,S> computeMinimalSet();

private:
    void clearParams();

    void buildFNandFV();

    bool isAtTheDegreeInvariantPoint(
            vector<size_t> & degreeInvariantSuppliers);

    void renewF(vector<size_t> const & degreeInvariantSuppliers);

    // Berlekamp procedure
    Polynomial<T,S> BP(size_t i, size_t j);

    // Subsidiary procedure -- the confluent version of BP
    template <size_t K>
    Polynomial<T,S> SP(size_t);

    void buildNewDeltaSet();
    PolySet<T,S> buildNewF();
    void buildNewG(PolySet<T,S> const &);
    bool fnBetweenFs(Polynomial<T,S> const & f, 
                Polynomial<T,S> const & fNext,
                Polynomial<T,S> const & fn) const;

    // constitutes building new delta-set
    void addPointsOfTypeI();
    void addPointsOfTypeII();
    void addPointsOfTypeIII();
    void addPointsOfTypeIV();

    DeltaSetPoint
    pointOfTypeIFromPolySubsetEntry(PolySubsetEntry<T,S> const &);

    Polynomial<T,S>
    buildPolyFromDeltaSetPoint(DeltaSetPoint const &);
};

} // namespace bmsa
#endif	/* _BMSALGORITHM_H */
\end{lstlisting}

\subsection*{BMSAlgorithm.cc}
\begin{lstlisting}
#include <algorithm>
#include <stdexcept>
#include <iterator>
#include <iostream>

#include <cstddef> // size_t

#include <boost/lambda/lambda.hpp>
#include <boost/lambda/bind.hpp>
#include <boost/function.hpp>

#include "Polynomial.h"
#include "PolySet.h"
#include "BMSAlgorithm.h"

using std::cout;
using std::endl;

namespace bmsa {

template<typename T, typename S>
PolySet<T,S> Algorithm<T,S>::computeMinimalSet() {
    for (n = Point::nil; totalLess(n, u.size()); ++n ) {
        clearParams();
        buildFNandFV(); // find polynomials non-valid
                        // at the current step
        if (FN.empty()) {
            cout << "F_N is empty - F would not change" << endl;
            continue;
        }
        if (// n is in the degree-invariant set for each f in FN
                isAtTheDegreeInvariantPoint(degreeInvariantSuppliers)) {
            cout << "Delta-set would not change" << endl;
            // delta-set and size of F unchanged
            renewF(degreeInvariantSuppliers);
            std::cout << "F renewed:\n\t" << F << std::endl;
        } else {
            buildNewDeltaSet();
            cout << "New delta-set: "<< endl << "\t" << newDeltaSet;
            PolySet<T,S> F_new = buildNewF();
            buildNewG(F_new);
            F = F_new;
        }
    }
    return F;
}

template<typename T, typename S>
void Algorithm<T,S>::clearParams() {
    FN.clear();
    FV.clear();
    degreeInvariantSuppliers.clear();
    degreeInvariantSuppliers.resize(F.size());
    discr.clear();
    newDeltaSet.clear();
}

template<typename T, typename S>
void Algorithm<T,S>::buildFNandFV() {
    size_t i = 0;
    for(
            typename PolySet<T,S>::const_iterator fIt = F.begin();
            fIt != F.end();
            ++fIt, ++i) {
        typename Polynomial<T,S>::Coef d; // current discrepancy;
        // for f in F: if f[u](n) could be evaluated...
        // .. and discrepancy not null then 
        // f is non-valid (goes to FN)
        if ( fIt->getDegree() <= n
                && (d = (*fIt)[u](n)) != typename Polynomial<T,S>::Coef() ) {
            discr.push_back(d);
            FN.push_back(PolySubsetEntry<T,S>(i, fIt));
        }
        else {
            discr.push_back(typename Polynomial<T,S>::Coef());
            FV.push_back(PolySubsetEntry<T,S>(i, fIt));
        }
    }
}

// test wether n is degree-invariant point for all f in FN;
// if yes, degreeInvariantSuppliers holds indices of aux poly's
// which help to preserve the degree in the Berlekamp-procedure;
// this info should be used in such a way (pseudocode):
//      for f in FN:
//          f = BP<f, degreeInvariantSuppliers[indexOf(f)]>
template<typename T, typename S>
bool Algorithm<T,S>::isAtTheDegreeInvariantPoint(
        vector<size_t> & degreeInvariantSuppliers) {
    bool result = true;
    for(typename PolySubset::const_iterator it = FN.begin();
            it != FN.end(); ++it) {
        size_t degInv
                = F.isInShiftedDeltaSet(n, it->polyIt->getDegree());
        result = !(degInv == PolySet<T,S>::NOT_IN_SHIFTED_DELTA_SET);
        degreeInvariantSuppliers[it->polyIndex] = degInv;
    }
    return result;
}

template<typename T, typename S>
void Algorithm<T,S>::renewF(vector<size_t> const & degreeInvariantSuppliers) {
    for(typename PolySubset::const_iterator it = FN.begin();
            it != FN.end(); ++it) {
        size_t nonvalidPolyIdx = it->polyIndex;
        F.replace(it->polyIt, BP(nonvalidPolyIdx,
                degreeInvariantSuppliers[nonvalidPolyIdx]));
    }
}

template<typename T, typename S>
void Algorithm<T,S>::buildNewDeltaSet() {
    addPointsOfTypeI();
    addPointsOfTypeII();
    addPointsOfTypeIII();
    addPointsOfTypeIV();
}

// though polys from FV give points of type I,
// they are not added to delta-set
// (due to some technical reasons arise in building F_new
// from this delts-set and possible improoved efficiency);
// buildNewF() (builds F_new) take this into account;
template<typename T, typename S>
void Algorithm<T,S>::addPointsOfTypeI() {
    for (typename PolySubset::const_iterator it = FN.begin(); it != FN.end(); ++it)
        if (degreeInvariantSuppliers[it->polyIndex]
                != PolySet<T,S>::NOT_IN_SHIFTED_DELTA_SET)
            newDeltaSet.push_back(
                    DeltaSetPoint(DeltaSetPoint::I, it->polyIndex,
                            degreeInvariantSuppliers[it->polyIndex]));
}

template<typename T, typename S>
DeltaSetPoint
Algorithm<T,S>::pointOfTypeIFromPolySubsetEntry(PolySubsetEntry<T,S> const & psentry) {
    return DeltaSetPoint(DeltaSetPoint::I, psentry.polyIndex,
            degreeInvariantSuppliers[psentry.polyIndex]);
}

template<typename T, typename S>
void Algorithm<T,S>::addPointsOfTypeII() {
    typename PolySubset::iterator it = std::adjacent_find(FN.begin(), FN.end(),
            adjacentPolyIndiciesForPolySubset<T,S>);
    while (it != FN.end()) {
        newDeltaSet.push_back(
            DeltaSetPoint(DeltaSetPoint::II, it->polyIndex,
                it->polyIndex)
        );
        it = std::adjacent_find(++it, FN.end(),
                adjacentPolyIndiciesForPolySubset<T,S>);
    }
}

template<typename T, typename S>
void Algorithm<T,S>::addPointsOfTypeIII() {
    using namespace boost::lambda;
    for (typename PolySubset::const_iterator jIt = FN.begin(); jIt != FN.end(); ++jIt) {
        const Point s_j = jIt->polyIt->getDegree();
        for (typename PolySet<T,S>::reverse_iterator kIt = F.rbegin(); kIt != F.rend(); ++kIt)
            if ( n[1] >= s_j[1] + kIt->getDegree()[1]) {
                if (n[0] >= s_j[0] + kIt->getDegree()[0]) {
                    typename PolySubset::const_iterator iIt
                        = std::find_if(FN.begin(), FN.end(),
                            bind(polyItWithPolySubsetEntryComparator<T,S>, --kIt.base(), _1));
                    if (iIt != FN.end())
                        newDeltaSet.push_back(DeltaSetPoint(DeltaSetPoint::III,
                                iIt->polyIndex, jIt->polyIndex));
                    else
                        break;
                }
                else
                    break;
            }
    }
}

template<typename T, typename S>
void Algorithm<T,S>::addPointsOfTypeIV() {
    using namespace boost::lambda;
    for (typename PolySubset::const_iterator jIt = FN.begin(); jIt != FN.end(); ++jIt) {
        const Point s_j = jIt->polyIt->getDegree();
        for (typename PolySet<T,S>::const_iterator kIt = F.begin(); kIt != F.end(); ++kIt)
            if ( n[0] >= s_j[0] + kIt->getDegree()[0]) {
                if (n[1] >= s_j[1] + kIt->getDegree()[1]) {
                    typename PolySubset::const_iterator iIt
                        = std::find_if(FN.begin(), FN.end(),
                            bind(polyItWithPolySubsetEntryComparator<T,S>, kIt, _1));
                    if (iIt != FN.end())
                        newDeltaSet.push_back(DeltaSetPoint(DeltaSetPoint::IV,
                                iIt->polyIndex, jIt->polyIndex));
                    else
                        break;
                }
                else
                    break;
            }
    }
}

template<typename T, typename S>
Polynomial<T,S> Algorithm<T,S>::buildPolyFromDeltaSetPoint(DeltaSetPoint const& pt) {
    std::cout << "DSPoint: " << pt;
    Polynomial<T,S> res;
    size_t k;
    Point t;
    switch(pt.pointType) {
        case DeltaSetPoint::I:
            res = BP(pt.i, pt.j); break;
        case DeltaSetPoint::II:
            t = Point(n[0] - F[pt.i].getDegree()[0] + 1,
                    n[1] - F[pt.j].getDegree()[1] + 1);
            k = findPolyWithLessDegIdxInPolySubset<T,S>(FN, t);
            res = BP(k, pt.i); break;
        case DeltaSetPoint::III:
            if (pt.i != F.size() - 1)
                res = BP(pt.j, pt.i);
            else
                res = SP<0>(pt.j);
            break;
        case DeltaSetPoint::IV:
            if (pt.j != 0)
                res = BP(pt.i, pt.j - 1);
            else
                res = SP<1>(pt.i);
            break;
        default: std::logic_error("Illegal type of point in delta-set!");
    }
    return res;
}

template<typename T, typename S>
PolySet<T,S> Algorithm<T,S>::buildNewF() {
    using namespace boost::lambda;
    PolySet<T,S> res;
    add(res, FV); // as points from FV weren't add to newDeltaSet
    for(DeltaSet::const_iterator it = newDeltaSet.begin();
            it != newDeltaSet.end(); ++it) {
        Polynomial<T,S> p = buildPolyFromDeltaSetPoint(*it);
        res.insert(p);
    }
    return res;
}

template<typename T, typename S>
bool gBetweenFs(Polynomial<T,S> const & f, Polynomial<T,S> const & fNext,
        PolyWithAuxInfo<T,S> const & g) {
    return f.getDegree()[0] == g.p[0] - g.poly.getDegree()[0] + 1
        && fNext.getDegree()[1] == g.p[1] - g.poly.getDegree()[1] + 1;
}

template<typename T, typename S>
bool Algorithm<T,S>::fnBetweenFs(
        Polynomial<T,S> const & f,
        Polynomial<T,S> const & fNext,
        Polynomial<T,S> const & fn) const {
    return f.getDegree()[0] == n[0] - fn.getDegree()[0] + 1
        && fNext.getDegree()[1] == n[1] - fn.getDegree()[1] + 1;
}

template<typename T, typename S>
void Algorithm<T,S>::buildNewG(PolySet<T,S> const & F_new) {
    PolyWithAuxInfoSet<T,S> GNew;
    for (typename PolySet<T,S>::const_iterator fit = F_new.begin(),
            fitNext = ++F_new.begin();
            fit != --F_new.end(); ++fit, ++fitNext) {
        bool gfound = false;
        for (typename PolyWithAuxInfoSet<T,S>::iterator git = G.begin();
                git != G.end() && !gfound;
                ++git) {
            if (gBetweenFs(*fit, *fitNext, *git)) {
                GNew.insert(*git);
                G.erase(git); // git will not be used any more
                gfound = true;
            }
        }
        if (gfound)
            continue;
        for(typename PolySubset::iterator fnit = FN.begin();
                fnit != FN.end() && !gfound;
                ++fnit) {
            Polynomial<T,S> const & fnPoly = *(fnit->polyIt);
            if (fnBetweenFs(*fit, *fitNext, fnPoly)) {
                GNew.insert( PolyWithAuxInfo<T,S>(fnPoly, n, fnPoly[u](n)) );
                FN.erase(fnit);
                gfound = true;
            }
        }
        G = GNew;
    }
}

// Berlekamp procedure
template<typename T, typename S>
Polynomial<T,S> Algorithm<T,S>::BP(size_t i, size_t j) {
    assert( 0 <= i && i < F.size());
    assert( 0 <= j && j < G.size());

    std::cout << "BP<" << i << ", " << j << ">" << endl;
    Polynomial<T,S> const & f = F[i];
    Polynomial<T,S> const & g = G[j].poly;
    Point r = Point(
        std::max(f.getDegree()[0], g.getDegree()[0] + n[0] - G[j].p[0]),
        std::max(f.getDegree()[1], g.getDegree()[1] + n[1] - G[j].p[1])
    );
    Polynomial<T,S> f1 = f.onMonomialMultiply(r - f.getDegree());
    Polynomial<T,S> f2 = g.onMonomialMultiply(r - n + G[j].p - g.getDegree());
    typename Polynomial<T,S>::Coef c = (-discr[i] / G[j].d);
    Polynomial<T,S> res = f1 + c * f2;
    return res;
}

template<typename T, typename S>
template<size_t K>
inline
Polynomial<T,S> Algorithm<T,S>::SP(size_t m) {
    Point deg;
    deg[K] = n[K] - F[m].getDegree()[K] + 1;
    return F[m].onMonomialMultiply(deg);
}

} // namespace bmsa
\end{lstlisting}

\subsection*{PolySet.h}
\begin{lstlisting}
#ifndef _POLYSET_H
#define	_POLYSET_H

#include <algorithm>
#include <iostream>
#include <iterator>
#include <limits>
#include <list>
#include <set>
#include <stdexcept>
#include <utility>

#include <boost/lambda/bind.hpp>

#include "Polynomial.h"

namespace bmsa {

template<typename T, typename S = T>
class PolySet {
    typedef Polynomial<T,S> PolynomialT;
    typedef std::set<PolynomialT> Container;

    Container data;

public:
    typedef typename Container::iterator iterator;
    typedef typename Container::const_iterator const_iterator;
    typedef typename Container::reverse_iterator reverse_iterator;
    typedef typename Container::const_reference const_reference;
    typedef typename Container::reference reference;

    static size_t NOT_IN_SHIFTED_DELTA_SET;

    PolySet() {}

    PolySet(PolynomialT const & p) {
        data.insert(p);
    }

    const_iterator begin() const        {return data.begin();}

    const_iterator end() const          {return data.end();}

    reverse_iterator rbegin()           {return data.rbegin();}

    reverse_iterator rend()             {return data.rend();}

    std::pair<iterator, bool>
    insert(PolynomialT const & poly)
        {return data.insert(poly);}

    iterator insert(iterator pos, PolynomialT const & poly)
        {return data.insert(pos, poly);}

    typename Container::size_type size() const   
        {return data.size();}

    size_t isInShiftedDeltaSet(Point const & pt, Point const & shift);

    void replace(const_iterator const & oldEntry,
            PolynomialT const & newEntry);

    // 0-based poly index in PolySet
    PolynomialT const & operator[](int n) const;

};

template<typename T, typename S>
inline
typename PolySet<T,S>::PolynomialT const &
PolySet<T,S>::operator[](int n) const {
    const_iterator it = data.begin();
    std::advance(it, n);
    return *it;
}

template<typename T, typename S>
inline
void PolySet<T, S>::replace(const_iterator const & oldEntry,
        Polynomial<T, S> const & newEntry) {
    std::cout << "Replacing a poly in the PolySet:" << std::endl
        << "\told entry: " << *oldEntry << std::endl;
    data.erase(oldEntry);
    std::cout << "\tnew entry: " << newEntry << std::endl;
    data.insert(newEntry);
}

template<typename T, typename S>
inline
std::ostream& operator<<(std::ostream& os, PolySet<T, S> const & ps) {
    os << "{ ";
    std::copy(ps.begin(), ps.end(),
            std::ostream_iterator<Polynomial<T, S> >(os, "; "));
    os << "}";
    return os;
}
/***************************************************************/

template<typename T, typename S>
struct PolySubsetEntry {
    size_t polyIndex;
    typename PolySet<T,S>::const_iterator polyIt;
    //Polynomial::Coef discrepancy;

    // constructor from fields
    PolySubsetEntry(
            size_t polyIndex_,
            typename PolySet<T,S>::const_iterator const & polyIt_) :
                polyIndex(polyIndex_), polyIt(polyIt_) {}
};

template<typename T, typename S>
inline
bool adjacentPolyIndiciesForPolySubset(
        PolySubsetEntry<T,S> const & pse1,
        PolySubsetEntry<T,S> const & pse2) {
    return pse1.polyIndex + 1 == pse2.polyIndex;
}

template<typename T, typename S>
inline
bool polyItWithPolySubsetEntryComparator(
        typename PolySet<T,S>::iterator it,
        PolySubsetEntry<T,S> const & pse) {
    return it == pse.polyIt;
}

template<typename T, typename S>
inline
bool greaterDegreeThenInPolySubsetEntry(Point const & deg,
        PolySubsetEntry<T,S> const & pse) {
    return pse.polyIt->getDegree() < deg;
}

template<typename T, typename S, typename PolySubset>
inline
size_t findPolyWithLessDegIdxInPolySubset(PolySubset const & ss, Point t) {
    using namespace boost::lambda;
    typename PolySubset::const_iterator it = std::find_if(
            ss.begin(), 
            ss.end(),
            bind(greaterDegreeThenInPolySubsetEntry<T,S>, t, _1));
    if (it == ss.end())
        throw std::logic_error("No poly in given subset which"
                " degree is less then given");
    return it->polyIndex;
}

template<typename T, typename S>
inline
std::ostream& operator<<(std::ostream& os, PolySubsetEntry<T,S> const & pse) {
    os << *pse.polyIt << " idx: " << pse.polyIndex;
    return os;
}
/***************************************************************/

template<typename T, typename S>
struct PolyWithAuxInfo {
    Polynomial<T,S> poly;

    Point p; // 'potential' -- last step at which corresponding
                // polynomial had been valid for given sequence;
    typename Polynomial<T,S>::Coef d; //discrepancy

    PolyWithAuxInfo(Polynomial<T,S> const & poly_, Point const & p_,
            typename Polynomial<T,S>::Coef const & d_)
            : poly(poly_), p(p_), d(d_) {}

    Polynomial<T,S>     const * getPoly()    const   {return &poly;}
    Point               const * getP()       const   {return &p;}
};

template<typename T, typename S>
inline
std::ostream& operator<<(std::ostream& os, PolyWithAuxInfo<T,S> const & pwai) {
    os << pwai.poly << " pot: " << pwai.p << " discr: " << pwai.d;
    return os;
}

template<typename T, typename S>
inline
bool operator<(PolyWithAuxInfo<T,S> const & pwai1,
                PolyWithAuxInfo<T,S> const & pwai2) {
    return pwai1.p[0] - pwai1.poly.getDegree()[0]
            > pwai2.p[0] - pwai2.poly.getDegree()[0]
            && pwai1.p[1] - pwai1.poly.getDegree()[1]
                < pwai2.p[1] - pwai2.poly.getDegree()[1];
}

template<typename T, typename S>
class PolyWithAuxInfoSet {
    typedef std::set<PolyWithAuxInfo<T,S> > Container;

    Container data;

public:
    typedef typename Container::iterator iterator;
    typedef typename Container::const_iterator const_iterator;

    const_iterator begin() const        {return data.begin();}

    iterator begin()                    {return data.begin();}

    const_iterator end() const          {return data.end();}

    iterator end()                      {return data.end();}

    std::pair<iterator, bool> 
    insert(PolyWithAuxInfo<T,S> const & pwai) 
        {return data.insert(pwai);}

    void erase(iterator pos)            {return data.erase(pos);}

    typename Container::size_type size() const   
        {return data.size();}

    PolyWithAuxInfo<T,S> const & operator[](size_t n) const;
};

template<typename T, typename S>
inline
PolyWithAuxInfo<T,S> const & PolyWithAuxInfoSet<T,S>::operator[](size_t n) const {
    const_iterator it = data.begin();
    std::advance(it, n);
    return *it;
}

template<typename T, typename S>
inline
std::ostream& operator<<(std::ostream& os,
        PolyWithAuxInfoSet<T,S> const & pwaiset) {
    std::copy(pwaiset.begin(), pwaiset.end(),
            std::ostream_iterator<PolyWithAuxInfo<T,S> >(os, "; "));
    return os;
}


template<typename T, typename S>
size_t PolySet<T,S>::NOT_IN_SHIFTED_DELTA_SET
                            = std::numeric_limits<size_t>::max();

// returns the 0-based index of F-element for which 
// pt is in shifted delta-set;
// NOT_IN_SHIFTED_DELTA_SET if n is not in shifted delta-set 
// for any f from F
template<typename T, typename S>
size_t PolySet<T,S>::isInShiftedDeltaSet(Point const & pt, Point const & shift) {
    if (data.size() < 2)
        return NOT_IN_SHIFTED_DELTA_SET; // delta-set is empty
    size_t fIdx = 0;
    for(const_iterator fIt = data.begin(); fIt != --data.end();
            ++fIt, ++fIdx) {
        const_iterator fNextIt(fIt);
        ++fNextIt;
        if (pt[0] < shift[0] + fIt->getDegree()[0]
            && pt[1] < shift[1] + fNextIt->getDegree()[1])
            return fIdx;
    }
    return NOT_IN_SHIFTED_DELTA_SET;
}

template<typename T, typename S>
Polynomial<T,S> polyFromPolySubsetEntry(PolySubsetEntry<T,S> const & pse) {
    return *pse.polyIt;
}

template<typename T, typename S, typename PolySubset>
void add(PolySet<T,S> & ps, PolySubset const & pss) {
    std::transform(pss.begin(), pss.end(), std::inserter(ps, ps.begin()),
            polyFromPolySubsetEntry<T,S>);
}
    
} // namespace bmsa
\end{lstlisting}

\subsection*{Poly.h}
\begin{lstlisting}
#ifndef _POLY_H
#define _POLY_H
#include <algorithm>
#include <deque>
#include <iostream>
#include <vector>

#include "ElementType.h"
#include "Sequence.h"
#include "Point.h"

namespace bmsa {

template <typename CoefT, typename SeqElemT = CoefT>
class Polynomial {
public:
    typedef CoefT Coef;
    typedef std::deque<Coef> OneDimStorage;
    typedef std::deque< OneDimStorage > StorageT;

    typedef Point DegreeT;

private:
    class ProxyPolySeqProduct {
        Polynomial const & poly;
        Sequence<SeqElemT> const & seq;
    public:
        ProxyPolySeqProduct(Polynomial const & poly_,
                Sequence<SeqElemT> const & seq_) : poly(poly_), seq(seq_) {}

        typename Polynomial::Coef operator()(Point const & pr) const;
    };

    friend class ProxyPolySeqProduct;

public:
    Polynomial() {}

    Polynomial(Coef const & c) : coefs(1) {
        coefs[0].push_back(c);
    }

    StorageT const & getCoefs() const { return coefs; }

    void setCoefs(StorageT const & data);

    DegreeT getDegree() const { return degree; }

    Polynomial& operator*=(Coef const &);

    Polynomial onMonomialMultiply(DegreeT const & monomialDegree) const;

    Polynomial& operator+=(Polynomial const & rhs);

    ProxyPolySeqProduct operator[](Sequence<SeqElemT> const & seq) const;

private:
    StorageT coefs;

    DegreeT degree;

    void countDegree();

};
/**************** End of Polynomial *********************/

template<typename T, typename S>
inline
Polynomial<T, S> operator*(typename Polynomial<T,S>::Coef const & coef,
        Polynomial<T, S> poly) {
    return poly *= coef;
}

template<typename T, typename S>
inline
Polynomial<T, S> operator*(Polynomial<T, S> poly,
        typename Polynomial<T, S>::Coef const & coef) {
    return coef*poly;
}

template<typename T, typename S>
inline
Polynomial<T, S> operator+(Polynomial<T, S> lhs,
        Polynomial<T, S> const & rhs) {
    return lhs += rhs;
}

template<typename T, typename S>
inline
// for establishing delta-set
bool operator<(Polynomial<T, S> const & lhs, Polynomial<T, S> const & rhs) {
    return lhs.getDegree()[0] > rhs.getDegree()[0]
            && lhs.getDegree()[1] < rhs.getDegree()[1];
}

// format: [[a b c][e f]] - witout spaces in "]["
template<typename T, typename S>
std::istream& operator>>(std::istream& is, Polynomial<T, S> & poly);

template<typename T, typename S>
std::ostream& operator<<(std::ostream& os, Polynomial<T, S> const & poly);

} // namespace bmsa

#endif  /* _POLY_H */
\end{lstlisting}

\subsection*{Poly.cc}
\begin{lstlisting}
#include <algorithm>
#include <cstddef>
#include <functional>
#include <iostream>
#include <ios>
#include <iterator>
#include <numeric>
#include <sstream>
#include <string>
#include <vector>

#include <boost/lambda/lambda.hpp>
#include <boost/lambda/bind.hpp>
#include <boost/lambda/algorithm.hpp>

#include "GenericTwoDimIO.h"
#include "Poly.h"

namespace bmsa {

/****************** Arithmetic operations ***********************/
template<typename T, typename S>
class OneDimAddition {
    typedef int IntType;
    typedef typename Polynomial<T,S>::OneDimStorage OneDimStorage;
public:
    OneDimStorage operator()(
            OneDimStorage const & lhs,
            OneDimStorage const & rhs) {
        if (lhs.size() < rhs.size() )
            return operator()(rhs, lhs);
        OneDimStorage res(lhs);
        std::transform(rhs.begin(), rhs.end(), lhs.begin(),
                res.begin(),
                std::plus<typename Polynomial<T,S>::Coef>());
        return res;
    }
};

template <typename T, typename S>
Polynomial<T,S>& Polynomial<T,S>::operator+=(Polynomial<T,S> const & rhs) {
    int lengthInFirstDimDif = coefs.size() - rhs.coefs.size();
    if (lengthInFirstDimDif < 0) {
        typename Polynomial<T,S>::StorageT::const_iterator it(rhs.coefs.begin());
        std::advance(it, coefs.size());
        std::copy(it, rhs.coefs.end(), std::back_inserter(coefs));
        std::transform(
                rhs.coefs.begin(), it,
                coefs.begin(),
                coefs.begin(),
                OneDimAddition<T,S>()
        );
    } else {
        std::transform(
                rhs.coefs.begin(), rhs.coefs.end(),
                coefs.begin(),
                coefs.begin(),
                OneDimAddition<T,S>()
        );
    }
    this->countDegree();
    return *this;
}

template <typename T, typename S>
void oneDimOnScalarMultiply(typename Polynomial<T,S>::OneDimStorage & data,
        typename Polynomial<T,S>::Coef const & coef) {
    using namespace boost::lambda;
    std::for_each(data.begin(), data.end(), _1 *= coef);
}

template <typename T, typename S>
Polynomial<T,S>& Polynomial<T,S>::operator*=(
        typename Polynomial<T,S>::Coef const & coef) {
    using namespace boost::lambda;
    if (coef == typename Polynomial::Coef() ) {
        coefs.clear();
        countDegree();
    } else
        std::for_each(coefs.begin(), coefs.end(),
                 bind(oneDimOnScalarMultiply<T,S>, _1, coef));
    return *this;
}

template <typename T, typename S>
void oneDimOnMonomialMultiply(typename Polynomial<T,S>::OneDimStorage & data,
        int monomialDegree) {
    data.insert(data.begin(), monomialDegree,
                typename Polynomial<T,S>::Coef());
}

template <typename T, typename S>
Polynomial<T,S> Polynomial<T,S>::onMonomialMultiply(
        typename Polynomial::DegreeT const & monomialDegree) const {
    using namespace boost::lambda;
    Polynomial res(*this);
    std::for_each(res.coefs.begin(), res.coefs.end(),
            bind(oneDimOnMonomialMultiply<T,S>, _1,monomialDegree[1]));
    res.coefs.insert(res.coefs.begin(), monomialDegree[0], OneDimStorage());
    res.degree += monomialDegree;
    return res;
}

template <typename T, typename S>
typename Polynomial<T,S>::Coef
Polynomial<T,S>::ProxyPolySeqProduct::operator()(
        Point const & pt) const {
    typename Polynomial::Coef res = typename Polynomial<T,S>::Coef();
    for(size_t i = 0; i < poly.coefs.size(); ++i)
        for(size_t j = 0; j < poly.coefs[i].size(); ++j)
            res += poly.coefs[i][j]
                    * seq( Point(i, j) + pt - poly.getDegree() );
    return res;
}
/**************** End of Arithmetic operations *****************/

template <typename T, typename S>
typename Polynomial<T,S>::DegreeT
countDegree(typename Polynomial<T,S>::StorageT const & coefs);

template <typename T, typename S>
void Polynomial<T,S>::setCoefs(StorageT const & data) {
    coefs = data;
    countDegree();
}

template <typename T, typename S>
typename Polynomial<T,S>::ProxyPolySeqProduct Polynomial<T,S>::operator[](
        Sequence<S> const & seq) const {
    return typename Polynomial<T,S>::ProxyPolySeqProduct(*this, seq);
}

/*********** Degree-computing related routines **************/
template <typename T, typename S>
void Polynomial<T,S>::countDegree() {
    degree = findLastInSeq(coefs);
}
/******* End of Degree-coumting related routines **********/

/****************** Streaming operations ******************/
// format: [[a b c][e f]]
template <typename T, typename S>
std::istream& operator>>(std::istream& is, Polynomial<T,S> & poly) {
    typename Polynomial<T,S>::StorageT data;
    if (fillStorage(is, data))
        poly.setCoefs(data);
    return is;
}

template <typename T, typename S>
std::ostream& operator<<(std::ostream& os, Polynomial<T,S> const & poly) {
    print(os, poly.getCoefs());
    os << " degree: " << poly.getDegree();
    return os;
}
/*************** End of Streaming operations ****************/

} // namespace bmsa

\end{lstlisting}

\subsection*{GenericTwoDimIO.h}
\begin{lstlisting}
#ifndef _GENERICTWODIMIO_H
#define	_GENERICTWODIMIO_H

#include <iostream>
#include <ios>

namespace bmsa {

template <typename TwoDimStorage>
std::istream& fillStorage(std::istream& is, TwoDimStorage& st) {
    if (is.peek() == '[')
        is.get();
    else
        is.setstate(std::ios::failbit);
    while(is) {
        char nextCh = is.peek();
        if (nextCh == '[')
            is.get();
        else if (nextCh == ']') {
            break;
        }
        else {
            is.setstate(std::ios::failbit);
            break;
        }
        typename TwoDimStorage::value_type v;
        while(is.peek() != ']') {
            typename TwoDimStorage::value_type::value_type c;
            is >> c;
            v.push_back(c);
        }
        is.ignore(2, ']');
        st.push_back(v);
    }
    return is;
}

template <typename TwoDimStorage>
std::ostream& print(std::ostream& os, TwoDimStorage const & data) {
    os << '[';
    for(
            typename TwoDimStorage::const_iterator vit = data.begin();
            vit != data.end();
            ++vit
        ) {
        os << '[';
        for(
                typename TwoDimStorage::value_type::const_iterator
                    it = vit->begin();
                it != vit->end();
                ++it
            ) {
            os << *it;
            if ((it + 1) != vit->end())
                os << ' ';
        }
        os << ']';
    }
    os << ']';
    return os;
}
} // namespace bmsa
#endif	/* _GENERICTWODIMIO_H */
\end{lstlisting}

\subsection*{DeltaSet.h}
\begin{lstlisting}
#ifndef _DELTASET_H
#define	_DELTASET_H

#include <iostream>
#include <iterator>
#include <list>

#include <cstddef>

namespace bmsa {

// delta set of excluded points -- based on some delta set 
// which points and the current sequence segment 
// define the excluded points of four types;
// '(un)changed' below -- comparatively with the points of
// delta set on which this DS of excluded points is based on;
// type I  : both coordinates remain unchanged
// type II : both coordinates changed
// type III: first coordinate changed
// type IV : second coordinate changed
struct DeltaSetPoint {
    enum {I = 1, II, III, IV}; // possible point types

    size_t pointType;

    size_t i;

    size_t j;

    DeltaSetPoint(size_t pointType_, size_t i_, size_t j_) :
            pointType(pointType_), i(i_), j(j_) {}
};

inline
bool
operator<(DeltaSetPoint const & p1, DeltaSetPoint const & p2) {
     return p1.pointType < p2.pointType;
}

inline
std::ostream& operator<<(std::ostream& os, DeltaSetPoint const & pt) {
    os << "type: " << pt.pointType
        << " i=" << pt.i
        << " j=" << pt.j;
    return os;
}

typedef std::list<DeltaSetPoint> DeltaSet;

inline
std::ostream& operator<<(std::ostream& os, DeltaSet const & ds) {
    std::copy(ds.begin(), ds.end(),
            std::ostream_iterator<DeltaSetPoint>(os, "\n\t"));
    os << std::endl;
    return os;
}
} // namespace bmsa
#endif	/* _DELTASET_H */
\end{lstlisting}

\subsection*{Sequence.h}
\begin{lstlisting}
#ifndef _SEQUENCE_H
#define	_SEQUENCE_H

#include <iostream>
#include <vector>

#include "ElementType.h"
#include "GenericTwoDimIO.h"
#include "Point.h"

namespace bmsa {

template<typename ElemType = NTL::GF2>
class Sequence {
public:
    typedef ElemType ElemT;
    typedef std::vector< std::vector<ElemT> > StorageT;

    Sequence(std::istream& is) :
            DEFAULT_VAL(ElemTypeTraits<ElemT>::Null()) 
        {is >> *this;}

    Point size() const {return size_;}

    ElemT const & operator()(Point const &) const;
private:
    StorageT rep;

    Point size_;

    const ElemT DEFAULT_VAL;

    template <typename T>
    friend
    std::istream& operator>>(std::istream& is, Sequence<T>& seq);

    template <typename T>
    friend
    std::ostream&
    operator<<(std::ostream& os, Sequence<T> const & seq);
}; // class Sequence

template <typename T>
typename Sequence<T>::ElemT
const & Sequence<T>::operator()(Point const & pt) const {
    if (pt[0] > static_cast<int>(rep.size())
            || pt[1] > static_cast<int>(rep[pt[0]].size()))
        return DEFAULT_VAL;
    return rep[pt[0]][pt[1]];
}

template<typename T>
inline
std::istream& operator>>(std::istream& is, Sequence<T>& seq) {
    typename Sequence<T>::StorageT st;
    if (fillStorage(is, st))
        seq.rep = st;
    seq.size_ = ++findLastInSeq(seq.rep);
    return is;
}

template<typename T>
inline
std::ostream& operator<<(std::ostream& os, 
                         Sequence<T> const & seq) {
    print(os, seq.rep);
    return os;
}

} // namespace bmsa

#endif	/* _SEQUENCE_H */
\end{lstlisting}

\subsection*{Point.h}
\begin{lstlisting}
#ifndef _POINT_H
#define	_POINT_H

#include <algorithm>
#include <cassert>
#include <cstddef>
#include <iostream>
#include <numeric>
#include <vector>
#include <tr1/array>

#include <boost/lambda/lambda.hpp>

namespace bmsa {

class Point {
    static const size_t DIMENSION = 2;

    typedef std::tr1::array<int, DIMENSION> ComponetsStorageT;
    ComponetsStorageT data;

public:
    static Point nil;

    Point() {
        data[0] = 0;
        data[1] = 0;
    }

    Point(int x, int y) {
        data[0] = x;
        data[1] = y;
    }

    int weight() const {
        return std::accumulate(data.begin(), data.end(), 0);
    }

    //subscript operators
    ComponetsStorageT::reference 
    operator[](ComponetsStorageT::size_type n) {
        assert( 0 <= n && n < DIMENSION);
        return data[n];
    }

    ComponetsStorageT::const_reference operator[](
            ComponetsStorageT::size_type n) const {
        assert( 0 <= n && n < DIMENSION);
        return data[n];
    }

    inline
    bool operator==(Point const & rhs) const {
        return (data == rhs.data);
    }

    Point& operator+=(Point const & rhs) {
        data[0] += rhs.data[0];
        data[1] += rhs.data[1];
        return *this;
    }

    Point& operator-=(Point const & rhs) {
        data[0] -= rhs.data[0];
        data[1] -= rhs.data[1];
        return *this;
    }

    Point& operator++();

    Point operator++(int);

    friend      //partial ordering
    bool byComponentLessOrEqual(Point const &, Point const &);

    friend
    std::ostream& operator<<(std::ostream& os, Point const &);
};

inline
bool operator!=(Point const & lhs, Point const & rhs) {
    return !(lhs == rhs);
}

// by component less
inline
bool operator<=(Point const & lhs, Point const & rhs) {
    return lhs[0] <= rhs[0] && lhs[1] <= rhs[1];
}

inline
bool operator<(Point const & lhs, Point const & rhs) {
    return (lhs <= rhs) && (lhs != rhs);
}

inline
bool totalLess(Point const & lhs, Point const & rhs) {
    int lhsPow = lhs.weight();
    int rhsPow = rhs.weight();
    if (lhsPow == rhsPow)
        return lhs[0] > rhs[0];
    else
        return lhsPow < rhsPow;
}

inline
bool totalLessOrEqual(Point const & lhs, Point const & rhs) {
    return lhs == rhs || totalLess(lhs, rhs);
}

inline
Point operator+(Point lhs, Point const & rhs) {
    // lhs: pass-by-copy optimization
    return lhs += rhs;
}

inline
Point operator-(Point lhs, Point const & rhs) {
    return lhs -= rhs;
}

template< typename OneDimStorage >
int lastInOneDim(OneDimStorage const & data) {
    using namespace boost::lambda;
    int deg = data.rend()
                - std::find_if(data.rbegin(), data.rend(),
                    _1 != typename OneDimStorage::value_type())
                - 1;
    return deg > -1 ? deg : 0;
}

template< typename TwoDimStorge >
Point findLastInSeq(TwoDimStorge storage) {
    if (storage.empty())
        return Point();
    typedef std::vector<Point> PtStorage;
    PtStorage points;
    int i = 0;
    for(
            typename TwoDimStorge::const_iterator vit = storage.begin();
            vit != storage.end();
            ++vit, ++i ) {
        int lessDimDegree = lastInOneDim(*vit);
        points.push_back(Point(i, lessDimDegree));
    }

    return
        //using total ordering on Point type:
        *(std::max_element(points.begin(), points.end(), totalLess));
}

} // namespace bmsa
#endif	/* _POINT_H */
\end{lstlisting}

\subsection*{Point.cc}
\begin{lstlisting}
#include <functional>
#include <numeric>
#include <sstream>

#include <boost/lambda/lambda.hpp>

#include "Point.h"

namespace bmsa {

Point Point::nil = Point();

inline
bool
byComponentLessOrEqual(Point const & lhs, Point const & rhs) {
    using namespace boost::lambda;
    return
        std::inner_product(lhs.data.begin(), lhs.data.end(), 
                rhs.data.begin(),
                true, // init value
                std::logical_and<bool>(),
                _1 <= _2);
}

std::ostream& operator<<(std::ostream& os, Point const & pt) {
        os << '(';
        std::ostringstream oss;
        copy(pt.data.begin(), pt.data.end(),
                std::ostream_iterator<int>(oss, ","));
        std::string s = oss.str();
        std::string ss(s.begin(), --(s.end()));
        os << ss << ')';
        return os;
}

Point& Point::operator++() {
    if (data[0] == 0) {
        data[0] = data[1] + 1;
        data[1] = 0;
    } else {
        --data[0];
        ++data[1];
    }
    return *this;
}

Point Point::operator++(int) {
    Point old(*this);
    ++*this;
    return old;
}

} // namespace bmsa
\end{lstlisting}

\subsection*{ElementType.h}
\begin{lstlisting}
#ifndef _ELEMENTTYPE_H
#define	_ELEMENTTYPE_H

namespace bmsa {

template <typename T>
struct ElemTypeTraits {
    static T id() {
        T res = T();
        return ++res;
    }

    static T Null() {
        return T();
    }
};
} // namespace bmsa
#endif	/* _ELEMENTTYPE_H */
\end{lstlisting}

