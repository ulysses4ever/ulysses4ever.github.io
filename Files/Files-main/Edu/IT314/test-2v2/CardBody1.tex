\begin{enumerate}
    \itemsep=\myitemsep
    \item Расставить скобки, нарисовать дерево и выполнить полную редукцию
    терма:
    \[
        (\Lvar{xy} x (\Lvar{z} (yz))) (\Ly yy) z y
    \] 
    Для каждой подстановки в процессе редукции указать номер используемого
    пункта определения операции подстановки.
    \item Выполнить редукцию, используя нормальный порядок и вызов по значению
    (при подстановке можно не указывать номер используемого правила)
    \[
        (\Lvar{xy} x x y) (\Lvar{xy} x x y) 
            ((\Lvar{xy} x y y) (\Lvar{xy} x y y))
    \]
    \item Вычислить:
    $$(\Lx \Lif {\isZero (\fst x)} {4} {(\snd x) + 3})(2, 3).$$
    Все вычисления проводятся с помощью редукции соответствующих \L-термов.
    \item Вычислить в комбин\'{а}торной логике:
    $\clS(\clS(\clK\clS)(\clS(\clK\clK)\clK))(\clK(\clS\clK\clK))XY$.
    %S(S(KS)(S(KK)K))(K(SKK))X = KX
    \item Дать рекурсивное определение функции \texttt{sumEven}, вычисляющей
    сумму чётных чисел от 1 до $n$. Записать с помощью
    комбинатора неподвижной точки соответствующий \L-терм. Вычислить
    $\text{\ttfamily sumEven } 3$.
    \emph{Указание}: операцию вычитания можно не проделывать в \L-термах. 
\end{enumerate}