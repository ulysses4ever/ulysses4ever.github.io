Помните, что отсутствие кода освобождения динамической памяти является грубой
ошибкой. Работоспособность решений всех заданий необходимо демонстрировать в
основной программе. Решения размещаются в различные файлы, как на
практических занятиях.
\begin{enumerate}
    \itemsep=\myitemsep
    \item Создайте структуру \texttt{Complex} комплексного числа, хранящую два
    поля типа \texttt{double}: вещественную и мнимую часть. Создайте функции
    печати и ввода с клавиатуры для неё.
    
    \item Создайте функцию \texttt{ReadComplexArray(n)}, которая 
    возвращает адрес массива \texttt{Complex} размера \texttt{n},
    размещённого в динамической памяти, данные для него вводятся с
    клавиатуры.
    
    \item Создайте структуру для представления узла односвязного списка,
    хранящего комплексные числа, и функции работы с таким списком (добавление в
    голову, удаление, печать).
    
    \item Создайте функцию \texttt{toList}, принимающую массив экземпляров
    \texttt{Complex} и возвращающую список, заполненный копиями элементов этого
    массива.

    \item Создайте функцию сортировки массива C-строк
    методом выбора. (Описание алгоритма можно посмотреть в английском разделе
    Википедии: Selection sort.)
\end{enumerate}