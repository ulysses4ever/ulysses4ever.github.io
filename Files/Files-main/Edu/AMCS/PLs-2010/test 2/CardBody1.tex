Для обработки массивов (в том числе, C-строк) использовать
арифметику указателей. Решения оформлять в виде функций в отдельном
cpp\nbdash{}файле.
\begin{enumerate}
    \itemsep=\myitemsep
    \item Дан целочисленный массив размера $N (> 2)$. Заполонить его первыми $N$
    членами последовательности Фибоначчи $F_K$ (%
    $F_1 = 1$; $F_2 = 1$; $F_K = F_{K-2} + F_{K-1}$, $K = 3, 4, \ldots$).
    
    \item Дан массив размера $N$ и целые числа $K$ и $L$
    ($0 \leqslant K \leqslant L < N$). Найти сумму всех элементов массива, кроме
    элементов с номерами от $K$ до $L$ включительно.
    
    \item Дана C-строка и два символа, $c_1, c_2$ ($c_1 \leqslant c_2$).
    Распечатать на консоль все символы данной строки, ASCII\nbdash{}коды которых
    находятся между кодами $c_1$ и $c_2$. Порядок символов должен оставаться
    таким же, как в исходной строке.
    
    \item Дана строка, состоящая из слов, разделенных пробелами (одним
    или несколькими). Найти длину самого короткого слова.
    
    \item Создайте функцию \texttt{copyIf}, которая копирует из
    заданного входного массива в заданный выходной массив все элементы,
    удовлетворяющие заданному предикату. Массивы — целочисленные. Функция должна
    возвращать целое число — логическую длину выходного массива. С помощью этой
    функции получите копию массива, содержащую только чётные элементы исходного.
\end{enumerate}